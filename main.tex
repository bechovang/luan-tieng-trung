% !Mode:: "TeX:UTF-8"
% 
% 编译说明
%		使用 XeLaTeX 编译引擎
%       TeX Live 的版本 2022 和 2021 目前都可以成功编译
%       如果使用 Overleaf 进行写作,请在左上角菜单中选择编译器为 XeLaTeX
%		在 MacBook 中推荐使用编译软件 Texpad


\documentclass[doctor]{bnuthesis}
	% bachelor/master/doctor: 学士/硕士/博士学位论文
	% twoside: 奇数页和偶数页页边距不一样,用于打印
	\usepackage{bnutils}
	% 调用宏包,使用 GB/T 7714-2015 参考文献著录格式
	\usepackage{gbt7714}
	
	% 添加脚注相关包
	% \usepackage{footnote}
	% \usepackage[perpage=false]{footmisc}
	
	% Override document class footnote system for continuous numbering
	\makeatletter
	% Disable per-page reset by redefining the footnote counter
	\newcounter{globalfootnote}
	\setcounter{globalfootnote}{0}
	\renewcommand{\thefootnote}{\arabic{globalfootnote}}
	\renewcommand{\thempfootnote}{\arabic{globalfootnote}}
	\makeatother
	
	% 确保natbib可用
	% \usepackage{natbib}
	% \usepackage{bibentry}
	% \nobibliography*
	
	% 创建引用跟踪系统 - 使用更简单的方法
	\makeatletter
	% 存储已使用的引用
	\def\@usedcites{}
	
	% 检查引用是否已经使用过
	\newcommand{\@checkusedcite}[1]{%
		\@ifundefined{@cite@#1}{%
			% 引用未使用过,创建新的
			\expandafter\gdef\csname @cite@#1\endcsname{\theglobalfootnote}%
			\stepcounter{globalfootnote}%
			\footnote{\bibentry{#1}}%
		}{%
			% 引用已使用过,使用已有的编号
			\csname @cite@#1\endcsname%
		}%
	}
	
	% 新的footcite命令,支持重复引用
	\newcommand{\footcite}[1]{%
		\@checkusedcite{#1}%
	}
	
	% 带说明文字的footcite
	\newcommand{\footcitewith}[2]{%
		\@ifundefined{@cite@#1}{%
			% 引用未使用过,创建新的
			\expandafter\gdef\csname @cite@#1\endcsname{\theglobalfootnote}%
			\stepcounter{globalfootnote}%
			\footnote{#2 \cite{#1}}%
		}{%
			% 引用已使用过,使用已有的编号
			\csname @cite@#1\endcsname%
		}%
	}
	
	% 多个参考文献的footcite
	\newcommand{\footcitemulti}[1]{%
		\footnote{%
			\cite{#1}%
		}%
	}
	\makeatother
	
	\begin{document}
	\graphicspath{{figures/}}
	
	\frontmatter
	% !Mode:: "TeX:UTF-8"

\ctitle{越南高等教育外部质量保证体系研究}
\etitle{Research on External Quality Assurance System of Vietnamese Higher Education}


\makeatother

% 学士学位封面
% \cbuyuanxi{天文系} % 部院系
% \czhuanye{天文学} % 专业
% \cxuehao{201511160109} % 学号
% \cxueshengxingming{某某某} % 学生姓名
% \czhidaojiaoshi{某某} % 指导教师
% \czhidaojiaoshizhicheng{教授} % 指导教师职称
% \czhidaojiaoshidanwei{北京师范大学天文系} % 指导教师单位

% 博士(硕士)学位封面
\czuozhe{张氏越祯} % 作者
\cdaoshi{高益民\ 教授} % 导师
\cxibienianji{教育学部2019级} % 系别年级
\cxuehao{201939010022} % 学号
\cxuekezhuanye{比较教育学} % 学科专业
\cwanchengriqi{2025年7月} % 完成日期


% 中文摘要
\begin{cabstract}
本论文系统研究了越南高等教育外部质量保障体系在快速扩张、国际化和问责要求不断提高背景下的转型与挑战。研究背景源于越南高等教育在过去二十年间的迅猛发展:从2000年的约200所高校、100万学生,发展到2023年的近500所高校、超过250万学生。这种规模扩张在带来机遇的同时,也带来了质量保障的严峻挑战,特别是在确保教育质量与数量增长同步、满足劳动力市场需求、以及应对国际竞争压力等方面。

通过全面梳理全球与区域发展趋势,论文揭示了推动外部质量保障(EQA)机制发展的主要动力,特别是在东盟区域内。研究发现,全球高等教育质量保障呈现出从单一合规导向向多元化、适应性模式转变的趋势。传统的以政府主导、标准化评估为主的质量保障模式正在被更加灵活、注重持续改进和利益相关者参与的混合模式所替代。在东盟地区,这种趋势表现得尤为明显,各国在保持本土特色的同时,也在积极寻求区域合作与标准协调。

研究采用定性比较案例研究方法,深入分析了越南EQA体系的现状、成就与挑战。通过对越南教育部质量保障政策文件、评估报告和专家访谈的系统分析,研究识别出越南EQA体系在制度建设、评估机制、国际合作等方面的主要成就。同时,研究也揭示了体系存在的关键问题,包括评估标准不够细化、评估专家队伍能力不足、评估结果应用不充分、以及利益相关者参与度不高等。

为了提供更全面的分析视角,研究借鉴了中国国家主导模式和东盟大学联盟质量保障(AUN-QA)框架的经验。中国案例展示了国家强力主导的质量保障体系在实现大规模、快速变革方面的优势,但也凸显了如何在严格控制与学术创新自主权之间取得平衡的挑战。AUN-QA案例则体现了基于合作、共识和同行网络的区域质量保障模式的力量,证明了在多样化环境中无需单一权力机构集中控制也能建立共同标准。

论文运用现代管理与组织理论,包括新制度主义、利益相关者理论和委托代理理论,构建了适合越南国情的V-AQA混合与适应性模型。该模型基于五个核心要素构建:(1)领导与治理,强调战略导向和有效治理结构;(2)质量文化,注重建立持续改进的组织文化;(3)利益相关者参与,确保多方利益的有效整合;(4)内部流程,建立系统化的质量保障机制;(5)合作与协调,促进内部协作和外部合作。这五个要素相互关联、动态互动,形成了一个完整的质量保障生态系统。

研究结果指出,越南高等教育在规模扩张、资源分配和人才培养与劳动力市场需求之间仍面临诸多挑战。具体表现在:教育质量与数量增长不同步,优质教育资源分布不均,人才培养与就业市场需求脱节,以及国际竞争力有待提升等方面。这些挑战的根源在于质量保障体系的不完善,特别是在评估标准、专家队伍、结果应用和利益相关者参与等方面存在不足。

基于理论分析和国际比较,论文最终提出了以领导力、质量文化、利益相关方参与、内部流程和合作为核心的EQA体系综合改革模型。该模型强调:(1)建立多层次、协调统一的质量保障治理体系;(2)培育以质量为核心的组织文化;(3)构建多元利益相关者参与机制;(4)完善系统化的内部质量保障流程;(5)加强国内协作和国际合作。这一改革模型旨在提升体系效能,促进质量持续改进,助力越南高等教育更好地融入区域与全球教育体系。

研究的理论贡献在于构建了适合发展中国家高等教育质量保障的分析框架,丰富了质量保障理论在转型国家背景下的应用。实践意义在于为越南高等教育质量保障体系改革提供了系统性的政策建议,同时也为其他类似背景的国家提供了有价值的参考。研究结论强调,有效的质量保障体系不仅是一套流程的集合,而是一个复杂的生态系统,需要领导力、文化、利益相关者、流程和合作之间的动态平衡。
\end{cabstract}
\ckeywords{质量保障;高等教育;越南;外部质量保障;改革模型}

% English abstract
\begin{eabstract}
This dissertation investigates the transformation and challenges of Vietnam's higher education quality assurance system in the context of rapid expansion, international integration, and increasing demands for accountability. The research background stems from Vietnam's remarkable higher education development over the past two decades: from approximately 200 institutions and 1 million students in 2000 to nearly 500 institutions and over 2.5 million students in 2023. While this scale expansion has brought opportunities, it has also created serious quality assurance challenges, particularly in ensuring synchronized quality and quantity growth, meeting labor market demands, and responding to international competitive pressures.

Through a comprehensive review of global and regional trends, the study identifies the driving forces behind the development of external quality assurance (EQA) mechanisms, particularly in the ASEAN region. The research reveals that global higher education quality assurance is transitioning from single compliance-oriented approaches to diversified, adaptive models. Traditional government-led, standardized assessment quality assurance models are being replaced by more flexible approaches that emphasize continuous improvement and stakeholder engagement. This trend is particularly evident in the ASEAN region, where countries maintain their local characteristics while actively seeking regional cooperation and standard harmonization.

Using a qualitative comparative case study design, the research analyzes Vietnam's EQA system in depth, examining its current status, achievements, and challenges. Through systematic analysis of Vietnam's Ministry of Education quality assurance policy documents, evaluation reports, and expert interviews, the study identifies major achievements in Vietnam's EQA system regarding institutional development, evaluation mechanisms, and international cooperation. Simultaneously, the research reveals critical issues within the system, including insufficiently detailed evaluation standards, inadequate capacity of evaluation expert teams, insufficient application of evaluation results, and low stakeholder participation levels.

To provide a more comprehensive analytical perspective, the study draws lessons from China's state-driven model and the ASEAN University Network Quality Assurance (AUN-QA) framework. The China case demonstrates the advantages of a state-dominated quality assurance system in achieving large-scale, rapid transformation, but also highlights the challenge of balancing strict control with academic innovation autonomy. The AUN-QA case exemplifies the power of a regional quality assurance model based on cooperation, consensus, and peer networks, proving that common standards can be established in diverse environments without centralized control by a single authority.

The study applies modern management and organizational theories—including new institutionalism, stakeholder theory, and principal-agent theory—to construct the V-AQA hybrid and adaptive model, tailored to Vietnam's unique context. This model is built on five core elements: (1) Leadership and Governance, emphasizing strategic orientation and effective governance structures; (2) Quality Culture, focusing on establishing continuous improvement organizational culture; (3) Stakeholder Engagement, ensuring effective integration of multiple interests; (4) Internal Processes, establishing systematic quality assurance mechanisms; and (5) Cooperation and Coordination, promoting internal collaboration and external cooperation. These five elements are interconnected and dynamically interactive, forming a complete quality assurance ecosystem.

Findings highlight both achievements and persistent challenges, such as rapid scale growth, uneven resource distribution, and the gap between training outcomes and labor market needs. Specific manifestations include: asynchronous quality and quantity growth in education, uneven distribution of quality educational resources, disconnection between talent cultivation and employment market demands, and room for improvement in international competitiveness. The root causes of these challenges lie in the imperfection of the quality assurance system, particularly in areas such as evaluation standards, expert teams, result application, and stakeholder participation.

Based on theoretical analysis and international comparison, the dissertation concludes by proposing a comprehensive reform model for Vietnam's EQA system, emphasizing leadership, quality culture, stakeholder engagement, internal processes, and cooperation. This model emphasizes: (1) establishing a multi-level, coordinated quality assurance governance system; (2) cultivating quality-centered organizational culture; (3) constructing multi-stakeholder participation mechanisms; (4) perfecting systematic internal quality assurance processes; and (5) strengthening domestic collaboration and international cooperation. This reform model aims to enhance the system's effectiveness, support sustainable quality improvement, and facilitate Vietnam's integration into the regional and global higher education landscape.

The theoretical contribution of this research lies in constructing an analytical framework suitable for higher education quality assurance in developing countries, enriching the application of quality assurance theory in the context of transitional countries. The practical significance lies in providing systematic policy recommendations for Vietnam's higher education quality assurance system reform, while also offering valuable references for other countries with similar backgrounds. The research conclusion emphasizes that an effective quality assurance system is not merely a collection of processes, but a complex ecosystem requiring dynamic balance among leadership, culture, stakeholders, processes, and cooperation.
\end{eabstract}
\ekeywords{quality assurance; higher education; Vietnam; external quality assurance; reform model}



	\makecover
	
	\tableofcontents
	\listoffigures % 插图索引
	\listoftables % 表格索引
	
	\mainmatter
	Đây là một câu có,cccccccccccccc, trích dẫn,dmmmmmmmm, trong footnote\footcite{smith2020}, 

	haiiiiiiiiii222222222222


% ======================================================================
% Chương 2
% ======================================================================
\chapter{高等教育质量保障的理论基础}
\label{chap:ly_luan}

% ======================================================================
% TRANG 1-2: GIỚI THIỆU CHƯƠNG
% ======================================================================
\section*{引言}
\addcontentsline{toc}{section}{引言}

在全球化和知识经济崛起的背景下,高等教育(GDĐH)的质量已成为决定每个国家竞争力的一个战略性因素。大学不再是与世隔绝的“象牙塔”,而已成为活跃的主体,受到来自社会、经济和政治等多重复杂压力的影响。为应对这些挑战,世界各地纷纷建立和发展了质量保障(ĐBCL)体系,特别是外部质量保障(External Quality Assurance - EQA)体系。然而,将发达国家的质量保障模式应用于像越南这样的转型经济体背景时,常会遇到诸多困难和矛盾。那些侧重于合规性控制或僵化标准的传统模式,显得不够灵活,难以解释和引导一个正处于快速发展和深刻变革阶段的高等教育体系。

这种复杂性对理论提出了一个迫切的要求:需要一个足够强大和全面的分析框架,以便能够“解剖”高等教育质量保障体系的本质。这样的理论框架不仅要能识别出影响因素,还必须能解释其动因、权力关系以及各主要主体——国家、认证机构和大学之间潜在的矛盾。现实表明,以往的许多研究通常只关注单一层面,或使用单一理论,导致了片面的看法。例如,一些研究可能很好地解释了为什么大学必须遵守国家规定,但却无法解释为什么培养方案在响应企业需求方面仍然变化缓慢。同样,一些分析侧重于问责制,却忽略了对组织行为有深远影响的文化和无形规范等因素。

这一“理论空白”正是本章的出发点。本论文认为,为了获得全面的理解,需要整合多种理论视角。具体而言,一个有效的分析框架必须能够同时解释三个核心问题:(1)为什么各大学会采用相似的质量保障结构和流程(\textit{关于合法性的问题})?(2)质量为谁定义和创造,以及如何平衡不同群体的利益(\textit{关于利益相关者的问题})?(3)采用何种机制来确保大学履行其承诺和责任(\textit{关于问责制的问题})?

为填补这一空白,本章将进行系统性的构建。\textbf{首先},本章将深入分析社会科学与公共管理的三大经典理论支柱:新制度主义、利益相关者理论和委托代理理论。分析将不仅停留在概念介绍,还将批判并指出每种理论在独立应用于高等教育领域时的局限性。\textbf{其次},本章将综述全球现代质量保障的发展趋势,特别是混合模型(Hybrid Model)和适应性框架(Adaptive Framework)的兴起,这些都是旨在超越传统理论局限的实践努力。\textbf{最后,也是最重要的},在这些分析的基础上,本章将提出并详细论证一个新的理论模型——\textbf{越南高等教育混合与适应性质量保障模型(V-AQA Model)}。该模型将作为核心的理论工具,为本论文后续章节的现状分析和方案建议提供指导和视角。


% done chuong 2 goi 1


% ======================================================================
% TRANG 3-5: THUYẾT TÂN THỂ CHẾ (PHẦN 1)
% ======================================================================
\section{质量保障体系分析的基础理论框架}
\label{sec:khung_ly_thuyet_nen_tang}

在质量保障体系中,政府、认证机构和大学之间的复杂关系可以通过多种理论视角来审视。以下三种理论提供了强大且互补的分析工具,有助于解读质量保障政策与实践背后的动因\footcite{OxfordResearch}\footcite{GovernanceTheories}\footcite{SAGE_HE}。

\subsection{新制度主义理论 (New Institutionalism Theory)}
\label{subsec:tan_the_che_nen_teng}

\subsubsection{起源与发展历史}
新制度主义(New Institutionalism)兴起于1970和1980年代,作为对纯理性经济学和组织理论的一种回应,后者认为组织行为主要由技术效率和利益最大化等因素决定。John W. Meyer、Brian Rowan,以及后来的Paul DiMaggio和Walter Powell等先驱思想家认为,组织,特别是像学校、医院这样的公共部门组织,不仅存在于经济市场中,也存在于一个更广泛的社会和文化环境中\footcite{MeyerRowan1977}\footcite{DiMaggioPowell1983}。

“旧制度主义”与“新制度主义”的核心区别在于分析的重点。旧制度主义(old institutionalism)通常关注法律、宪法和组织结构图等正式、有形的结构。相反,新制度主义(new institutionalism)将“制度”的概念扩展到包括不成文规则、社会规范、符号、认知脚本和文化模式等无形但影响强大的因素\footcite{MeyerPowell2020}。据此,一个组织之所以以某种方式行事,不仅因为法律要求(规制性因素),还因为“这是正确的做法”(规范性因素),或者仅仅因为“大家都是这么做的”(文化-认知性因素)。该理论提供了一个强大的视角来解释为什么在同一领域内的组织,例如大学,倾向于发展出极其相似的结构、流程和实践,这一现象被称为“制度同形”。

\subsubsection{核心概念及其在越南高等教育中的应用}

要将此理论应用于分析越南高等教育的质量保障体系,需要阐明三个核心概念:

\paragraph{制度场域 (Institutional Field):}
一个“制度场域”被定义为共同构成一个公认的制度生活领域的组织集合(如供应商、消费者、监管机构)\footcite{DiMaggioPowell1983}。在越南高等教育的背景下,制度场域包括一个复杂的主体网络:
\begin{itemize}
    \item \textbf{国家管理机构:} 教育培训部、其他主管部委以及地方教育培训厅。
    \item \textbf{高等教育机构:} 国家大学、区域性大学、公立大学、私立大学以及有外国因素的大学。
    \item \textbf{质量保障组织:} 国内的教育质量认证中心(如VNU-CEA)以及国际/区域性认证组织(如AUN-QA, HCERES, FIBAA)。
    \item \textbf{其他利益相关方:} 企业(雇主)、行业协会、学生、家长以及整个社会。
\end{itemize}
所有这些组织相互作用,共享一套关于“质量”的规则、规范和定义,创造了一个塑造每所大学行为的制度环境。

\paragraph{合法性 (Legitimacy):}
合法性是社会对一个组织行为的广泛认可,认为其在社会建构的规范、价值、信念和定义体系中是“可取的、正确的或适当的”\footcite{Suchman1995}。对于一所越南大学来说,获得并维持合法性至关重要,因为它能带来具体的利益:
\begin{itemize}
    \item \textbf{获取资源:} 获得质量认可的学校更容易获得国家预算、吸引企业项目和其他资助。
    \item \textbf{吸引优秀学生:} 来自认证机构的声誉和认可是吸引学生和家长的关键因素。
    \item \textbf{稳定与生存:} 在竞争环境中,被视为“合法”有助于学校巩固地位并确保可持续生存。
\end{itemize}
因此,参与认证等质量保障活动不仅是一项技术活动,也是寻求和巩固合法性的重要战略。

\paragraph{制度同形 (Institutional Isomorphism):}
这是指同一制度场域内的组织变得越来越相似的过程。DiMaggio和Powell(1983)指出了导致同形的三个主要机制,这三个机制在越南高等教育的质量保障体系中都可以清晰地观察到\footcite{DiMaggioPowell1983}。

\textbf{1. 强制性同形 (Coercive Isomorphism):} 这是来自一个组织所依赖的其他组织的正式和非正式压力的结果。在越南的背景下,这是最强大的机制,体现在:
\begin{itemize}
    \item \textit{法律规定:} 《高等教育法》、教育培训部的法令和通知要求各大学必须每五年参加一次质量认证。不遵守规定可能导致不被批准开设新专业或减少招生名额等制裁。
    \item \textit{主管机构的压力:} 主管部委或省人民委员会也对质量和排名提出要求,迫使下属学校遵守一个共同的框架。
    \item \textit{资源依赖:} 质量保障的要求通常与国家预算的分配或参与国家重点项目(如关于培养博士师资的89号提案)挂钩。
\end{itemize}
其结果是,越南大多数大学都必须建立质量保障单位,按照国家规定的统一脚本执行自评过程并参与外部认证。


% done chuong 2 goi 1 v2


Chắc chắn rồi, đây là nội dung đã được dịch sang tiếng Trung và định dạng lại theo yêu cầu của bạn:

% ======================================================================
% TRANG 6-7: THUYẾT TÂN THỂ CHẾ (PHẦN 2)
% ======================================================================

\paragraph{2. 模仿性同形 (Mimetic Isomorphism):}
该机制产生于组织面临不确定性(uncertainty)之时。当目标不明确或环境过于复杂时,组织倾向于模仿同一领域内被它们认为更成功或更具合法性的其他组织\footcite{DiMaggioPowell1983}。这是一种降低风险并快速获得认可的策略。在越南高等教育的背景下,这种现象非常普遍:
\begin{itemize}
    \item \textit{培养方案的模仿:} 许多新成立的大学或地方院校在构建其培养方案时,通常会参考甚至复制顶尖大学(如河内国家大学、胡志明市国家大学或河内理工大学)的课程结构。这种行为不仅有助于节省开发课程的时间和资源,更重要的是,它创造了一种“安全”和“正确”的感觉,因为它们正走在成功者的道路上。
    \item \textit{治理模式的模仿:} 质量管理模式的应用、质量保障部门的结构设置或内部报告系统等,通常是各学校相互学习的对象。一个在某所大学成功实施的模式会迅速成为其他学校效仿的“典范”,从而在管理实践中掀起一股同步化的浪潮。
\end{itemize}
关于“何为国际一流大学”或“何为有效的质量保障体系”的不确定性,正是模仿性同形得以蓬勃发展的沃土。

\paragraph{3. 规范性同形 (Normative Isomorphism):}
该机制主要源于专业化(professionalization)过程\footcite{DiMaggioPowell1983}。当一个行业发展时,它会通过以下渠道形成一套共同的规范、价值观和工作方法:
\begin{itemize}
    \item \textit{专业培训:} 这是在质量保障领域影响最强的渠道。越南各大学的认证专家、质量保障负责人通常会参加由国家认证中心或东盟大学网络(AUN)等区域组织举办的培训课程。在这些课程中,他们学习同一套标准(例如,AUN-QA标准),实践同一套评估方法。回到学校后,他们带来并应用这些共同的规范,逐渐使不同学校的质量保障实践变得相似\footcite{AUN-QAGuide}。
    \item \textit{专业网络:} 专家网络、协会(如越南大学与学院协会)以及关于质量保障的科学研讨会的发展,创造了一个让职业规范得以传播和巩固的空间。“最佳实践”(best practices)被分享并迅速成为全行业的非官方标准。
    \item \textit{人员招聘与流动:} 招聘在先进教育体系中受过培训的教师,或管理人员在各校之间的调动,也是传播管理和质量保障规范的一个渠道。
\end{itemize}

\subsubsection{新制度主义理论的应用与批判}

通过新制度主义的视角来分析越南高等教育的质量保障体系可以发现,该体系的形成和运作是一个复杂的过程,由三种同形机制的相互作用所塑造。各大学不仅单纯遵守教育培训部的规定(强制性),还主动学习其他学校的模式(模仿性),并受到质量保障专家社群共同规范的影响(规范性)。

然而,对新制度主义最大的批判之一是其倾向于过分强调合规性和环境压力,有时会轻视组织自身的主动性和创新能力(institutional agency)。该理论很好地解释了为什么组织会变得相似,但却较少解释为什么某些组织能够采取突破性、创造性的举措,或者有能力进行“脱钩”(decoupling)——即形式上采纳所要求的结构和流程以获得合法性,而内部核心活动却保持不变\footcite{MeyerRowan1977}。

此外,新制度主义未能提供一个足够强大的工具来分析大学如何处理和平衡来自同一制度场域内不同群体的矛盾压力。例如,来自政府的压力可能是提升服务地方产业的能力,而来自国际学术界的压力则是增加在ISI/Scopus期刊上的发表。要理解这种拉锯和利益协商过程,我们需要借助另一种理论视角:利益相关者理论。

\subsection{利益相关者理论 (Stakeholder Theory)}
\label{subsec:ben_lien_quan_nen_tang}

\subsubsection{起源与核心原则}

由R. Edward Freeman在其经典著作《战略管理:一种利益相关者的方法》(1984)中系统发展的利益相关者理论,在管理思维上掀起了一场革命。该理论作为对传统管理模式(仅关注股东利益)的回应而诞生。Freeman认为,一个组织的可持续成功,不能仅通过最大化股东利润来实现,而必须为所有能够影响或被组织目标实现所影响的个人和群体创造并分配价值。这些个人和群体被称为“利益相关者”(stakeholders)\footcite{Freeman1984}。

该理论已被广泛扩展并应用于不存在“股东”概念的公共和非营利组织中。在高等教育领域,它尤为适用,因为大学有众多利益诉求多样且时而冲突的利益相关者\footcite{Langrafe2020}。后续的研究系统化了基于该理论的治理核心原则,包括\footcite{LangrafeEUR2020}\footcite{IJLTER2024}:
\begin{enumerate}
    \item \textbf{参与 (Participation):} 利益相关者需要积极参与到影响他们的决策过程中。
    \item \textbf{信息交换 (Information Exchange):} 需要有机制让组织倾听利益相关者的要求,同时透明地通报其活动和决策。
    \item \textbf{信任 (Trust):} 建立基于相互信任和尊重的关系是有效合作的基础。
    \item \textbf{战略规划 (Strategic Planning):} 利益相关者的利益和要求必须被整合到组织的战略规划过程中。
\end{enumerate}

\subsubsection{越南高等教育质量保障中的利益相关者地图}

要将此理论应用于具体情境,首要且最重要的一步是识别(mapping)越南高等教育质量保障体系中的利益相关者并分析他们的利益(stake)。这样的地图有助于明确质量保障体系需要服务的对象群体。
\begin{figure}[h!]
    \centering
    \includegraphics[width=\textwidth]{image/stakeholder_map.jpg} 
    \caption{越南高等教育质量保障体系中的利益相关者地图}
    \label{fig:stakeholder-map}
\end{figure}

\begin{itemize}
    \item \textbf{内部利益相关者 (Internal Stakeholders):}
        \begin{itemize}
            \item \textit{学校领导层(校董会、校领导班子):} 主要利益是学校的声誉、财务稳定、达成KPI指标以及遵守上级规定。
            \item \textit{教师与研究人员:} 利益包括学术自由、良好的工作条件、专业发展机会以及减少行政负担。
            \item \textit{行政人员(包括质量保障处干部):} 利益是流程清晰、有足够资源完成工作以及得到其他部门的合作。
            \item \textit{学生:} 核心利益是培养方案的质量、有效的教学方法、良好的学习环境、毕业后的就业机会以及合理的学费。
        \end{itemize}
    \item \textbf{外部利益相关者 (External Stakeholders):}
        \begin{itemize}
            \item \textit{教育培训部:} 利益是确保高等教育体系稳定运行、质量均衡,满足国家经济社会发展目标,并向政府和社会履行问责。
            \item \textit{教育质量认证中心:} 利益是维持其组织的声誉、专业性和独立性;完成被赋予的认证任务。
            \item \textit{企业与雇主:} 利益是能够招聘到具备与工作要求直接匹配的技能和知识的人力资源,减少再培训成本。
            \item \textit{家长与家庭:} 利益是为子女的投资(时间、金钱)带来应有的回报(孩子有好工作、成才)。
            \item \textit{校友:} 利益是学位证书的价值和母校声誉的提升,从而在职业网络中产生自豪感和机会。
            \item \textit{社会:} 利益是高等教育有助于创造一个公平、进步的社会,提供高质量的人力资源并维护文化价值观。
        \end{itemize}
\end{itemize}
该地图显示,质量保障体系必须服务于一个极其多样化的利益网络。一个质量改进的决策,例如增加学生的实践要求,可能会受到企业的欢迎,但却可能遭到教师(因工作量增加)或财务部门(因成本增加)的反对。识别和分析这些潜在的利益冲突,是构建一个有效的质量保障体系的第一步。



% done chuong 2 goi 2

Chắc chắn rồi, đây là nội dung đã được dịch sang tiếng Trung và định dạng lại theo yêu cầu của bạn:

% ======================================================================
% TRANG 11-13: THUYẾT CÁC BÊN LIÊN QUAN (PHẦN 2)
% ======================================================================

\subsubsection{利益相关者的显著性 (Stakeholder Salience)}

识别利益相关者仅仅是第一步。治理中的一个巨大挑战是如何处理来自这些群体的多样且常常相互矛盾的要求。并非所有利益相关者都具有同等的影响力。Mitchell, Agle和Wood(1997)提出了一个极具影响力的模型,用以确定利益相关者的“显著性”(salience),该模型基于三个核心属性\footcite{Mitchell1997}:

\begin{enumerate}
    \item \textbf{权力 (Power):} 利益相关者将其意志强加于组织之上的能力,迫使组织去做那些若无此压力则不会做的事情。权力可以来自对关键资源(如预算)的控制、颁布法规的能力或影响媒体的能力。
    \item \textbf{合法性 (Legitimacy):} 社会承认某一利益相关者的要求或行动在特定社会背景下是正当、适宜和正确的。合法性来自法律规定、合同或社会道德规范。
    \item \textbf{紧迫性 (Urgency):} 利益相关者的要求需要立即关注的程度。紧迫性取决于两个因素:时间敏感性(若不及时处理,要求将失去价值)和该要求对利益相关者的重要程度。
\end{enumerate}

根据拥有一、二或全部三个属性,利益相关者可被划分为不同优先级的群体,从“潜在”群体(latent stakeholders,仅有1个属性)到需要领导层最多关注的“决定性”群体(definitive stakeholders,拥有全部3个属性)。

将此模型应用于越南高等教育的质量保障体系,我们可以解释一些重要现象:
\begin{itemize}
    \item \textbf{教育培训部}是一个“决定性”利益相关者。该部拥有\textit{权力}(通过分配预算、颁发许可),拥有\textit{合法性}(依据《高等教育法》),并且其要求通常具有\textit{紧迫性}(与认证、报告周期挂钩)。因此,该部的声音分量最重,各大学通常优先满足其要求。
    \item \textbf{企业/雇主}是一个拥有\textit{合法性}(他们有权要求高质量的人力资源)和\textit{紧迫性}(人力需求不断变化)的利益相关者,但却缺乏直接\textit{权力}来迫使大学立即改变培养方案。这解释了为什么尽管企业不断抱怨学生质量,但大学培养方案的变革仍然缓慢。
    \item \textbf{学生和家长}拥有很高的\textit{合法性}和\textit{紧迫性},但他们的权力是分散的。只有当他们的要求通过大规模调查或媒体汇集起来时,他们的权力才会变得更加显著。
\end{itemize}
因此,显著性模型帮助我们理解,质量保障过程并非一个所有声音都平等的绝对民主过程,而是一个“政治舞台”,各大学必须不断权衡并优先考虑来自最具影响力的利益相关者的要求。

\subsubsection{利益相关者理论的应用与批判}

利益相关者理论提供了一个有效的分析工具,用以解读质量保障体系中的利益冲突。它有助于回答“质量为谁服务?”这一问题。通过识别利益相关者并分析其优先级,我们可以理解为什么质量的某些方面(例如:国际论文发表数量)会比其他方面(例如:学生的实践技能)更受重视。

然而,该理论也有其局限性。应用它时最大的挑战在于,当利益相互冲突时,如何找到一个最优方案来\textbf{平衡}这些利益。例如,为满足企业要求而加强实践教学会增加培养成本,可能导致学费上涨,从而引起学生和家长的反对。利益相关者理论很好地描述了这场争论,但并未提供解决它的明确公式。此外,该理论侧重于有形的关系和利益,但较少关注对组织行为有强大影响的无形文化规则和规范——而这正是新制度主义理论的长处。要理解具体的问责机制,我们需要借助第三种理论。

\subsection{委托代理理论 (Principal-Agent Theory)}
\label{subsec:uy_nhiem_nen_tang}

\subsubsection{起源与核心概念}
委托代理理论起源于经济学,特别是金融经济学和公司理论,其基础性工作由Stephen Ross等学者,特别是Michael Jensen和William Meckling在1970年代完成\footcite{JensenMeckling1976}。该理论旨在分析在一方(委托人 - Principal)雇佣或委托另一方(代理人 - Agent)代其执行某项工作时所产生的问题。

这种关系的核心问题(代理问题)源于两个基本条件:
\begin{enumerate}
    \item \textbf{目标冲突 (Conflict of Interest):} 委托人和代理人的利益并非总是一致。例如,股东(委托人)希望最大化利润,而管理者(代理人)可能希望最大化权力或其他个人利益。
    \item \textbf{信息不对称 (Information Asymmetry):} 代理人通常比委托人拥有更多关于工作及自身努力的信息。委托人难以完美地监督代理人的每一个行动。
\end{enumerate}
这两个条件的结合产生了两种主要风险\footcite{Eisenhardt1989}:
\begin{itemize}
    \item \textbf{道德风险 (Moral Hazard):} 合同签订后,由于委托人无法观察其努力程度,代理人可能不会尽全力工作。例如,一所大学(代理人)在从政府(委托人)获得预算后,可能不会尽最大努力投入于改进教学质量。
    \item \textbf{逆向选择 (Adverse Selection):} 签订合同前,代理人可能隐瞒信息或歪曲自身能力以求被选中。例如,一所大学可能会“美化”其自评报告,以获得认证机构的承认。
\end{itemize}
为解决这些问题,该理论提出了设计有效合同、创建激励机制(incentives)以协调利益等解决方案,最重要的是,建立\textbf{监督与报告系统(monitoring and reporting systems)}以减少信息不对称。


% done chuong 2 goi 3

Chắc chắn rồi, đây là nội dung đã được dịch sang tiếng Trung và định dạng lại theo yêu cầu của bạn:

% ======================================================================
% TRANG 16-18: THUYẾT ỦY NHIỆM (PHẦN 2) VÀ PHÊ PHÁN
% ======================================================================

\subsubsection{作为监督和合同机制的质量保障体系}

从委托代理理论的视角来看,整个外部质量保障体系可以被视为一个复杂的监督和合同机制,旨在解决“代理问题”。

\paragraph{多层次的委托代理关系:}
在高等教育中,关系不仅仅是单个委托人与代理人之间的关系。它是一系列相互嵌套的委托代理关系链,使得管理和监督变得复杂\footcite{Borgos2013}。我们可以将越南的这一关系链想象如下:
\begin{itemize}
    \item \textbf{第一层(政府 $\rightarrow$ 教育培训部):} 政府(委托人)将管理和发展国家高等教育体系的任务交给教育培训部(代理人)。
    \item \textbf{第二层(教育培训部 $\rightarrow$ 大学):} 教育培训部(委托人)向各大学(代理人)颁发办学许可并分配预算,期望各大学能够培养出高质量的人力资源并执行服务社会的研究任务。
    \item \textbf{第三层(校领导班子 $\rightarrow$ 院/系):} 校领导班子(委托人)将招生指标和预算分配给各院系(代理人),要求各院系保证其单位的培养和研究质量。
    \item \textbf{第四层(院/系 $\rightarrow$ 教师):} 院长/系主任(委托人)将教学任务分配给每位教师(代理人),并期望他们能以最佳方式完成授课。
\end{itemize}
这一委托代理链的存在解释了为什么来自最高层(政府)的要求在传达到下层时常常会被“干扰”或变形。

\paragraph{质量保障实践中的监督机制:}
为了在上述关系链中最大限度地减少道德风险和逆向选择,质量保障体系发展出了多种具体的监督机制:
\begin{itemize}
    \item \textbf{报告系统 (Reporting Systems):} 自评报告、年度报告、数据统计等要求,正是上级委托人收集下级代理人活动信息的工具。
    \item \textbf{外部认证 (External Accreditation):} 这是一种专门且正式的监督形式。认证机构(如VNU-CEA或AUN-QA)扮演着第三方“审计员”的角色,受委托人(教育培训部)信任,以核实信息并评估各大学(代理人)的活动\footcite{Borgos2013}。这一过程显著减少了信息不对称。
    \item \textbf{基于成果的合同 (Outcome-based Contracts):} 尽管在越南尚不普遍,但世界范围内的趋势是将预算分配与具体的产出结果(例如:学生就业率、国际发表数量)挂钩。这是一种旨在协调委托人与代理人利益的合同形式。
    \item \textbf{声誉建设 (Reputation Building):} 排名系统和公开认证结果也是一种监督机制。一所学校(代理人)的声誉成为一项重要资产,为了保护这项资产,他们有动力按照各委托人(学生、社会)的期望行事。
\end{itemize}

\subsubsection{委托代理理论的应用与批判}

委托代理理论提供了一套锐利的分析工具,用以剖析质量保障体系中的问责结构和监督机制。它逻辑地解释了为何需要报告、检查、定期认证等流程。特别是在分析合同性关系,如国家与被赋予自主权的公立大学之间的关系时,它尤其有用。

然而,机械地将委托代理理论应用于高等教育领域也有其显著的局限性\footcite{RIHE2022}:
\begin{itemize}
    \item \textbf{过分简化目标:} 该理论假设委托人的目标可以被明确定义。但在高等教育中,“质量”是一个多维度且难以衡量的概念。政府的目标可能是发展经济,而社会的目标是公平,学术界的目标则是创作自由。谁才是真正的“委托人”?
    \item \textbf{忽略文化与规范因素:} 该理论认为主体之间的关系主要基于经济利益和理性计算。它无法解释信任、共同价值观以及职业规范(例如:教师道德)在调节教师和学校行为中的作用。
    \item \textbf{难以应用完美的合同:} 在现实中,很难设计一个能预见所有情况并准确衡量代理人所有努力的合同。教育的“产品”(人)是极其复杂的,不能像工业产品那样容易衡量。
\end{itemize}
正因为这些局限性,分析质量保障体系不能仅仅依赖委托代理理论。必须将其与新制度主义理论(以理解无形规则)和利益相关者理论(以理解多样化利益)相结合,才能获得更全面、更准确的图景。

% ======================================================================
% TRANG 19-20: CÁC MÔ HÌNH ĐBCL HIỆN ĐẠI
% ======================================================================
\section{世界现代质量保障模型}
\label{sec:mo_hinh_hien_dai_the_gioi}

治理理论的发展和实践中的挑战,推动了日益精密的质量保障模型的诞生。这些模型不再固守单一理论,而是对传统方法的局限性进行综合和超越的努力。其中,混合模型和适应性框架这两个突出的趋势,对越南具有重要的参考意义\footcite{HybridModel2023}\footcite{AdaptiveQA2022}。

\subsection{混合模型 (Hybrid Model): 协调问责与改进}

\subsubsection{诞生背景与理念}
传统的质量保障体系通常陷入两个极端之一。一是,体系过分注重对外部的\textbf{问责制(accountability)}(自上而下),导致学校只专注于形式上、应付性地遵守规定,形成一种“合规文化”(culture of compliance)而非实质性改进。这是一种接近委托代理理论思维的模式。二是,体系过分注重\textbf{内部改进(improvement)}(自下而上),将全部权力交给单位自行评估,导致缺乏共同标准且难以确保对社会的问责。

混合模型应运而生,旨在解决这种紧张关系\footcite{EUA_Integration}。其理念是承认并整合这两个目标。它认为一个可持续的质量保障体系必须既能满足外部的控制和透明度要求,又能激发内部的改进动力和发展质量文化。混合模型不将问责与改进视为对立的目标,而是视其为一枚硬币的两面,相辅相成。例如,外部透明的问责要求可以为推动内部改进努力提供必要的数据和压力。反之,强大的内部改进文化将使学校更容易满足并超越外部的问责要求。

\subsubsection{组成部分与特点}
一个典型的混合模型通常具有以下特点:
\begin{itemize}
    \item \textbf{双重标准框架:} 体系可能包括一套用于问责目的的强制性\textbf{最低标准(minimum standards)},以及一套用于改进目的的鼓励性\textbf{提升标准(enhancement standards)}。
    \item \textbf{嵌套流程:} 外部认证活动的设计不仅旨在检查合规性,还旨在提供建设性建议,支持学校的改进过程。反之,内部自评和改进活动的结果被用作外部认证轮次中的重要证据。
    \item \textbf{质量保障机构的灵活角色:} 外部质量保障机构不仅扮演“裁判员”的角色,还扮演“伙伴”、“顾问”的角色,在改进过程中为学校提供支持。
\end{itemize}
欧洲质量保障体系的经验,尤其是在博洛尼亚进程之后,显示出向混合模型转变的明显趋势,其中认证流程越来越被设计为服务于质量提升的目标(enhancement-led accreditation)\footcite{EUA_Integration}。越南的实践,既要满足教育培训部的国家标准,又要参与如AUN-QA等区域认证项目,也正显示出一种混合模型正在逐渐形成的迹象,尽管可能尚非完全自觉\footcite{VNU-CEA2023}\footcite{HangNguyen2017}。

% done chuong 2 goi 4















	
%======================================================================
% GÓI 1 (VIẾT LẠI): MỞ ĐẦU VÀ BẰNG CHỨNG VĨ MÔ VỀ "NGHỊCH LÝ PHÁT TRIỂN"
% (Phần 1: Tăng trưởng và Thành tựu)
%======================================================================

\chapter{越南外部质量保障体系现状分析}
\label{chap:thuc_trang}

\section*{引言}
\addcontentsline{toc}{section}{引言}

本章将对越南高等教育(GDĐH)外部质量保障(ĐBCL)体系的现状进行深入和多维度的分析,重点关注2015年至2024年这一强劲转型阶段。本章将运用第二章所论证的V-AQA理论模型视角,重点“解剖”一个深刻的\textbf{发展悖论(development paradox)}:即规模和投入指标的爆炸性增长,并未伴随着产出质量及与劳动力市场契合度的相应提升。

通过综合分析宏观统计数据、世界银行等权威国际组织的报告、学术研究,特别是通过图表进行可视化的数据,本章将提供坚实的科学论据。目标不仅是描述现状,更是要解释阻碍质量提升努力的系统性“瓶颈”和“恶性循环”。从而,本章将为后续章节提出战略性解决方案奠定坚实的实践基础。

\section{越南高等教育中的发展悖论}
\label{sec:nghich_ly_phat_trien_vimo}

2015-2024年阶段标志着越南高等教育一个充满变动的转型历程,揭示了一个深刻的发展悖论。一方面,该体系在教育大众化、扩大培养规模以及提升投入资源质量方面取得了前所未有的成就。另一方面,这些关于“量”的成就,却掩盖了关于“质”的持续挑战,体现在毕业生技能与劳动力市场实际需求之间日益扩大的不匹配。分析此悖论的两个方面,是理解该体系核心挑战的先决条件。

\subsection{悖论的第一面:规模的爆炸性增长与投入资源的改善}
\label{subsec:ve_thu_nhat_nghich_ly}

不可否认,越南高等教育在扩大民众教育机会方面取得了令人瞩目的进展。这一时期见证了强劲的大众化进程,体现在办学机构网络的拓展和学生规模的飞跃式增长。

\subsubsection{拓展教育网络与培养规模}

过去十年越南高等教育体系发展的全貌,通过图\ref{fig:so_truong_quy_mo_sv}得以清晰展现。

\begin{figure}[h!]
    \centering
    \includegraphics[width=\textwidth]{image/dh_viet_nam_2015_2024.png}
    \caption{越南高等教育办学机构数量与学生规模发展情况(2015-2024年)}
    \label{fig:so_truong_quy_mo_sv}
\end{figure}

分析图\ref{fig:so_truong_quy_mo_sv}显示了两种同步但速度不同的增长趋势。
\begin{itemize}
    \item \textbf{关于网络(蓝色线):} 高等教育机构数量呈现稳定而坚实的增长趋势,从2015年的215所增至2024年的243所。这一增长虽然不算突变,但反映了政府在拓展办学机构网络方面的一贯政策,包括成立新大学和升级专科学校,旨在为全国民众提供多样化的学习机会。
    \item \textbf{关于学生规模(红色线):} 与学校数量的平稳增长形成对比的是,学生规模呈现出异常强劲的爆炸性增长。在经历了2015年至2021年的相对稳定期后,学生规模在最后三年(2022-2024)内从约205万猛增至超过\textbf{235万学生}。这一飞跃不仅体现了高等教育日益增长的吸引力,也表明该体系正承受着为前所未有的大量学生提供培养需求的巨大压力。
\end{itemize}

这一增长正逐步使越南更接近到2030年实现\textbf{每万人口260名大学生}的国家战略目标\footcite{sggp_en_3million_2030}。此外,它还体现在重要的国际指标——高等教育毛入学率(GER)上。世界银行和联合国教科文组织的数据显示,越南的毛入学率已从2000年的仅10.3\%飙升至2018年的28.6\%,并于\textbf{2022年达到创纪录的42.22\%}\footcite{worldbank_humancapital_2022}。这是一项值得称道的成就,显示了教育大众化政策的成功。

\subsubsection{改善投入资源质量}
在扩大规模的同时,该体系的投入资源质量也取得了显著改善,显示了在提升内在能力方面的有目的的投资。
\begin{itemize}
    \item \textbf{师资队伍质量:} 拥有研究生学历(硕士或博士)的大学教师比例几乎翻了一番,从2007年的47\%增至\textbf{2020年达到85\%}\footcite{worldbank_p178112}。对提升师资队伍水平的投资是一个基础性因素,有望直接转化为教学和研究质量。
    \item \textbf{科学研究能力:} 提升教师水平的成就已产生具体影响。越南人均在国际权威期刊上可引用的科学文献数量,在十年间(2010-2020)\textbf{增长了三倍}\footcite{worldbank_improvingperformance_2020}。这表明该体系的研究能力取得了飞跃式进展,正逐步与国际科学界接轨。
\end{itemize}

上述数字和图表描绘了一幅关于悖论第一面的乐观画面:一个正在强劲发展、在网络、学生规模乃至投入资源质量等各方面都在扩张的高等教育体系。这些是不可否认的成就,为一个现代化和国际化的高等教育体系奠定了坚实的基础。然而,这幅画面只是一个更为复杂的悖论的一半。当我们将这些关于“量”的成就与关于产出质量和体系效率的指标进行对比时,一个充满挑战和警示的另一面故事开始显现。悖论的第二面将在下一部分深入分析。



% het goi 1


\subsection{悖论的第二面:质量的潜在危机与不匹配}
\label{subsec:ve_thu_hai_nghich_ly}

与规模和投入资源令人印象深刻的增长图景相反的是,关于产出质量和体系可持续性的 alarming signals。如果说悖论的第一面是一曲增长数字的交响乐,那么第二面则是一个残酷的现实,体现在培养规模与满足劳动力市场能力之间的分化。图\ref{fig:nghich_ly_quy_mo_chat_luong}清晰地将这一悖论可视化。

\begin{figure}[h!]
    \centering
    \includegraphics[width=\textwidth]{image/nghich_ly_tang_truong_quy_mo_chat_luong.jpg}
    \caption{越南高等教育规模增长与产出质量之间的悖论(2015-2024年)}
    \label{fig:nghich_ly_quy_mo_chat_luong}
    \vspace{0.2cm}
    \footnotesize{\textit{来源:综合整理自教育培训部数据及相关报告。}}
\end{figure}

\subsubsection{剖析“量”与“质”的分化}

图\ref{fig:nghich_ly_quy_mo_chat_luong}将两个重要指标置于同一坐标系中:学生总规模(左纵轴,蓝色线)和毕业一年后就业率(右纵轴,红色线)。理论上,在一个可持续发展的体系中,这两条线应呈正相关或至少保持稳定。然而,实际数据显示出一种令人担忧的分化(divergence):
\begin{itemize}
    \item \textbf{规模线(蓝色)}呈现出持续增长,特别是从2020年不到200万学生猛增至2024年超过230万学生。这再次证实了该体系面临的扩大规模的压力。
    \item \textbf{质量线(红色)}则呈现出完全相反的趋势。在2019-2020年左右达到峰值(超过90%)后,学生就业率出现了惊人的急剧下降,到2024年降至仅约87%。
\end{itemize}
这种分化正是发展悖论的“核心”:当体系在规模上日益“膨胀”时,其最核心的价值——帮助学习者获得就业并为社会做贡献的能力——却在下降。这提出了一个紧迫的问题:难道对数量的追求已经严重损害了质量?

\subsubsection{来自劳动力市场的证据:失业、技能差距与不平等}

图表中“质量”线的下降趋势并非感性判断,而是由一系列来自劳动力市场和社會學分析的確鑿數據所證實。

\paragraph{失业与学非所用。} 劳动荣军与社会部的报告多次警示本科毕业生失业率高企,\textbf{2017年高达23.7万人},2018年仍有\textbf{22.55万人},约占全国失业总人数的20\%\footcite{vietnamnews_unemployed_2017}。这一数字,加上估计约有\textbf{60\%的毕业生从事与专业不符的工作},是因培养与需求脱节而造成社会和个人资源浪费的明显证据\footcite{britishcouncil_grad_employability_2021}。

\paragraph{技能差距。} 上述状况的深层原因在于严重的“技能差距”。英国文化协会的一项全面调查指出了一个令人担忧的现实:\textbf{73\%}的越南企业在招聘具备领导和管理技能的人才时遇到困难;\textbf{68\%}在寻找具备足够专业技能的员工时遇到困难;\textbf{54\%}表示缺乏具备必要社交情感技能的人才\footcite{britishcouncil_skills_gap_2021}。世界银行的报告也强调,近\textbf{80\%}的制造业公司在招聘熟练工人时遇到困难\footcite{worldbank_p178112}。这表明培养方案并未能为学生充分装备经济真正需要的技能。

\paragraph{机会不均与财务负担。} 除了质量问题,发展悖论还体现在不平等方面。尽管越南高等教育的个人回报率位居世界最高之列(年均超过15\%)\footcite{worldbank_improvingperformance_2020},但这一利益并未得到公平分配。数据显示,高达\textbf{80\%来自收入最高20\%家庭的青年}已经或正在接受大学教育,而这一比例在两个最低收入组中仅占学生总数的10\%\footcite{worldbank_p178112}。由于成本负担日益转向家庭(占公立学校总收入的\textbf{77\%})以及学生贷款项目的缩减,这种不平等正面临着加剧的风险\footcite{worldbank_p178112}。

总之,悖论的第二面展示了一幅充满挑战的图景:一个尽管在规模和资源上投入巨大,但其产出却未能满足市场要求,同时潜藏着社会不平等风险的体系。清晰地认识这一悖论的两个方面,是能够准确“诊断”系统性“病症”的先决步骤,这一任务将在本章的后续部分进行。


% het goi 2



\section{塑造质量的主体与制度框架}
\label{sec:khung_the_che}

从已证明的“发展悖论”图景中,一个核心问题被提出:体系中的哪些力量和主体,以何种方式行动,从而造成了这一现状?要回答这个问题,首先需要识别主要的利益相关者,并分析主导该体系“游戏规则”的制度框架。

\subsection{高等教育质量保障生态系统中的利益相关者图}

越南高等教育质量保障体系是一个复杂的生态系统,有众多利益相关者参与其中,每一方都扮演着不同的角色、拥有不同的利益和影响力。图\ref{fig:so_do_he_thong_dbcl}提供了该生态系统中主要主体的概览。

\begin{figure}[h!]
    \centering
    \includegraphics[width=\textwidth]{image/he_thong_dbcl_gddh.jpg}
    \caption{越南高等教育质量保障体系中的主要利益相关者图}
    \label{fig:so_do_he_thong_dbcl}
\end{figure}

上图可分为两大主要群体:
\begin{enumerate}
    \item \textbf{管理与执行群体(左侧与中心):} 包括国家机构和被直接授权的组织,扮演着制定政策和执行认证活动的角色。该群体包括\textbf{教育培训部}和各\textbf{认证中心}。
    \item \textbf{监督与受益群体(右侧):} 包括社会中的各个主体,他们是高等教育“产品”的直接使用者,并为质量提供重要的反馈。该群体包括\textbf{企业与雇主}、\textbf{家长与家庭}、\textbf{校友},以及更广泛的整个\textbf{社会}。
\end{enumerate}
分析这些主体,特别是制定“游戏规则”的群体之间的角色和关系,将阐明正在塑造越南质量保障现状的动因和压力。

\subsection{教育培训部:游戏规则的制定者与合规性压力}
\label{subsec:vai_tro_moet}

在越南的质量保障生态系统中,教育培训部扮演着核心权力主体的角色,负责构建和协调整个体系。从委托代理理论的视角来看,教育培训部是最高的“委托人”,将培养和保障质量的任务委托给各大学(代理人)\footcite{Kivisto2008}。同时,根据新制度主义理论,该部是产生最强大“强制性压力”的源头,迫使各大学遵守一个共同的法规框架,以获得合法性并维持运作\footcite{MeyerPowell2020}。

这种引领作用通过颁布一系列法规文件得以清晰体现,其核心精神由关于“根本性、全面性教育与培训革新”的第29号决议所指导\footcite{nghi_quyet_29_2013}。该决议明确提出了“增强教育培训机构的自主权和社会责任”以及建立一个“独立的认证体系”的要求。正是这些战略性导向,构建了法律走廊,推动各大学逐步与区域及国际标准接轨。

然而,这种集中的垂直管理机制,虽然对于确保统一性是必要的,但也正是导致许多教育机构形成一种\textbf{“合规文化”}而非\textbf{“改进文化”}的根本原因之一\footcite{pham2021governance}。各大学,特别是严重依赖国家预算的公立大学,倾向于优先开展旨在满足该部的报告要求和认证标准的活动,有时会轻视来自内部的实质性改进。因此,教育培训部既扮演着“游戏规则”制定者的角色,又是最高监督者,创造了一个合规性通常被置于创新之上的制度环境。

\subsection{质量认证中心:政策执行与独立性问题}
\label{subsec:vai_tro_trungtamkd}

如果说教育培训部是“委托人”,那么各教育质量认证中心则扮演着重要的“代理人”角色,其任务是具体化和执行认证政策。这些中心体系的发展,是越南质量保障活动制度化努力的证明。截至2024年初,全国共有7家获准运营的教育质量认证中心,包括公立和私立单位\footcite{tuoitre_kdcl_stats_2024}。

该体系已积极运作,在执行国家质量政策方面扮演着重要的“延伸手臂”角色。截至2023年底,这些中心已对全国\textbf{1855个培养项目}和\textbf{187所教育机构}进行了认证和承认\footcite{tuoitre_kdcl_stats_2024}。2021年两家私立中心的成立,也标志着朝着社会化、评估单位多样化方向迈出了新的一步。

然而,学术界和管理层提出的一个重要问题是这些中心的\textbf{实质性独立性}。尽管是以独立法人身份成立,但7家中心中有5家仍是大型大学或协会的下属单位,并且所有中心都在教育培训部的严密监督下运作。这种“既是主管单位,又是被认证对象”的关系,可能会引起人们对评估结果绝对客观性的怀疑\footcite{giaoducnet_kdcl_list_2023}。更重要的是,它有可能会进一步强化大学的合规压力。许多学校并未将认证中心视为共同改进的伙伴,而是仍然抱着应付上级“检查”的心态。这种复杂的关系将在后续章节中,在审视质量文化和体系协调等挑战时进行更深入的分析。


% het goi 3




\section{基于V-AQA模型的现状分析}
\label{sec:phan_tich_vqa}

在确定了主要主体的角色和体系的“游戏规则”之后,本章将运用V-AQA模型的五个要素,系统地“解剖”越南高等教育质量保障体系所面临的核心挑战。分析将从问题的源头——领导能力和组织文化——这两个具有紧密因果关系并塑造校内所有质量活动的要素开始。

\subsection{要素一:领导与治理的挑战——自主与合规之间}
\label{subsec:thach_thuc_lanhdao}

\textbf{领导与治理}要素是整个质量保障体系的发起、导向和提供能量的源泉。它体现在高层领导团队(校董会、校领导班子)在战略上引领学校实现质量目标的愿景、承诺和能力。然而,在越南,这一领导角色正处于两难境地,被困在推动自主的政策和仍然带有浓厚合规性的管理实践之间。

\subsubsection{自主政策与管理实践的矛盾}
尽管大学自主的主张已在修订后的《高等教育法》中制度化,但在实际推行中仍存在诸多障碍。顶尖专家已指出规定与实践之间的严重不匹配。根据顶尖高等教育政策专家之一\textbf{范氏丽博士}的分析,现行大学自主的法律框架如同“一件遮不住身体的紧身衣”\footcite{lypham_aosat_2024}。她论证道,“主管部门”的概念仍然是一个巨大的障碍,这种理解与国际惯例相去甚远,削弱了自主的本质。公立大学倾向于只向管理层“报告”,而不是真正地对社会就培养质量负责,这使得问责制变得形式化。

同样,前高等教育司副司长\textbf{黎曰劝博士}也提出了一个坦率的看法:“不废除主管部门机制,就别急着转向自主”\footcite{khuyen_bochuquan_2022}。他分析称,只要主管部门还存在并有权干预人事和财务决策,校董会的实质性自主权就会受到限制。届时,学校领导者的角色很难摆脱行政命令执行者的地位,而不是成为一个真正的战略管理者。

\subsubsection{后果:领导层优先级的转移}
这种情况的直接后果是,领导团队在质量方面的战略管理能力未能得到充分发挥。许多管理者不得不将大部分时间和精力用于处理行政程序和满足上级要求,而不是能够专注于制定长期决策,如建设质量文化、推动创新或建立战略联盟。他们的重心从“如何实质性、可持续地提升质量?”这一问题,转移到更短期的“如何完成报告并通过认证评估?”这一问题上。这种优先级的转变是一个无形但极其强大的障碍,它抑制了来自基层的创新努力,并直接催生了下文将分析的应付式质量文化。

\subsection{要素二:质量文化的挑战——从被动合规到主动改进}
\label{subsec:thach_thuc_vanhoa}

如果说领导与治理要素是质量保障体系的“大脑”,那么\textbf{质量文化}就是其“灵魂”,是决定质量流程是得到实质性执行还是仅为形式应付的关键因素。哈维和斯坦塞克(2008)将质量文化定义为一个共享的价值观和信念集合,组织中的每个成员都自觉地致力于持续改进\footcite{HarveyStensaker2008}。然而,实践证据表明,这是越南高等教育中最重大且固有的挑战之一,是上述带有浓厚合规性管理模式的直接后果。

\subsubsection{“反应型质量文化”的定性}
一项关于越南大学质量文化的深入研究,将这一特征定性为一种\textbf{“反应型质量文化”}。在此层次上,组织仅在出现问题或有外部压力(例如:认证评估)时才关注质量,而不是主动寻找改进机会\footcite{vjol_reactiveculture_2021}。质量保障活动通常只在认证周期临近时才被大力推动,并在学校获得证书后趋于“沉寂”。这是质量保障研究中常称的“季节性活动”或“认证风暴”现象。

从新制度主义理论的角度看,这是“脱钩”现象的典型表现,即组织形式上采纳所要求的结构和流程以获得合法性,而内部核心活动(教学、学习、研究)却无实质性改变\footcite{MeyerRowan1977}。这种文化的直接后果是学术团队的被动参与。本应是改进过程主体的教师和员工,却常常将质量保障活动视为一种行政负担,一种教学专业之外的“额外工作”\footcite{iosr_passiveparticipation_2021}。

\subsubsection{通过社会信任的波动表现出来}
内在质量文化的脆弱性必然会通过社会的信任度反映出来,而考生和家庭的入学决定是一个敏感指标。图\ref{fig:ty_le_nhap_hoc}显示了2015-2024年间考生确认入学比例的显著波动。

\begin{figure}[h!]
    \centering
    \includegraphics[width=\textwidth]{image/ty_le_nhap_hoc_2015-2024.png}
    \caption{大学新生确认入学比例(2015-2024年)}
    \label{fig:ty_le_nhap_hoc}
    \vspace{0.2cm}
    \footnotesize{\textit{来源:综合整理自教育培训部历年招生数据。}}
\end{figure}

图表并未呈现稳定的增长趋势,而是显示出剧烈波动。特别是,从2016年的86.8\%急剧下降到\textbf{2017年的仅64\%},是一个令人警醒的信号。这种突降不能仅用招生规定的变化来解释,更可以被解读为当时社会对大学文凭价值和质量的一次“信任危机”,此前媒体已长期广泛报道本科生失业问题。

近年来该比例的回升并维持在较高水平(80%以上),显示了整个体系的改进努力,尤其是在修订后的《高等教育法》生效之后。然而,正是这种波动表明,利益相关者的信任仍然脆弱,越南高等教育的感知质量尚未真正稳固。一个实质性的、来自内部的质量文化,需要创造一个稳定可靠的承诺,而不是像这样充满波动且依赖外部因素的结果。

\subsubsection{打破惰性的努力:促进机制的案例研究}
尽管合规文化仍然普遍,但一些教育机构已开始率先采用具体的治理机制来打破这种惰性。一个典型案例是\textbf{胡志明市技术大学}。自2022年起,该校推行了一项具有杠杆作用的政策:学生必须完成关于教师教学活动的在线调查,才能查看该课程的期末考试成绩\footcite{hutech_khao_sat_2022}。
这项政策创造了一个积极而强大的反馈循环。学生完成调查问卷的比例达到非常高的平均水平,提供了一个巨大而全面的反馈数据源。面对这些数据,教师被迫倾听并参与到改进过程中。结果显示,许多教师已主动调整方法、更新课件幻灯片并补充实践案例。这个案例生动地说明,质量文化并非一个抽象概念,而是可以通过足够强大的具体治理机制来影响和塑造,从而从被动参与转变为主动调整。


% het goi 4























	

\chapter{为越南高等教育提出混合与适应性质量保障模型(V-AQA模型)}
\label{chap:de_xuat_vqa}

\section[引言]{引言:背景与新模型的紧迫性}
\addcontentsline{toc}{section}{引言}

基于第三章对越南高等教育质量保障体系现状及其系统性挑战的深入分析,得出了一个重要结论:现存问题之间存在因果联系,形成了复杂的“恶性循环”。世界银行的诊断报告已明确指出治理薄弱、培养与市场脱节以及质量保障体系尚未真正有效等问题\footcite{worldbank_improvingperformance}。这些问题并非孤立存在,而是相互交织,包括:大学自主权仍然有限\footcite{world-bank_improvingperformance},利益相关者参与度不高\footcite{pmc_article_9127449},质量文化带有浓厚的应付色彩\footcite{vjol_reactiveculture},以及质量保障人力资源既缺又弱\footcite{pmc_article_9127449}。

这表明,采用零散、片面的解决方案或机械地照搬国外的传统模式,将无法从根本上解决问题。越南需要一种新的方法,一个全面的行动框架,既能适应一个转型期经济体的特殊背景,又能与国际质量标准接轨。

为响应这一要求,本章将提出并详细论证一个新模型——\textbf{混合与适应性质量保障模型(简称V-AQA)}。该模型不仅是一系列孤立解决方案的集合,更是一个系统的思维和行动框架,旨在直接打破已指出的“恶性循环”,并从内部推动一种可持续的质量文化。

\section{越南现行质量保障框架的差距分析}
\label{sec:phan_tich_khoang_trong}

为申明V-AQA模型的紧迫性和适用性,首先需要分析在越南普遍应用的现有质量保障框架未能彻底解决的差距。

\subsection{东盟大学网络质量保障标准}

东盟大学网络及其质量保障标准在推动质量文化和区域一体化方面做出了巨大贡献。

\paragraph{优势} 东盟大学网络质量保障提供了一套全面的标准(4.0版共8项标准),重点关注成果导向教育和以学习者为中心\footcite{AUN-QAGuide}。获得东盟大学网络质量保障标准认证有助于越南的培养项目提升声誉,并为区域内的学生交换和学分互认创造便利条件\footcite{hoasen_benefits_aunqa}。

\paragraph{差距与局限}
\begin{itemize}
    \item \textbf{主要集中于课程层面:} 尽管有机构层面的标准,但在越南应用东盟大学网络质量保障标准主要发生在课程层面。这可能导致质量不均衡的状况,即少数几个项目达到国际标准,而整个机构,特别是在治理和资源分配方面,仍然沿用旧模式\footcite{aun_institutional_v2}。
    \item \textbf{流程复杂且成本高昂:} 寻求东盟大学网络质量保障认证需要大量的财政和人力资源,这为许多学校,特别是民办或地方院校设置了障碍\footcite{stdjssh_637}。
    \item \textbf{“形式化合规”的风险:} 获得认证的压力可能导致学校专注于完善档案和证据,以机械地满足标准,而不是从内部推动实质性改进\footcite{pmc_article_9127449}。
\end{itemize}

\subsection{教育培训部的规定}
教育培训部的法规文件体系,特别是关于质量认证的通知(如基于东盟大学网络质量保障标准制定的第12/2017号通知),为质量保障活动创造了强制性的法律走廊。

\paragraph{优势} 教育培训部的规定在整个体系内设立了最低质量要求,迫使教育机构进行自评和外部认证,有助于提升对质量保障的普遍认识。

\paragraph{差距与局限}
\begin{itemize}
    \item \textbf{体系缺乏独立性:} 各教育质量认证中心仍受教育培训部的直接管理,这引发了关于国家管理机构与专业认证机构之间客观性和独立性的担忧\footcite{ncdt_journal_219}。
    \item \textbf{大学自主权仍然有限:} 尽管自主政策已经颁布,但执行中仍存在诸多障碍。根据世界银行的报告,只有一小部分公立大学真正参与了自主试点,且自主范围仍然狭窄,特别是在组织和人事方面\footcite{world-bank_improvingperformance}。这降低了学校根据自身需求灵活改进质量的能力。
    \item \textbf{合规文化:} 按照教育培训部规定进行的认证通常被视为一项必须完成的行政义务,导致了“应付式合规”文化,而非内在的改进需求\footcite{vjol_reactiveculture}。
\end{itemize}

\subsection{构思-设计-实现-运作能力框架}
构思-设计-实现-运作是一个先进的框架,被越南许多工科院校采用,以改革培养方案,使其满足企业要求。

\paragraph{优势} 构思-设计-实现-运作提供了一个集成的学习框架,将理论与实践相结合,帮助学生全面发展个人、沟通和专业技能。应用构思-设计-实现-运作极大地推动了工科院校教学方法的创新\footcite{vietnamplus_cdio_reform}。

\paragraph{差距与局限}
\begin{itemize}
    \item \textbf{范围狭窄且难以推广:} 构思-设计-实现-运作主要为工程和技术学科设计。将该模型应用于社会科学、经济学或师范等学科是一个巨大挑战。
    \item \textbf{要求深刻的文化变革:} 成功实施构思-设计-实现-运作要求在思维和组织文化上发生重大变革,从领导层到每一位教师。在越南的研究已指出这种变革的困难,包括初期领导层缺乏正式承诺以及核心教师团队与其余人员之间互动有限\footcite{nguyen_cdio_2016}。
\end{itemize}

\section{V-AQA模型的定位:比较优势与优越性}
\label{sec:dinh_vi_vqa}

基于对上述差距的分析,提出V-AQA模型并非旨在完全替代,而是为了整合并克服现有框架在越南特殊背景下未能彻底解决的固有弱点。

与这些模型相比,V-AQA拥有突出的优势:
\begin{itemize}
    \item \textbf{混合性:} 主动平衡来自外部的\textbf{问责制}压力(教育培训部的要求)和来自内部的\textbf{持续改进}动力(东盟大学网络质量保障和构思-设计-实现-运作的精神),而不是只关注一个方面。
    \item \textbf{适应性:} 按照灵活的短期改进周期运作,帮助学校快速响应市场变化,而不是遵循僵化的长期计划。
    \item \textbf{全面性与内生性:} 作用于学校的全部五个核心方面(领导与治理、文化、利益相关者、流程、合作),并推动来自内部的变革(内生性),而不仅仅是遵守外部要求。
    \item \textbf{技术整合性:} 以管理信息系统为“神经系统”,使基于证据的治理成为可能,这是其他框架未直接提及的因素。
\end{itemize}

为更清晰地说明这种差异,下表将V-AQA与越南普遍应用的框架进行对比:

\begin{table}[h!]
\centering
\caption{V-AQA模型与普遍应用框架的对比表}
\label{tab:doi_sanh_vqa}
\begin{tabular}{|p{3cm}|p{3.5cm}|p{3.5cm}|p{3.5cm}|}
\hline
\textbf{比较标准} & \textbf{东盟大学网络质量保障(课程层面)} & \textbf{教育培训部规定(第12号通知)} & \textbf{V-AQA模型(建议)} \\
\hline
\textbf{主要目标} & 根据东盟共同标准评估、比对课程质量。 & 为各校规定最低标准和强制性认证流程。 & \textbf{从内部创建一个全面、自我改进且可持续的质量体系。} \\
\hline
\textbf{应用范围} & 主要在培养课程层面。 & 课程层面和教育机构层面。 & \textbf{全面覆盖学校层面,从领导、文化到每一个流程。} \\
\hline
\textbf{自主程度} & 学校自愿参加,但必须严格遵守东盟大学网络流程。 & 低,认证机构仍依赖教育培训部,合规性强\footcite{ncdt_journal_219}。 & \textbf{高且有导向:推动自主与通过关键绩效指标明确问责相结合。} \\
\hline
\textbf{利益相关者参与} & 有提及,但实际上学生和企业参与仍然有限\footcite{pmc_article_9127449}。 & 咨询性质,缺乏实质性约束机制。 & \textbf{将利益相关者的角色机制化(行业咨询委员会、学术委员会中的学生代表)。} \\
\hline
\textbf{灵活性} & 标准化流程,灵活性低。 & 规定具有普遍适用性,较为僵化。 & \textbf{高,是核心原则(适应性),允许根据短期周期进行调整。} \\
\hline
\end{tabular}
\end{table}

\textit{(注:构思-设计-实现-运作能力框架未直接纳入比较表,因为其本质是针对工科领域的专门培养方案开发框架,与东盟大学网络质量保障和教育培训部规定的综合性质量保障体系性质不同)。}

上表显示,V-AQA不仅是一套标准,更是一个\textbf{治理框架}。V-AQA不只关注“需要达成什么”,更关注“如何”构建一个能够自我改进的组织。这种方法直接解决了自主、质量文化和利益相关者参与等基础性问题,而仅仅应用单一标准是无法触及这些问题的。

为更深入地理解塑造这些比较优势的思想基础,下一部分将深入论证V-AQA模型的混合理念与适应性原则。


% het goi 1


\section{V-AQA模型的哲学基础与原则}
\label{sec:triet_ly_nguyen_tac_vqa}

V-AQA模型建立在两大哲学基础上,这两大基础是从国际研究和对越南特殊背景的分析中总结出来的:混合哲学与适应性原则。这正是该模型的“灵魂”,塑造了其所提出的方法和解决方案。

\subsection{混合哲学:协调问责制与质量改进}

越南高等教育体系目前正承受着双重压力。一方面,由于国家的主导作用和对公共预算的依赖,各大学必须强有力地满足\textbf{问责制}的要求。另一方面,在日益激烈的竞争和劳动力市场要求的背景下,各大学又迫切需要进行\textbf{改进}以提升质量和品牌。

这两个目标常常产生矛盾,导致要么过分注重形式上的合规以满足外部要求,要么是自发、无纪律的内部改进。V-AQA模型的“混合”哲学正是为了解决这种紧张关系而提出的。正如哈维和威廉姆斯(2010)在对高等教育质量十五年的综述中所分析的,一个有效的体系必须设法整合并协调这两个目标\footcite{harvey_williams_2010}。V-AQA模型通过承认外部透明的问责要求可以为推动内部改进努力提供必要的数据和压力来实现这一点。反之,强大的改进文化将使学校更容易满足并超越问责要求。这种方法有助于将关系从对抗转为共生,其中遵守规定成为前提,而提升质量成为最终目标。

\subsection{适应性原则:在变化背景下的灵活性}

越南的经济社会和政策背景正在以极快的速度变化。第四次工业革命、关于大学自主的新政策以及劳动力市场的波动,都要求质量保障体系必须具有高度的灵活性。一个为期五年的僵化质量改进计划,有可能在尚未执行完毕时就已经过时。

因此,V-AQA模型建立在“适应性”原则之上,其灵感来自于像适应性项目框架这样的敏捷项目管理框架\footcite{Wysocki2009}。该原则鼓励采用短期的、重复的改进周期,而不是一个长期的PDCA循环。各学校可以按学年甚至学期设定质量目标,实施、快速收集反馈数据,并为下一个周期调整计划。这种方法帮助各学校“边做边学”,最大限度地减少长期错误决策的风险,并确保改进努力始终与不断变化的实际背景相符。

\section{V-AQA模型的总体结构与要素}
\label{sec:cau_truc_tong_the}

基于上述两大哲学基础,V-AQA模型被建构为一个由理论基础、五个相互作用的核心要素和最终目标组成的总体结构。

\begin{figure}[h!]
    \centering
    \includegraphics[width=\textwidth]{image/mo_hinh_V-AQA.jpg}
    \caption{混合与适应性质量保障模型 (V-AQA)}
    \label{fig:v-aqa-model-detailed}
\end{figure}

V-AQA模型的五个要素相互作用,形成一个完整的质量体系。为给读者,特别是那些非质量保障领域的专业人士,提供一个全面且易于理解的概览,下表将总结每个要素的目标、具体表现以及建议的关键绩效指标。该表如同一张“地图”,为后续的详细分析部分勾画了结构。

\begin{longtable}{|p{2.5cm}|p{3.5cm}|p{4.5cm}|p{3.5cm}|}
\caption{V-AQA模型5要素总结}
\label{tab:tong_hop_5_thanh_to}\\
\hline
\textbf{要素} & \textbf{主要目标} & \textbf{具体表现(行动)} & \textbf{衡量方式(关键绩效指标示例) \footcite{uq_kpi_dashboard}} \\
\hline
\endfirsthead
\multicolumn{4}{c}%
{{\bfseries \tablename\ \thetable{} -- 续前页}} \\
\hline
\textbf{要素} & \textbf{主要目标} & \textbf{具体表现(行动)} & \textbf{衡量方式(关键绩效指标示例) \footcite{uq_kpi_dashboard}} \\
\hline
\endhead
\hline \multicolumn{4}{r}{{续下页}} \\
\endfoot
\hline
\endlastfoot

% 第1行
\textbf{1. 领导与治理} & 将领导者从“控制者”转变为“环境创造者”。 & 
\begin{itemize}
    \item 制定并执行校级质量战略。
    \item 向院系下放强有力的权力并辅以问责制。
    \item 为中层管理团队提供能力提升培训。
\end{itemize} & 
\begin{itemize}
    \item 质量战略中各项目标的完成率。
    \item 各院系对自主权的满意度。
    \item 完成现代治理培训的中层管理者数量。
\end{itemize} \\
\hline

% 第2行
\textbf{2. 质量文化} & 从“应付式合规”文化转变为“主动改进”文化。 & 
\begin{itemize}
    \item 发起系统的质量宣传运动。
    \item 建立表彰和奖励改进创举的体系。
    \item 成立并授权“质量改进小组”。
\end{itemize} & 
\begin{itemize}
    \item 关于质量文化认知的定期调查得分。
    \item 每年提出并实施的改进创举数量。
    \item 参与改进活动的教职工比例。
\end{itemize} \\
\hline

% 第3行
\textbf{3. 利益相关者的参与} & 将利益相关者(企业、学生、校友)转变为“战略伙伴”。 & 
\begin{itemize}
    \item 将行业咨询委员会的活动机制化。
    \item 让学生代表在科学与培养委员会中拥有投票权。
    \item 建立主动的“校友大使”网络。
\end{itemize} & 
\begin{itemize}
    \item 行业咨询委员会的建议被整合到培养方案中的比例。
    \item 有学生代表参与的学术决策数量。
    \item 企业对毕业生契合度的满意度得分。
\end{itemize} \\
\hline

% 第4行
\textbf{4. 内部流程} & 基于数据实现学术和管理流程的现代化、标准化。 & 
\begin{itemize}
    \item 采用基于成果导向教育理念的培养方案开发流程。
    \item 多样化教学和评估方法。
    \item 构建并整合质量保障管理信息系统。
\end{itemize} & 
\begin{itemize}
    \item 按照成果导向教育流程制定和审查的培养方案比例。
    \item 课程中过程性评估分数的平均比重。
    \item 基于质量保障管理信息系统数据做出的管理决策比例。
\end{itemize} \\
\hline

% 第5行
\textbf{5. 合作与协调} & 打破“孤岛”状态和封闭思维,创造一个开放的质量生态系统。 & 
\begin{itemize}
    \item 为跨学科、跨单位的质量改进项目提供经费。
    \item 建立并参与标杆比对网络。
    \item 战略性地利用国际认证活动进行学习。
\end{itemize} & 
\begin{itemize}
    \item 成功实施的跨院系/部门项目数量。
    \item 组织的标杆比对活动及产生改进报告的数量。
    \item 国际认证建议的完成比例。
\end{itemize} \\
\end{longtable}

本章的后续部分将深入分析和论证上述五个要素中的每一个,包括建议的解决方案、科学依据、潜在风险及缓解策略。



% het goi 2 chuong 4



\section{V-AQA模型的各要素与建议解决方案}
\label{sec:cac_thanh_to_vqa}

V-AQA模型由五个相互作用的要素构成,每个要素代表一组战略性解决方案,旨在解决第三章中已分析的挑战。接下来的部分将深入探讨每个要素,论证其建议的解决方案及其背后的科学与实践依据。

\subsection{要素一:以分权和责任为导向重构领导与治理}
\label{subsec:giaiphap_lanhdao}

如前所述,当前质量保障体系的最大障碍之一是领导角色带有浓厚的行政、合规色彩,且权力过于集中\footcite{vnujs_fs_4303}。因此,V-AQA模型的第一个要素聚焦于重构这一角色,目标是将领导者从“控制者”转变为质量文化的“环境创造者”和“推动者”\footcite{unesco_gem_report_2024}。

\paragraph{解决方案1.1:制定并执行校级质量战略。}
为摆脱季节性的应付状态,每所大学都需要制定一份正式的五年期\textbf{质量战略},并将其紧密融入学校的总体发展战略中。这并非一份形式化的文件,而是最高层领导的政治承诺。该战略必须回答以下核心问题:
\begin{enumerate}
    \item 学校从哪些维度定义“质量”(研究质量、教学质量、市场契合度、学生体验)?
    \item 未来五年的优先质量目标是什么,并以可衡量的关键绩效指标具体化?(例如:将学生就业率提高到95%,国际发表数量每年增加20%)。
    \item 将分配哪些资源(财政、人力、技术)来实现目标?
    \item 谁为每个具体目标负责(明确各处、室、院系的角色分工)?
\end{enumerate}
拥有一份清晰且广为宣传的战略,是有效和负责任地执行自主权的先决条件,正如教育培训部关于大学自主的报告所强调的\footcite{moet_report_autonomy}。

\paragraph{解决方案1.2:按“指导核心”模型进行分权并明确授权。}
基于委托代理理论\footcite{JensenMeckling1976},V-AQA模型提出了一种更强有力的分权机制,并辅以明确的问责制。学校需要从集权结构转向克拉克(1998)所分析的“指导核心”模型\footcite{clark_1998}。据此:
\begin{itemize}
    \item \textbf{校董会:} 专注于批准总体战略,监督战略目标的执行,并确保学校对社会的问责。
    \item \textbf{校领导班子:} 负责运营,制定支持性政策,并有效分配资源,以便各单位能够实现目标。
    \item \textbf{各院/系:} 必须在学术事务,如改进培养方案、创新教学方法和开展研究活动等方面,被赋予实质性的自主权。
\end{itemize}
这种自主权必须与一个清晰透明的关键绩效指标体系相结合。校领导班子使用该体系(通过质量保障管理信息系统上的仪表盘)来监督成效,而无需对具体的专业活动进行微观干预,这符合“远程指导”的精神。

\paragraph{解决方案1.3:提升中层管理团队的能力。}
一个好的战略无法由一个薄弱的管理团队来执行。因此,V-AQA模型强调投资于中层管理团队(院/系主任、副主任)能力的必要性。需要设计强制性的培训项目,不仅涉及行政业务,还包括现代大学治理技能,如:
\begin{itemize}
    \item 变革管理
    \item 战略思维与规划
    \item 数据驱动决策
    \item 激励型领导力
\end{itemize}
政府的\textbf{89号提案}等海外博士师资培训项目,应被利用并扩展,以包括为潜在管理者提供关于大学治理的培训和进修项目\footcite{moet_project_89}。这是对“人力资本”的投资,以确保改革从根本上取得成功。

\subsection{领导与治理要素的风险分析与缓解方案}
\label{subsec:risk_lanhdao}
重构领导与治理角色总是潜藏着重大风险。识别并制定预防这些风险的策略是确保成功的关键。

\begin{itemize}
    \item \textbf{风险1:高层领导的承诺浮于表面。}
    \textit{表现:} 领导可能在文件上支持改革,但行动上并不坚决,不提供足够资源,或在遇到困难时轻易改变优先事项。
    \textit{缓解策略:}
    \begin{itemize}
        \item 成立一个由校长或常务副校长直接担任组长的\textbf{质量改革指导委员会},有定期的会议日程和公开的报告。
        \item 为\textbf{校领导班子和各单位领导制定并颁布关键绩效指标},其中执行质量战略的指标在年终评优奖励中占有重要比重。
    \end{itemize}

    \item \textbf{风险2:来自中层管理的抵制。}
    \textit{表现:} 院系主任、处长可能会觉得在一个更透明的体系下运作会失去权力。他们可能会通过不合作、拖延实施或提供不准确的数据来进行暗中抵制。
    \textit{缓解策略:}
    \begin{itemize}
        \item 为100%的中层管理干部组织关于\textbf{变革管理和数据驱动治理的强制性培训},使他们清楚理解新模型的益处。
        \item \textbf{授权与赋能并行。} 当一个院系被分配了更高的关键绩效指标时,也必须在财务和人事方面赋予其相应的自主权,以便能够完成任务。
        \item \textbf{沟通与倾听:} 组织校领导班子与中层管理之间的定期对话会,倾听困难并共同寻找解决方案,而不是单向命令。
    \end{itemize}

    \item \textbf{风险3:分权但无法控制成效。}
    \textit{表现:} 过度下放自主权而没有一个有效的监督体系,可能导致院/系运作偏离方向,无法为学校的总体目标做出贡献。
    \textit{缓解策略:}
    \begin{itemize}
        \item \textbf{部署质量保障管理信息系统是先决条件。} 如果没有一个足够强大的信息系统来为校领导班子提供及时准确的数据,就无法进行分权。
        \item 建立一个\textbf{自上而下统一的、清晰的关键绩效指标体系},确保院/系/处室级别的指标都旨在实现校级战略指标。
    \end{itemize}
\end{itemize}

通过预见并主动管理上述风险,领导与治理要素可以成为强大的动力,而不是障碍,推动整个质量改革进程。


% het goi 3



\subsection{要素二:从合规文化向改进文化的转型}
\label{subsec:giaiphap_vanhoa}

“反应型质量文化”的挑战——即质量保障活动仅为应对外部认证要求而进行——已被确定为阻碍越南大学实质性质量提升的最大障碍之一\footcite{vjol_reactiveculture}。因此,V--AQA模型的第二个要素聚焦于制定战略性解决方案,以创造一个环境,使质量改进成为组织内从领导到员工每个成员的内在需求和日常习惯。

\paragraph{解决方案2.1:基于欧洲大学协会能力框架发起质量文化建设运动。}
文化不能靠命令创造,而必须通过共识和共享价值观来培育。V--AQA模型建议各大学基于\textbf{欧洲大学协会的质量文化框架},开展系统的宣传和意识提升运动\footcite{eua_quality_culture}。根据该能力框架,质量文化建立在机构层面和个人层面因素的互动之上。
\begin{itemize}
    \item \textbf{在机构层面:} 学校领导必须率先垂范,不断宣传质量的愿景、定义和重要性。关于质量的核心价值观必须被整合到发展战略、政策文件和学校的运作流程中。
    \item \textbf{在个人层面:} 运动需要侧重于提升每位教师、员工的意识、态度和能力。具体活动包括:组织研讨会、关于质量思维的培训课程,以及为坦诚对话质量改进中的障碍与机遇而设的开放论坛。
\end{itemize}
该运动的目标是让所有成员明白,质量不仅是质量保障部门的责任,而是每个人的责任。

\paragraph{解决方案2.2:建立改进创举的表彰与奖励体系。}
要将认知转化为行动,需要有实际的激励。V--AQA模型建议建立一个正式的表彰与奖励体系,以嘉奖质量改进的努力。正如哈维和威廉姆斯(2010)所指出的,创建激励机制是维持质量活动的重要组成部分\footcite{harvey_williams_2010}。具体而言,学校可以:
\begin{itemize}
    \item \textbf{设立年度奖项:} “教学创新年度教师奖”、“高效质量保障模式院系奖”、“最佳质量改进创举奖”。这些奖项需要有足够大的物质和精神价值以产生吸引力。
    \item \textbf{融入职业评估与发展流程:} 对质量改进的贡献标准必须被正式纳入年终干部、教师的评估、评级和评优奖励流程中。在规划、任命干部或考虑提前晋升时,这是一个重要的考量因素。
    \item \textbf{公开表彰:} 取得优异成绩的个人和集体需要在学校的宣传渠道(网站、简报、社交媒体)上隆重表彰,以传播良好榜样。
\end{itemize}
这些行动将发出一个强有力的信息:学校真正重视并奖励改进的努力,而不仅仅是看重完成行政任务。

\paragraph{解决方案2.3:通过“改进小组”为教师赋能并建设其能力。}
质量文化必须从基层建立。为解决教师“被动参与”的状况\footcite{iosr_passiveparticipation},V--AQA模型建议授权并创造条件,让他们成为改进过程的主体。具体解决方案是在系或院层面成立\textbf{“质量圈”}。这些是由自愿参加的教师组成的小组,任务是定期讨论教学、研究中的质量问题并提出解决方案。学校需要为这些小组提供小额预算(创举支持基金),以便他们可以试验新的教学方法、开发创新学习材料或组织专业研讨会。这种授权和资源提供将促进一种主人翁精神,将教师从被管理者转变为质量的创造者。

\subsection{质量文化要素的风险分析与缓解方案}
\label{subsec:risk_vanhoa}
改变组织文化是最困难和最长期的挑战之一。实施上述解决方案可能会面临以下风险:

\begin{itemize}
    \item \textbf{风险1:团队的惰性和怀疑态度。}
    \textit{表现:} 教师、员工已习惯于旧的工作方式,认为各种运动和活动只是形式主义,“雷声大雨点小”。他们可能会不情愿地参与,并且不相信会发生实质性变化。
    \textit{缓解策略:}
    \begin{itemize}
        \item \textbf{领导的模范作用:} 领导必须率先参与所有活动,从研讨会到提出创举。领导的行动比任何言语都更有说服力。
        \item \textbf{专注于“速赢”:} 在初期阶段,需要集中支持小型改进小组,以创造出积极、可见的成果。广泛宣传并奖励这些初步成功,将为改革过程创造动力并巩固信心\footcite{kotter_leading_change}。
    \end{itemize}

    \item \textbf{风险2:奖励体系不公平或吸引力不足。}
    \textit{表现:} 奖励评审过程不透明,带有主观性,或者奖励过小,无法产生真正的激励。这可能适得其反,造成不满和嫉妒。
    \textit{缓解策略:}
    \begin{itemize}
        \item \textbf{建立清晰、公开的创举评估标准:} 标准需要能够衡量创举对提升质量的影响(例如:改善学生学习成果,缩短手续办理时间)。
        \item \textbf{多样化奖励形式:} 除了财务奖励,还需要有其他形式的认可,如参加国际培训的机会,在职称评定中优先考虑,或在学校重大活动中获得表彰。
    \end{itemize}

    \item \textbf{风险3:“改进小组”运作效率低下。}
    \textit{表现:} 小组成立后没有实质性活动,只是形式上的会议,或者小组的建议未得到上级的倾听和支持。
    \textit{缓解策略:}
    \begin{itemize}
        \item \textbf{提供必要的资源和支持:} 学校需要设立一个申请和拨付流程简单的“创举支持基金”。同时,需要有一个单位(例如:质量保障处)扮演协调角色,为各小组提供专业支持。
        \item \textbf{建立正式的反馈机制:} 必须有一个清晰的流程,让院/校领导倾听并回应来自改进小组的建议。建议被认真考虑并付诸实施,将是小组继续运作的最大动力。
    \end{itemize}
\end{itemize}

建设质量文化是一场马拉松,而不是短跑。它需要学校各级人员的毅力、一致性和真正的承诺。

% het goi 4 chuong 4



\subsection{要素三:将利益相关者转变为战略伙伴}
\label{subsec:giaiphap_lienquan}

第三章已指出,越南质量保障体系的固有弱点之一是与利益相关者,特别是企业界和学习者本身,缺乏实质性联系\footcite{worldbank_improvingperformance}。这种关系通常停留在形式化、被动的层面,如单向的咨询活动或影响甚微的调查\footcite{vnujs_er_3848}。V-AQA模型的第三个要素提出了旨在打破学校与社会之间“壁垒”的解决方案,将关系从“形式化咨询”转变为“战略合作”。

\paragraph{解决方案3.1:通过咨询委员会将校企联系机制化。}
为使与企业的合作不再依赖于少数领导的个人关系或季节性活动,V-AQA模型提出了一个制度性解决方案:为每个专业或专业群成立并运作\textbf{行业咨询委员会}。这是在东盟大学网络质量保障或ABET等权威国际认证标准中的一项核心要求,这些标准强调利益相关者在设计和改进培养方案中的中心作用\footcite{aunqa_guidelines_v4}。

行业咨询委员会的作用不仅限于参加研讨会。其运作章程需要明确制定,规定具体且具有约束力的任务:
\begin{itemize}
    \item \textbf{审定预期学习成果:} 行业咨询委员会负责审定并对专业院系制定的预期学习成果提出建议,确保其真实反映劳动力市场当前和未来的能力要求(知识、技能、态度)。
    \item \textbf{为培养方案提供定期建议:} 每年至少一次,专业院系必须向行业咨询委员会提交培养方案的审查报告,以获取关于技术、软技能和行业新趋势的更新建议。这些建议必须被记录并有相应的回应计划。
    \item \textbf{参与学术活动:} 邀请行业咨询委员会成员参与毕业设计答辩委员会;在专业课程中担任客座讲师;或为学生提出实际的项目课题。
\end{itemize}
这种机制化将创造一个可持续的对话与合作渠道,确保培养方案始终“跟上”实践,解决培养与市场“不匹配”的问题\footcite{worldbank_improvingperformance}。

\paragraph{解决方案3.2:提升学生和校友的角色。}
除了企业,学生和校友是对一个培养方案质量有最深刻见解的利益相关者。然而,他们通常被视为质量保障体系中的被动对象\footcite{pmc_article_9127449}。V-AQA模型基于阿恩斯坦的“参与阶梯”精神,提出了将他们的角色从被调查对象提升为真正合作伙伴的解决方案\footcite{Arnstein1969}。
\begin{itemize}
    \item \textbf{为学生赋权:} 学生代表(通过民主方式选举产生)不应只参与活动性组织,而应在像院级科学与培养委员会这样的学术性委员会中拥有一个带有\textbf{投票权}的正式席位。这确保了学习者的声音在关于课程、教学方法和学生支持政策等重要决策中得到认真考虑。
    \item \textbf{建立主动的校友网络:} 为每个专业建立一个\textbf{“校友大使”}体系。这些是事业有成且热心的校友,被正式邀请参与为低年级学生提供咨询和职业指导。更重要的是,他们将是定期提供关于培养方案在实际工作中的契合度的结构化反馈的来源。来自学生就业调查的数据\footcite{moet_graduate_survey}将是该网络有效运作的重要输入。
\end{itemize}

\subsection{利益相关者要素的风险分析与缓解方案}
\label{subsec:risk_lienquan}
加强利益相关者的参与,尽管非常必要,但在实施过程中也面临不少风险。

\begin{itemize}
    \item \textbf{风险1:企业参与流于形式,不具实质性。}
    \textit{表现:} 行业咨询委员会成员可能因人情关系而接受邀请,但不花时间研究文件并提出肤浅的建议。他们将此视为一项“社交”活动而非专业责任。
    \textit{缓解策略:}
    \begin{itemize}
        \item \textbf{建立“双赢”关系:} 除了为社会做贡献,学校需要为企业创造实际利益,如:为他们的员工提供进修课程,提供招聘品牌宣传机会,或优先对接研究和技术转让项目。
        \item \textbf{谨慎选择行业咨询委员会成员:} 优先选择真正热心于教育、具有深厚专业知识并愿意承诺时间的专家。行业咨询委员会的运作章程需要有替换不活跃成员的条款。
    \end{itemize}

    \item \textbf{风险2:学生代表的冷漠或能力不足。}
    \textit{表现:} 学生被赋予权力但因胆怯、缺乏批判性思维能力或未被提供足够信息以提出建设性意见而不发言。
    \textit{缓解策略:}
    \begin{itemize}
        \item \textbf{为学生代表组织技能培训课程:} 在参加委员会之前,学生代表需要接受关于批判性思维、如何阅读和分析政策文件以及有效沟通技巧的培训。
        \item \textbf{充分、及时地提供信息:} 委员会会议的材料必须在合理的时间前发送给学生代表,以便他们可以研究并收集其他学生的意见。
    \end{itemize}

    \item \textbf{风险3:利益和观点的冲突。}
    \textit{表现:} 来自企业的建议可能过分集中于眼前的实践技能,轻视基础知识。反之,校方可能固守己见,不愿改变已稳定的培养方案。
    \textit{缓解策略:}
    \begin{itemize}
        \item \textbf{院系/单位负责人的协调作用:} 单位负责人必须扮演“桥梁”角色,有能力协调、分析和融合不同观点,以找到平衡学术与实践要求的最佳解决方案。
        \item \textbf{基于数据做决策:} 所有争论都需要有客观数据的支持,例如就业率数据、校友调查以及劳动力市场趋势分析报告。
    \end{itemize}
\end{itemize}

将利益相关者转变为合作伙伴不仅仅是一个权宜之计,而是一项长期的战略投资。一旦成功,它将创造一个良性循环,其中培养质量不断改进以满足社会的实际需求。

% het goi 5 chuong 4


\subsection{要素四:核心学术流程的现代化}
\label{subsec:giaiphap_quytringnoibo}

为从根本上解决第三章所分析的培养与实践“脱节”问题,改革核心学术流程是一项强制性任务。这些流程正是学校的“生产机器”;如果它们落后,产出产品(毕业生)将难以满足现代社会的要求。因此,V-AQA模型的第四个要素聚焦于两个基础流程的标准化和现代化:(1)培养方案的开发,以及(2)教与学及评估的方法。

\subsubsection{解决方案4.1:采用基于成果导向教育的培养方案开发流程。}
V-AQA模型建议各大学完全转向基于\textbf{成果导向教育}理念的培养方案开发方法。这是东盟大学网络质量保障和ABET等世界上大多数权威认证组织所采用的方法\footcite{aunqa_guidelines_v4}。该理念将重点放在学生毕业后所能达到的成果,而不是仅仅关注需要传授的知识内容。成果导向教育有助于回答“学生能做什么?”这个问题,而不仅仅是“学生知道什么?”。

肤浅地应用成果导向教育,仅仅停留在罗列预期学习成果而没有在整个课程中实现紧密联系,是越南大学正在面临的困难之一\footcite{ijlter_elo_copy}。因此,V-AQA建议一个基于成果导向教育的培养方案开发流程必须包含以下五个紧密相连的步骤:

\begin{enumerate}
    \item \textbf{第一步:利益相关者需求分析。} 流程始于系统地收集和分析来自劳动力市场(通过报告、调查)、行业专家(通过访谈、研讨会)、成功校友和其他利益相关者的要求,以确定学生毕业后所需的能力框架。
    
    \item \textbf{第二步:预期学习成果的制定与审定。} 基于需求分析,专业院系将制定一套清晰、具体、可衡量(SMART)的课程预期学习成果。重要的是,这套预期学习成果必须经过\textbf{行业咨询委员会的审定和建议},以确保其适切性和实践性。
    
    \item \textbf{第三步:设计课程地图。} 这是确保培养方案逻辑性和一致性的关键步骤。构建一个矩阵,清晰地展示课程中的每一门课如何为实现一个或多个预期学习成果做出贡献。这种技术,也被称为“建设性对齐”\footcite{biggs_constructive_alignment},确保了没有任何一门课程是孤立存在的,并且所有的教-学-评活动都指向已确定的预期学习成果。
    
    \item \textbf{第四步:审定与批准。} 设计完成的培养方案(包括预期学习成果、课程地图、各课程详细大纲)必须经过一个权威的委员会(科学与培养委员会,包括外部专家)的审定和批准,然后才能正式颁布。
    
    \item \textbf{第五步:定期审查与改进。} 成果导向教育是一个持续的循环。学校必须建立一个正式的培养方案审查周期(例如:每2-3年一次),基于重要的输入数据,如:来自行业咨询委员会的反馈、毕业生对工作满足度的调查、学生的实际学习成果以及行业的新趋势。
\end{enumerate}

\subsubsection{解决方案4.2:多样化教与学及评估方法。}
一个按照成果导向教育精心设计的培养方案,需要通过有效的教学和评估方法来实现,这些方法应能够衡量预期学习成果。为克服重理论教学、评估主要靠期末考试的状况,V-AQA模型提出以下解决方案:

\begin{itemize}
    \item \textbf{鼓励积极学习方法:} 学校需要颁布具体政策,以鼓励(通过奖励机制、减少标准课时、提供经费支持)教师应用以学习者为中心的方法。这些方法有助于发展复杂技能,而不仅仅是记忆知识。典型例子包括:
    \begin{itemize}
        \item \textbf{基于项目的学习:} 学生执行长期的实际项目以解决一个复杂问题。
        \item \textbf{基于问题的学习:} 学生分组解决一个开放性问题,从而自主探索和构建知识。
        \item \textbf{混合式学习:} 将在线学习活动与课堂教学相结合,以优化学习者体验。教育培训部关于数字化转型的报告也强调了此项的必要性\footcite{moet_digital_transformation}。
    \end{itemize}
    
    \item \textbf{改革学习成果评估规定:} 评估是衡量预期学习成果达成度的最重要工具。因此,评估规定需要朝以下方向改革:
    \begin{itemize}
        \item \textbf{增加过程性评估的比重:} 新规定应要求\textbf{过程性评估的比重至少占课程总成绩的40-50\%}。这有助于跟踪学生的进步并提供及时反馈,而不是只关注期末的总结性评估。
        \item \textbf{多样化评估形式:} 规定应鼓励和承认适合不同类型能力的多种评估形式,例如:项目评估、实习报告、演讲、实践产品、作品集和同行评估。
    \end{itemize}
\end{itemize}

同步改革培养方案和教-学、评估方法,将创造一个实质性的学术环境,其中所有活动都有目的地联系在一起,以帮助学生全面发展其在预期学习成果中承诺的能力。

\subsection{学术流程的风险分析与缓解方案}
\label{subsec:risk_hocthuat}
改革核心学术流程是一个复杂的过程,常常会遇到巨大的阻力。

\begin{itemize}
    \item \textbf{风险1:教师能力不足且不愿改变。}
    \textit{表现:} 许多教师习惯于传统教学方法(以讲授为主),没有足够的技能、时间或动力来按照基于项目的学习或其他积极方法重新设计课程。按照成果导向教育构建培养方案也需要并非人人都具备的新技能\footcite{ijlter_elo_copy}。
    \textit{缓解策略:}
    \begin{itemize}
        \item \textbf{建立系统的教师发展项目:} 组织关于按成果导向教育设计培养方案、积极教学方法和多样化评估技术的深入“手把手”培训课程。
        \item \textbf{创建具体的激励机制:} 实施新方法的第一个学期提供减少标准课时的政策,或提供小额经费以开发新教材。
        \item \textbf{建立实践社群:} 创建论坛,让教师们可以在创新过程中分享经验、困难并相互学习。
    \end{itemize}

    \item \textbf{风险2:行政规定和基础设施跟不上。}
    \textit{表现:} 财务规定不允许为项目式学习活动灵活支出。教室按传统方式设计,不适合小组合作。信息技术系统不够强大,无法支持混合式学习。
    \textit{缓解策略:}
    \begin{itemize}
        \item \textbf{同步审查和调整相关规定:} 质量保障处必须与计划财务处和行政综合处协调,调整过时的规定,为创新创造一个通畅的法律走廊。
        \item \textbf{有重点地投资于物质设施和基础设施:} 制定一个长期计划,升级教室、图书馆和信息技术基础设施,以服务于新的教与学方法。
    \end{itemize}
\end{itemize}


% het goi 6 chuong 4



\subsubsection{解决方案4.3:构建质量保障管理信息系统}
\label{subsubsec:giaiphap_qamis}

如第三章所分析,数据管理和使用的薄弱是质量保障体系的“阿喀琉斯之踵”,造成了缺乏依据的决策恶性循环。为打破此循环,V-AQA模型提出了一个基础性的技术解决方案:构建一个集成的\textbf{质量保障管理信息系统}。这不仅是一个技术工具,更是整个模型的“神经系统”,是从基于经验的治理转向基于证据的治理的手段。

\paragraph{建议的系统架构。}
基于国际大学的最佳实践\footcite{seaairweb_journal_v22}以及欧洲质量保障协会的建议\footcite{enqa_forum_report_201X},一个有效的质量保障管理信息系统需要包含三个主要层次的架构:数据源层、处理与存储层、以及呈现与分析层(见图\ref{fig:kien_truc_qamis})。

\begin{figure}[h!]
    \centering
    \includegraphics[width=\textwidth]{image/mo_hinh_V-AQA.jpg}
    \caption{混合与适应性质量保障模型 (V-AQA)}
    \label{fig:v-aqa-model-detailed}
\end{figure}

该系统需设计为能自动整合校内多个分散来源的数据,以创建关于质量的360度视图,包括:
\begin{itemize}
    \item \textbf{教学管理系统:} 学生数据、成绩、学习进度、辍学率。
    \item \textbf{人事系统:} 教师的学历、经验、专业发展活动和科研成果数据。
    \item \textbf{科技管理系统:} 研究课题、经费、科学论文数据。
    \item \textbf{在线调查系统:} 来自学生、教师、校友和雇主满意度调查的数据。
    \item \textbf{财务系统:} 用于教学、研究活动的预算分配和使用数据。
\end{itemize}

\paragraph{输出产品:智能报告与仪表盘。}
质量保障管理信息系统的最终目标不是存储数据,而是将数据转化为对各级决策有用的信息。为实现此目标,系统需配备一个\textbf{商业智能}模块,以自动生成可视化的报告和仪表盘,并为不同用户角色进行定制和授权\footcite{uq_kpi_dashboard}。

为更清晰地说明其可行性以及V-AQA模型如何通过数字工具运作,以下是为质量保障管理信息系统上不同角色定制的仪表盘示例。

\begin{table}[h!]
\centering
\caption{校长/校领导班子在质量保障管理信息系统上的关键绩效指标仪表盘示例}
\label{tab:dashboard_hieu_truong}
\begin{tabular}{|p{4.5cm}|p{2.5cm}|p{2.5cm}|p{2cm}|c|}
\hline
\textbf{战略指标(校级)} & \textbf{当前值} & \textbf{与上期比较} & \textbf{年度目标} & \textbf{状态} \\
\hline
12个月后学生就业率 & 92.5\% & $\blacktriangle$ 1.5\% & 95\% & 预警 \\
\hline
国际发表数量(Scopus/ISI) & 350 & $\blacktriangle$ 10\% & 400 & 达标 \\
\hline
雇主满意度 & 4.2/5 & $\blacktriangledown$ 0.1 & 4.5/5 & 未达标 \\
\hline
国际学生比例 & 8\% & $\blacktriangle$ 0.5\% & 10\% & 预警 \\
\hline
科研与技术转让收入 & 500亿 & $\blacktriangle$ 5\% & 600亿 & 预警 \\
\hline
\end{tabular}
\end{table}

\begin{table}[h!]
\centering
\caption{院长在质量保障管理信息系统上的关键绩效指标仪表盘示例}
\label{tab:dashboard_truong_khoa}
\begin{tabular}{|p{5cm}|p{2.5cm}|p{2.5cm}|p{2cm}|c|}
\hline
\textbf{运营指标(信息技术学院)} & \textbf{当前值} & \textbf{与上期比较} & \textbf{年度目标} & \textbf{状态} \\
\hline
按时毕业率 & 85\% & $\longleftrightarrow$ & 90\% & 预警 \\
\hline
学生对培养方案的满意度 & 3.9/5 & $\blacktriangle$ 0.2 & 4.2/5 & 预警 \\
\hline
院级科研课题数量 & 15 & $\blacktriangle$ 2 & 20 & 未达标 \\
\hline
参与培养方案改进的教师比例 & 60\% & $\blacktriangle$ 10\% & 80\% & 未达标 \\
\hline
\end{tabular}
\end{table}

\begin{table}[h!]
\centering
\caption{教师在质量保障管理信息系统上的关键绩效指标仪表盘示例}
\label{tab:dashboard_giang_vien}
\begin{tabular}{|p{5.5cm}|p{2.5cm}|p{3.5cm}|c|}
\hline
\textbf{个人指标(阮文A教师)} & \textbf{本学期} & \textbf{与全院平均比较} & \textbf{备注} \\
\hline
学生对教学的反馈得分 & 4.5/5 & +0.6 & 优秀 \\
\hline
已完成/计划科研时数 & 120/200 小时 & -40 小时 & 需改进 \\
\hline
学生课程通过率 & 95\% & +5\% & 良好 \\
\hline
专业发展活动 & 2门课程 & +1 & 良好 \\
\hline
\end{tabular}
\end{table}
    

部署一个质量保障管理信息系统不仅是一个技术解决方案,更是一场工作方式的改革。它要求领导层的坚定承诺、相应的资源投入,以及一个系统的培训计划,以使全体团队能够有效利用数据的力量。

\subsection{质量保障管理信息系统的风险分析与缓解方案}
\label{subsec:risk_qamis}
尽管带来巨大益处,部署质量保障管理信息系统也面临许多技术和人为风险。

\begin{itemize}
    \item \textbf{风险1:投资和维护成本超出能力。}
    \textit{表现:} 学校没有足够预算来构建、购买许可和维护一个复杂的信息系统。升级、维护和运营人员等隐性成本未被计算在内。
    \textit{缓解策略:}
    \begin{itemize}
        \item \textbf{分阶段实施:} 从最核心的模块开始,如学生数据管理和在线调查。更复杂的模块如商业智能将在证明其有效性后于后续阶段开发。
        \item \textbf{考虑开源解决方案:} 研究并利用开源平台以降低软件许可费用,将预算集中于定制和培训。
    \end{itemize}

    \item \textbf{风险2:数据不同步、不准确(垃圾进,垃圾出)。}
    \textit{表现:} 来自不同部门的数据在格式和可靠性上不一致。如果将“垃圾”数据输入系统,报告和分析将变得毫无意义。
    \textit{缓解策略:}
    \begin{itemize}
        \item \textbf{建立数据治理规定:} 颁布关于数据录入责任、格式标准化以及在集成到统一系统前对各部门数据进行检查、验证流程的明确规定。
        \item \textbf{投资于ETL(提取、转换、加载)过程:} 数据的清洗、转换和集成过程极其重要,需要投入相应的技术资源。
    \end{itemize}
    
    \item \textbf{风险3:来自最终用户的抵制。}
    \textit{表现:} 干部、教师不想使用新系统,因为它复杂、增加了工作量,或者他们感到被“监控”得太紧。
    \textit{缓解策略:}
    \begin{itemize}
        \item \textbf{设计友好的界面并关注用户利益:} 系统设计必须旨在帮助用户减少手工工作(例如:自动生成报告而非手动制作),而不仅仅是服务于管理层。
        \item \textbf{组织持续的培训和支持:} 建立一个支持团队(服务台),并提供直观易懂的指导材料。为特定用户群体组织实践培训。
    \end{itemize}
\end{itemize}

% het goi 7 chuong 4

\subsection{要素五:构建一个开放的质量生态系统}
\label{subsec:giaiphap_hoptac}

正如第三章所分析的,“孤岛”状态以及同一所学校内各单位之间、各大学之间缺乏合作,是降低整个体系整体效率的一大障碍。V-AQA模型的第五个也是最后一个要素,聚焦于旨在打破封闭竞争思维、推动一个开放质量生态系统的解决方案,在此生态系统中,各大学、单位和外部伙伴本着“学习型组织”的精神共同学习和发展\footcite{Senge2006}。

\paragraph{解决方案5.1:通过跨学科改进项目促进内部合作。}
为打破各院、处、室之间无形的“壁垒”,学校需要有政策鼓励并资助有\textbf{多个单位参与的质量改进项目}。复杂的质量问题通常无法由单个单位解决。例如:
\begin{itemize}
    \item 一个\textbf{“提升学生软技能”}的项目,需要学生支持中心、共青团、各专业院系(以融入课程)和质量保障处(以衡量效果)的协调。
    \item 一个\textbf{“建设数字学习资源系统”}的项目,需要图书馆(提供平台)、信息技术中心(技术支持)和所有院系(建设内容)的合作。
\end{itemize}
学校需要设立一个\textbf{专门资助跨学科项目的基金},其优先标准是各单位之间实质性合作的数量和程度。资助这些项目不仅能解决复杂问题,还能创造共同的工作机制,增进各单位之间的理解与协调,打破“各自为政”的局部思维。

\paragraph{解决方案5.2:建立校际标杆比对网络。}
V-AQA模型建议各大学主动建立网络以相互学习,而不是进行默然且无方向的竞争。具体是成立按专业领域划分的\textbf{质量标杆比对俱乐部}(例如:经济类院校群、工科类院校群、师范类院校群)。基于杰克逊和伦德(2000)的模型,这些网络可以基于以下原则运作\footcite{jackson_lund_2000}:
\begin{itemize}
    \item \textbf{(匿名)数据共享:} 成员学校共同商定一套质量指标(例如:生师比、生均培养成本、就业率),并将其数据匿名分享给一个中立的第三方(例如:一个有信誉的认证中心或一个行业协会)进行汇总和分析。
    \item \textbf{比较报告:} 第三方将提供比较报告,帮助每所学校了解自己相对于群体平均水平的位置(例如:处于前25%、平均水平或后25%),而无需知道其他每所学校的具体数据。
    \item \textbf{同行评审与最佳实践分享:} 组织定期的同行评审活动,即一所学校的专家本着建设性和学习的精神访问另一所学校并提出建议。在某一特定方面指标最好的学校将被邀请向其他成员分享经验和最佳实践。
\end{itemize}
该机制将把竞争转化为学习的动力,帮助各校客观地认识自己的优势和劣势,并推动整个系统的改进。

\paragraph{解决方案5.3:战略性地利用国际合作。}
参与国际认证不应仅仅停留在获得证书以提升品牌的目标上。它必须被视为一个全面学习和改进系统的战略机会。V-AQA模型建议各校需要有一个清晰的流程,以最大限度地利用这些活动的价值。在每次接受东盟大学网络质量保障、ABET或FIBAA等组织的认证后,质量保障处必须负责:
\begin{itemize}
    \item 深入分析评估报告,特别是建议和需要改进的领域。
    \item 主持制定一个\textbf{具体的行动计划}来解决这些建议,有明确的责任分工、路线图和衡量指标。该计划必须被整合到质量保障管理信息系统中以跟踪进度。
    \item 定期(例如:每6个月一次)向校领导班子报告并向相关单位公开该行动计划的执行进度。
\end{itemize}
这种方法将把每一次国际认证都转变为一个实质性的改进周期,帮助学校的质量保障体系日益趋近国际标准。世界银行支持的SAHEP等项目也是实现此目标所需有效利用的重要资源\footcite{worldbank_sahep}。

\subsection{合作与协调要素的风险分析与缓解方案}
\label{subsec:risk_hoptac}
在大学竞争日益激烈的背景下,推动合作是一项巨大挑战。

\begin{itemize}
    \item \textbf{风险1:“各自为政”文化与内部竞争。}
    \textit{表现:} 各院、处、室倾向于只关注本单位的利益,不愿分享资源或与其他单位协调,视其他单位为预算分配的竞争对手。
    \textit{缓解策略:}
    \begin{itemize}
        \item \textbf{设计智能的财务机制:} 设立专门基金,仅用于有两个或以上单位参与的项目。将“跨单位合作程度”标准纳入评估院、处领导绩效的关键绩效指标体系。
        \item \textbf{高层领导的干预与协调:} 校领导班子需要主动发起和协调具有全校意义的战略项目,要求多个单位强制性参与。
    \end{itemize}

    \item \textbf{风险2:各校不愿分享数据进行比对。}
    \textit{表现:} 担心分享数据(即使是匿名的)可能会暴露弱点,影响学校的声誉、排名或招生工作。
    \textit{缓解策略:}
    \begin{itemize}
        \item \textbf{通过一个有信誉的第三方建立信任:} 一个独立、有信誉的第三方(例如:一个认证中心、一个协会)的角色对于确保数据的保密性和匿名性至关重要。
        \item \textbf{从敏感度较低的指标开始:} 在初期阶段,标杆比对俱乐部可以从分享关于流程或最佳实践的指标开始,而不是像财务或招生这样“敏感”的指标,以逐步建立信任。
    \end{itemize}

    \item \textbf{风险3:形式化地利用国际合作。}
    \textit{表现:} 学校仅将国际认证视为一种营销活动。获得证书后,评估报告和建议被“束之高阁”,没有任何改进活动被执行。
    \textit{缓解策略:}
    \begin{itemize}
        \item \textbf{将“认证后”流程制度化:} 颁布学校的正式规定,要求质量保障处在收到报告后3个月内提交基于认证结果的行动计划。
        \item \textbf{将执行结果与关键绩效指标挂钩:} 完成外部评估团建议的纠正行动,必须被视为质量保障处及相关单位的一个重要关键绩效指标。
    \end{itemize}
\end{itemize}


% het goi 8


\section{V-AQA模型的实施路线图与条件}
\label{sec:lo_trinh_trien_khai}

一个改革模型,无论多么全面,如果缺乏一个现实可行的实施路线图,都将失败。V-AQA模型并非一场可以一蹴而就的“大手术”,而是一个系统的转型过程,需要时间、毅力和一个合理分阶段的计划。基于区域内各国高等教育改革的经验以及世界银行的建议\footcite{worldbank_reform_agenda},本论文提出了一个为期7年的V-AQA模型实施路线图,分为3个主要阶段。

\subsection{分阶段实施路线图建议}
\label{subsec:lo_trinh_giai_doan}

下表详细设计了实施路线图,包括每个阶段的目标、战略行动、关键绩效指标以及资源要求。这种方法有助于确保可行性、透明度,并允许在执行过程中进行灵活调整。

\begin{longtable}{|p{1.5cm}|p{1.5cm}|p{2.8cm}|p{3.5cm}|p{3cm}|p{2cm}|}
\caption{V-AQA模型实施路线图(2025 - 2032年)}
\label{tab:lo_trinh_vqa}\\
\hline
\textbf{阶段} & \textbf{时间} & \textbf{主要目标} & \textbf{战略行动} & \textbf{主要衡量关键绩效指标} & \textbf{资源(估算)} \\
\hline
\endfirsthead
\multicolumn{6}{c}%
{{\bfseries \tablename\ \thetable{} -- 续前页}} \\
\hline
\textbf{阶段} & \textbf{时间} & \textbf{主要目标} & \textbf{战略行动} & \textbf{主要衡量关键绩效指标} & \textbf{资源(估算)} \\
\hline
\endhead
\hline \multicolumn{6}{r}{{续下页}} \\
\endfoot
\hline
\endlastfoot

% 阶段 1
\textbf{1. 基础与试点} & 2年 (2025-2027) & 建立核心能力,形成共识,并小规模试验模型。 & 
\begin{itemize}
    \item 成立质量改革指导委员会。
    \item 对100%的管理者进行V-AQA意识培训。
    \item 构建质量保障管理信息系统的初始版本。
    \item 选择2-3个院系全面试点5个要素。
\end{itemize} &
\begin{itemize}
    \item 100%的院级及以上管理者接受培训。
    \item 质量保障管理信息系统基本投入运作。
    \item 试点单位报告初步成功(“速赢”)。
    \item 试点单位的满意率。
\end{itemize} &
\begin{itemize}
    \item 质量保障管理信息系统建设成本(第一阶段)。
    \item 培训、研讨会组织成本。
    \item 指导委员会的人力资源。
\end{itemize} \\
\hline

% 阶段 2
\textbf{2. 扩展与标准化} & 3年 (2027-2030) & 将模型推广至全校,并标准化流程与政策。 &
\begin{itemize}
    \item 颁布基于V-AQA的正式质量保障规定。
    \item 推广试点单位的最佳实践。
    \item 完善带有商业智能模块的质量保障管理信息系统。
    \item 按新模型执行首个内部评估周期。
\end{itemize} &
\begin{itemize}
    \item 100%的单位按新流程运作。
    \item 在定期管理会议中使用质量保障管理信息系统。
    \item 流程数字化的比例。
    \item 完成1个全校内部评估周期。
\end{itemize} &
\begin{itemize}
    \item 质量保障管理信息系统升级与维护成本。
    \item 跨院系改进项目的预算。
    \item 内部评估活动的成本。
\end{itemize} \\
\hline

% 阶段 3
\textbf{3. 优化与传播} & 2年 (2030-2032) & 质量保障体系高效运作,自我改进,学校成为质量典范。 &
\begin{itemize}
    \item 使用高级数据分析优化运作。
    \item 领导与其他学校的标杆比对网络。
    \item 向社区分享经验与V-AQA模型。
\end{itemize} &
\begin{itemize}
    \item 战略质量指标的改善程度。
    \item 参与标杆比对网络的学校数量。
    \item 发表的关于该模型的论文、报告数量。
\end{itemize} &
\begin{itemize}
    \item 用于数据分析的研发预算。
    \item 活动、研讨会组织成本。
    \item 系统可持续维护的资源。
\end{itemize} \\
\end{longtable}

\subsection{成功的先决条件}
\label{subsec:dieu_kien_thanh_cong}
提出实施路线图是必要的,但还不够。要使V-AQA模型从计划走向现实,必须具备以下先决条件。这些是不可协商的因素,决定了整个改革努力的成败。
\begin{enumerate}
    \item \textbf{来自最高层的政治承诺:} 这是生死攸关的因素。按照V-AQA模型的转型过程将触及权力结构、工作习惯以及许多单位和个人的利益。如果没有校董会和校领导班子的绝对、一致和持续的支持,所有改革努力都将轻易地被系统的惰性所抵消。
    
    \item \textbf{相应的资源投入:} 改革需要投资。学校需要有一个清晰的财务计划,为重要项目分配预算,特别是用于建设和维护质量保障管理信息系统的经费、用于培训和宣传活动的经费,以及一个足够吸引人的奖励基金,以激励质量文化的建设。
    
    \item \textbf{建立一个足够强大的核心团队:} 需要成立一个被充分授权的质量保障处/部门,拥有深谙现代质量管理、数据分析和项目管理的专家。这个团队将是“火车头”,是为领导层提供参谋并引导、支持其他单位实施过程的核心。
\end{enumerate}

\subsection{总体风险管理策略}
\label{subsec:quan_ly_rui_ro}
实施一个新模型最大的挑战不在于技术,而在于人和组织文化。下表总结了在实施V-AQA模型每个要素时可能遇到的主要风险,并提出了相应的缓解方案。科学地识别并制定管理这些风险的策略,将大大增加改革过程的成功可能性。

\begin{longtable}{|p{3cm}|p{5.5cm}|p{6cm}|}
\caption{实施V-AQA的风险分析与缓解方案}
\label{tab:risk_analysis}\\
\hline
\textbf{V-AQA要素} & \textbf{可能遇到的潜在风险} & \textbf{缓解方案/配套条件} \\
\hline
\endfirsthead
\multicolumn{3}{c}%
{{\bfseries \tablename\ \thetable{} -- 续前页}} \\
\hline
\textbf{V-AQA要素} & \textbf{可能遇到的潜在风险} & \textbf{缓解方案/配套条件} \\
\hline
\endhead
\hline \multicolumn{3}{r}{{续下页}} \\
\endfoot
\hline
\endlastfoot

% 第1行
\textbf{1. 领导与治理} & 
\begin{itemize}
    \item 领导承诺不实,仅形式上支持。
    \item 中层管理因被分权和要求更高责任而抵制。
\end{itemize} & 
\begin{itemize}
    \item 成立由校长直接担任组长的改革指导委员会。
    \item 各级领导的关键绩效指标必须与质量战略执行结果挂钩。
    \item 组织关于变革管理的强制性培训课程。
\end{itemize} \\
\hline

% 第2行
\textbf{2. 质量文化} & 
\begin{itemize}
    \item 教师、员工中根深蒂固的惰性和“不愿改变”的文化。
    \item 奖励政策吸引力不足,无法产生动力。
\end{itemize} & 
\begin{itemize}
    \item 持续、多渠道地宣传模型的重要性和益处。
    \item 从试点单位创造“速赢”,以建立信心。
    \item 奖励必须是实质性的(财务、晋升机会、公开表彰)。
\end{itemize} \\
\hline

% 第3行
\textbf{3. 利益相关者的参与} & 
\begin{itemize}
    \item 企业因缺乏时间和利益而形式化参与,无实质性贡献。
    \item 学生和校友漠不关心,看不到自己的角色。
\end{itemize} & 
\begin{itemize}
    \item 为行业咨询委员会制定明确的运作章程,包含具体的权利和责任。
    \item 在学术委员会中给予学生代表真正的投票权。
    \item 建立表彰和为积极的校友、企业创造利益的机制。
\end{itemize} \\
\hline

% 第4行
\textbf{4. 内部流程(包括质量保障管理信息系统)} & 
\begin{itemize}
    \item 构建质量保障管理信息系统成本过高。
    \item 数据不同步、不准确。
    \item 教师因必须应用新方法而负担过重。
\end{itemize} & 
\begin{itemize}
    \item 分阶段实施质量保障管理信息系统;考虑开源解决方案。
    \item 颁布数据治理规定。
    \item 为教师应用创新提供支持政策(减少标准课时、加绩效分)。
\end{itemize} \\
\hline

% 第5行
\textbf{5. 合作与协调} & 
\begin{itemize}
    \item 各院、处、室之间的“暗中竞争”和“各自为政”文化。
    \item 其他学校在标杆比对网络中不愿分享数据。
\end{itemize} & 
\begin{itemize}
    \item 设计专门仅用于跨单位项目的特别资助基金。
    \item 从敏感度较低的指标开始建立标杆比对网络,并由一个有信誉的第三方承诺保密。
\end{itemize} \\
\end{longtable}

% het goi 9


\section*{第四章结论:V-AQA模型及其应用潜力}
\addcontentsline{toc}{section}{第四章结论}

本章完成了本论文的核心任务:构建并详细论证了\textbf{混合与适应性质量保障模型(V-AQA)}。本章超越了提出零散建议的范畴,构建了一个整体的行动框架,其中各项解决方案基于坚实的理论基础和实践证据,逻辑地相互关联,旨在解决越南高等教育质量保障体系中固有的挑战与差距。

所呈现的V-AQA模型不仅是一个理论构想,更是一个具有高度应用性的建议,体现在以下核心特性:

\paragraph{全面性与系统性}
V-AQA模型以整体的方式处理质量问题,同步作用于五个相互作用的要素:从\textbf{领导与治理}的导向作用,\textbf{质量文化}的内在动力,\textbf{利益相关者}与社会的连接,“机器”运作的\textbf{内部流程},到\textbf{合作与协调}的协同力量。这种方法有助于避免零敲碎打的解决方案,确保变革在整个体系中可持续地发生。

\paragraph{混合性与适应性}
V-AQA模型的两大哲学基础直接解决了越南背景下的内在矛盾。
\begin{itemize}
    \item \textbf{混合性:} 该模型不否认对国家负责的必要性,但同时为来自内部的主动改进创造了空间和动力,旨在实现卓越和真正的竞争力。它协调了来自外部的合规压力和来自内部的改进需求。
    \item \textbf{适应性:} 认识到背景的不断变化,该模型不提出僵化的计划。取而代之的是,短期的改进周期和持续的反馈循环帮助各大学能够“边做边学”并及时调整其战略。
\end{itemize}

\paragraph{可行性与可衡量性}
本论文通过提出一份详细的“行动手册”证明了V-AQA模型的可行性,包括:
\begin{itemize}
    \item \textbf{针对每个要素的具体解决方案,} 从重构治理、建设文化、将校企联系机制化,到现代化流程和构建信息系统。
    \item \textbf{一个为期7年、分3个阶段的实施路线图,} 为每个阶段设定了明确的目标、行动和关键绩效指标,帮助管理者能够逐步应用该模型。
    \item \textbf{一个基于数据的管理系统(质量保障管理信息系统),} 通过示例仪表盘进行说明,显示该模型可以通过数字工具运作,增强决策的透明度和效率。
    \item \textbf{一个全面的风险分析框架,} 识别潜在挑战并提出缓解方案,大大增加了在实践中成功的可靠性和可能性。
\end{itemize}

\subsection*{应用潜力与推广条件}
V-AQA模型设计灵活,可以调整以适应越南不同类型的高校,从研究型大学到应用型大学。然而,要使该模型能够成功推广,需要具备三个先决条件:
\begin{enumerate}
    \item \textbf{来自最高层领导的强有力且一致的政治承诺,} 他们必须扮演“总建筑师”和整个改革过程的激励者角色。
    \item \textbf{相应的资源投入,} 特别是用于建设和维护质量保障管理信息系统的预算,以及用于培训、奖励以促进质量文化的经费。
    \item \textbf{一个系统的变革管理战略,} 具备毅力、对话能力和同理心,以克服组织文化中的惰性和障碍。
\end{enumerate}

总之,第四章实现了本论文的主要研究目标,即构建一个系统性、实证性且可行的改革模型。V-AQA模型有望成为一项重要贡献,为越南的政策制定者和高等教育管理者在充满挑战但也充满希望的征途上,提供一个新的视角和一套有效的工具,以可持续地提升高等教育质量,满足国家发展和国际一体化的要求。


% het goi 10 chuong 4




















	\chapter{结论与建议}
\label{chap:ket_luan_khuyen_nghi}

%-----------------------------------------------------------------------
% Mục 5.1: Tổng quan và Tóm tắt các Phát hiện Chính
%-----------------------------------------------------------------------

\section{主要研究成果总结与论文定位}
\label{sec:tong_hop_ket_qua}

本章作为总结与结论章节,汇集并凝练了本论文的全部研究成果。在经历了从理论基础分析(第二章)、"解剖"系统现状(第三章)到构建全面改革模型(第四章)的历程后,最后一章将承担三项核心任务:
\begin{enumerate}
    \item \textbf{总结}本论文的主要科学发现与贡献。
    \item \textbf{提出}一套具体可行的政策与行动建议,旨在完善越南高等教育质量保障体系。
    \item \textbf{坦诚地讨论}实施这些建议时面临的挑战、先决条件,同时指出研究的局限性及未来的潜在研究方向。
\end{enumerate}

本章的目标不仅是为研究工作画上句号,更是为政策制定者、教育管理者以及相关方提供一个系统的视角和一份有科学依据的"行动手册",共同致力于可持续地提升越南高等教育质量。

\subsection{现状分析的核心发现总结}
\label{subsec:tom_tat_phat_hien}

第三章详细呈现的对2015-2024年间越南高等教育质量保障体系现状的深入分析,得出了重要结论,勾勒出一幅光明与阴影交织的多维度画面。

\paragraph{论点一:规模上的成就与广泛的合规性。}
不可否认,整个体系在扩大规模和执行质量认证规定方面取得了显著的努力和成就。截至2023-2024学年末,越南高等教育体系已发展到\textbf{243所培养机构,学生规模超过235万}\footcite{thanhnien_quymo_2024}。特别是在质量保障方面,截至2025年初,已有\textbf{208所高等教育机构(占74.8\%)完成了认证周期},并被承认达到国内或国际质量标准\footcite{dantri_kiemdinh_2025}。这些数字显示了在意识和合规行动上的广泛积极转变,为更深层次的质量改进步骤奠定了重要的初步基础。

\paragraph{论点二:"质量悖论"与系统性"恶性循环"的存在。}
然而,在规模和认证比例这些亮眼数字的背后,是一个深刻的"发展悖论"。尽管毕业生在12个月内的就业率相当高,但调查显示,其中只有约\textbf{76\%的人从事与所学专业相符的工作}\footcite{neu_tylevieclam}。这种持续的"不匹配"是一个明确的证据,表明培养质量尚未真正满足劳动力市场的要求。

本论文证明,这一悖论并非偶然,而是具有系统性的"恶性循环"的后果。第三章的分析指出了核心症结,包括:
\begin{itemize}
    \item 一种\textbf{带有合规性和应付色彩的质量文化},这种文化被一种仍然带有浓厚行政色彩和自上而下强加的管理机制所"滋养"\footcite{vjol_reactiveculture}。
    \item 与外部利益相关者,特别是企业界的\textbf{联系松散且缺乏实质性},导致培养方案变得陈旧且缺乏应用性\footcite{worldbank_improvingperformance}。
    \item 最严重的是,一个\textbf{薄弱、碎片化的数据管理系统},被视为体系的"阿喀琉斯之踵",导致改进决策常常缺乏实证依据,未能达到预期效果。
\end{itemize}

清晰地识别这一悖论及其恶性循环,是迫切需要一个整体和系统的解决方案模型而非零散措施的实践基础。这也正是本章后续部分将重点解决的任务。

\subsection{基于V-AQA模型对核心挑战进行系统化梳理}
\label{subsec:he_thong_hoa_thach_thuc}

为获得一个系统的视角并为制定解决方案奠定基础,第三章已分析的挑战将按照V-AQA理论框架的5个要素进行总结。下表\ref{tab:tom_tat_thach_thuc_chuong5}提供了一份综合"诊断书",指出了越南高等教育质量保障体系中问题的"症状"和"根本原因"。

\begin{table}[h!]
\centering
\caption{基于V-AQA模型5要素的系统性挑战总结表}
\label{tab:tom_tat_thach_thuc_chuong5}
\resizebox{\textwidth}{!}{%
\begin{tabular}{|p{3cm}|p{6.5cm}|p{6cm}|}
\hline
\textbf{V-AQA要素} & \textbf{主要挑战/"症状"} & \textbf{根本性系统原因} \\
\hline
\textbf{1. 领导与治理} & 
推动自主的政策与仍带有浓厚合规色彩的管理实践之间的矛盾。领导层将更多精力用于应付行政要求而非质量战略治理\footcite{lypham_aosat_2024}。 & 
"主管部门"机制依然存在,削弱了校董会的实质性自主权\footcite{khuyen_bochuquan_2022}。自主的法律框架不完善且缺乏同步性。 \\
\hline
\textbf{2. 质量文化} & 
"反应型质量文化"占主导,质量保障活动仅为应付认证而季节性进行。教师被动参与,视质量保障为负担\footcite{vjol_reactiveculture}。 & 
自上而下管理模式的后果。缺乏鼓励和表彰实质性改进努力的机制。缺乏关于质量文化的系统性培训项目。 \\
\hline
\textbf{3. 利益相关者的参与} & 
与企业的联系松散,形式化。在培养方案设计中咨询利益相关者(企业、学生)的做法有限,导致"技能差距"\footcite{worldbank_improvingperformance}。 &
"象牙塔思维"的遗留物。缺乏互利共赢的双边合作机制。学生和校友在学术治理中的角色仍然模糊\footcite{pmc_article_9127449}。 \\
\hline
\textbf{4. 内部流程} & 
\begin{itemize}
    \item 培养方案开发流程落后,预期学习成果模糊,打破了"建设性对齐"原则\footcite{ijlter_elo_copy}。
    \item 物质设施和信息通信技术基础设施薄弱\footcite{worldbank_improvingperformance}。
    \item \textbf{"阿喀琉斯之踵":} 数据系统碎片化、不可靠,阻碍了基于证据的治理\footcite{aunsec_redesigningIQA}。
\end{itemize} &
重理论的课程开发思维。投资预算有限,采购流程复杂。在国家和学校层面缺乏一个集成的管理信息系统。 \\
\hline
\textbf{5. 合作与协调} & 
校内"孤岛"状态。各校之间的竞争环境压倒了合作与比对精神。学校、其他学校与社会之间缺乏战略合作模式。 & 
根深蒂固的"本位主义"、"部门主义"思维。缺乏推动实质性经验分享和相互学习的机制与平台。 \\
\hline
\end{tabular}
}
\end{table}

\vspace{0.5cm}

上述综合图景表明,越南质量保障体系的问题并非在于单一的某个方面,而是一系列具有因果关系、相互交织并相互强化的挑战集合。一种被动的质量文化是集权管理模式的后果,而它又反过来使得与企业建立联系或改进内部流程的努力变得更加困难。数据管理的薄弱又使得领导者缺乏战略决策的依据,恶性循环就这样持续下去。

清晰地识别这些系统性挑战及其相互关系,是最重要的科学基础,申明了任何想要成功的改革努力都必须采取一种整体性的方法。不能只专注于培训如何编写预期学习成果而忽略了与企业建立合作机制。同样,如果不能赋予学校实质性的自主权和现代化的管理工具,也不能要求学校进行创新。

基于对这些"瓶颈"的深刻理解,本论文将转到下一部分,申明其在理论和实践方面的贡献,然后提出一套旨在打破已指出的恶性循环的战略性解决方案体系。


% het goi 1 2


\section{本论文的新贡献}
\label{sec:dong_gop_luan_an}

基于实证分析结果,本论文在理论和实践两个层面都做出了具有意义的新贡献。本部分将重点阐明其在理论方面的贡献,申明所构建和验证的分析框架的科学价值与独特性。

\subsection{理论贡献}
\label{subsec:dong_gop_ly_luan}

本论文核心且贯穿始终的理论贡献在于成功构建并论证了一个\textbf{新的综合理论框架——混合与适应性质量保障模型(V-AQA)}。该贡献体现在三个主要方面。

\paragraph{首先,本论文指出了在复杂背景下应用单一理论的局限性。}
如第二章所分析和批判的,当经典的治理理论被独立地应用于像越南这样的转型期国家的高等教育领域时,都暴露了其局限性。
\begin{itemize}
    \item \textbf{新制度主义理论}很好地解释了同形现象以及为获得合法性而产生的合规压力\footcite{MeyerPowell2020},但未能充分解释大学自身的主动性、创新能力和对抗性策略。
    \item \textbf{利益相关者理论}揭示了利益冲突以及为多个群体创造价值的必要性\footcite{Freeman1984},但缺乏分析有形监督机制和塑造利益相关者优先级的权力结构的工具。
    \item \textbf{委托代理理论}为分析问责结构和控制机制提供了一套锐利的工具\footcite{JensenMeckling1976},但有将教育中不仅基于合同,还基于文化和无形规范的复杂关系简单化的风险。
\end{itemize}
通过明确指出这些"盲点",本论文有力地申明了采用一种综合的多理论方法以把握全局的必要性。

\paragraph{其次,本论文成功构建了一个多维度、系统的分析模型。}
V-AQA模型并非机械的拼接,而是将不同理论视角综合并结构化为一个连贯分析框架的努力。该模型的五个要素——(1)领导与治理、(2)质量文化、(3)利益相关者的参与、(4)内部流程,以及(5)合作与协调——是建立在三大基础理论的交汇和互补之上的。
\begin{itemize}
    \item \textit{领导与治理}和\textit{内部流程}要素深受委托代理理论视角的影响。
    \item \textit{质量文化}和\textit{合作与协调}要素通过新制度主义理论得到了清晰的映照。
    \item \textit{利益相关者的参与}要素是利益相关者理论的直接应用。
\end{itemize}
构建这样一个综合模型,提供了一个新的、更全面的分析工具,允许从权力结构、文化规则到技术流程等多个维度来"解剖"质量保障体系的问题。

\paragraph{第三,V-AQA模型在分析转型期国家高等教育体系方面具有特殊的理论价值。}
一个重要的理论贡献在于该模型对特殊背景的适用性。与在欧美国家(拥有悠久的大学自主传统和发达的公民社会)发展的质量保障模型不同,V-AQA模型旨在捕捉像越南这样体系的典型张力:即\textbf{一个强大的、集中的国家管理机制}与来自\textbf{市场和社会利益相关者日益增长的压力}并存\footcite{WB_TransitionalQA}。V-AQA模型,凭借其在问责制与改进之间的"混合"哲学,以及应对变化背景的"适应性"原则,为解释和指导这些处于转型阶段的体系提供了一个更敏锐的理论框架。因此,本论文不仅对越南有价值,也可以在类似的国家背景下得到参考和验证。

总之,通过超越单一理论的局限性并构建一个综合、系统且对背景敏感的分析框架,本论文为丰富高等教育治理和质量保障的理论宝库做出了贡献。

\subsection{实践贡献}
\label{subsec:dong_gop_thuc_tien}

除了理论价值,本论文最重要和最实际的贡献在于其应用能力,即为越南的管理者和政策制定者提供了一个具体的分析框架和行动路线图。

\paragraph{首先,本论文提供了一个系统性诊断工具,用以识别核心"瓶颈"。}
V-AQA模型及第三章的分析为管理者提供了一个系统性的视角来"审视"和"诊断"自己的组织,而不是仅仅描述表面症状。通过从5个相互作用的要素来分析问题,一所大学可以:
\begin{itemize}
    \item 准确确定质量问题的根本原因。例如,学校可以不仅仅得出"学生软技能薄弱"的结论,而是可以追溯到"利益相关者参与"薄弱(没有来自企业的反馈)或"内部流程"薄弱(预期学习成果无法衡量软技能)的根源。
    \item 识别正在抑制自身发展的"恶性循环",例如缺乏数据与决策效率低下之间的循环\footcite{aunsec_redesigningIQA}。
\end{itemize}
这个诊断工具有助于管理者从被动、零散地解决问题,转向主动和系统性的方法。

\paragraph{其次,本论文构建了一份全面且可行的"行动手册"。}
改革努力最大的挑战之一是政策与执行之间的差距\footcite{OECD_PolicyToAction}。本论文通过不仅仅停留在分析,还在第四章中构建了一套解决方案和一个详细的实施路线图,努力缩小这一差距。
\begin{itemize}
    \item \textbf{行动具体化:} 为V-AQA的每个要素提出的解决方案都非常具体,从建立行业咨询委员会、部署质量保障管理信息系统,到为跨学科项目设立资助基金。
    \item \textbf{优先级排序:} 为期7年、分3个阶段的实施路线图(基础与试点、扩展与标准化、优化与传播)提供了一条清晰的路径,帮助学校知道从何处着手以及接下来的步骤。
    \item \textbf{风险管理:} 对每组解决方案的潜在风险进行分析并提出缓解措施,大大增加了模型应用的现实性和成功可能性。
\end{itemize}
这份手册对于各大学的校领导班子、质量保障处负责人以及政策制定者在构建质量改革战略和行动计划的过程中尤其有用。

\paragraph{第三,本论文提供了一个协调的视角,帮助越南各校在一体化背景下定位。}
本论文深入分析了由国家主导的质量保障模型(中国经验)和基于网络、同行合作的模型(东盟大学网络质量保障框架)之间的差异。在此基础上,提出的V-AQA模型如同一条"中间道路",一种有选择的综合,帮助越南大学:
\begin{itemize}
    \item 既能满足一个由国家管理体系的问责要求。
    \item 又能建立内在能力和一种主动的质量文化,以趋近国际标准。
\end{itemize}
这种方法帮助管理者避免了两个极端:要么不适当地机械照搬国际标准,要么封闭、自满于国内的最低规定。

总之,本论文不仅是一部供阅读的著作,更是一套供"使用"的工具。它为"诊断"问题提供了实证实据,为"思考"解决方案提供了理论框架,为"执行"改革提供了行动计划。这正是本论文所追求的最重要的实践贡献。

% het goi 3 4


\section{旨在完善越南高等教育质量保障体系的建议体系}
\label{sec:he_thong_khuyen_nghi}

基于现状分析结果和已确认的科学贡献,本论文提出了一套政策建议和战略行动体系。这些建议并非孤立的解决方案,而是经过精心构建,旨在\textbf{打破第三章已识别的"恶性循环"},并创造一个可持续质量改进的\textbf{"良性循环"}。

该建议体系基于V-AQA模型的核心原则构建:
\begin{itemize}
    \item \textbf{系统性:} 同步作用于模型的全部5个要素,从宏观层面(国家政策)到微观层面(各学校的活动)。
    \item \textbf{可行性:} 建议按优先级划分,并结合越南的实际情况。
    \item \textbf{可衡量性:} 每组行动都提出了具体的关键绩效指标,以便进行跟踪和评估。
\end{itemize}

下表\ref{tab:tong_quan_khuyen_nghi}提供了整个建议体系的概览,按主要对象群体、具体行动、优先级及与V-AQA模型要素的关联进行分类。该表将作为后续详细分析部分的指南。

\begin{longtable}{|p{3.5cm}|p{5.5cm}|p{2cm}|p{3.5cm}|}
\caption{基于V-AQA模型的建议体系概览表}
\label{tab:tong_quan_khuyen_nghi} \\
\hline
\textbf{对象} & \textbf{建议的战略行动} & \textbf{优先级} & \textbf{主要作用的V-AQA要素} \\
\hline
\endfirsthead
\multicolumn{4}{c}%
{{\bfseries \tablename\ \thetable{} -- 续前页}} \\
\hline
\textbf{对象} & \textbf{建议的战略行动} & \textbf{优先级} & \textbf{主要作用的V-AQA要素} \\
\hline
\endhead
\hline \multicolumn{4}{r}{{续下页}} \\
\endfoot
\hline
\endlastfoot

% 第1行:国家管理机构
\textbf{1. 国家管理机构(教育培训部)} & 
1.1. 将角色从微观控制转变为环境创造和宏观调控。 \newline
1.2. 推动基于绩效的拨款机制。 \newline
1.3. 促进一个多样化和竞争性的质量保障生态系统,承认国际质量认证组织。 \newline
1.4. 战略性投资建设集成的国家高等教育数据系统。 & 
短期 (1.1, 1.4) \newline 中期 (1.2, 1.3) & 
1. 领导与治理 \newline
5. 合作与协调 \\
\hline

% 第2行:高等教育机构
\textbf{2. 高等教育机构} & 
2.1. 实质性地制定并执行校级质量战略。 \newline
2.2. 发起从基层建设质量文化的运动。 \newline
2.3. 建立并机制化行业咨询委员会的运作。 \newline
2.4. 按照成果导向教育实现培养方案开发流程现代化并多样化评估方法。 \newline
2.5. 投资建设质量保障管理信息系统。 \newline
2.6. 主动参与标杆比对和同行评审网络。 & 
短期 (2.1, 2.2) \newline 中期 (2.3, 2.4, 2.5) \newline 长期 (2.6) & 
同步作用于V-AQA模型的全部5个要素。 \\
\hline

% 第3行:质量认证中心
\textbf{3. 质量认证中心} & 
3.1. 提升认证员队伍的专业能力和职业素养。 \newline
3.2. 为不同类型的学校开发专门的评估标准或指南。 \newline
3.3. 加强结果和综合分析报告的公开透明化。 & 
中期 & 
4. 内部流程 \newline
5. 合作与协调 \\
\hline

% 第4行:各协会
\textbf{4. 企业与行业协会} & 
4.1. 主导构建"职业能力框架"。 \newline
4.2. 主动参与行业咨询委员会及培养方案评估活动。 & 
中期 & 
3. 利益相关者的参与 \\
\hline

\end{longtable}

\vspace{0.5cm}

如上表所示,各项建议并非仅集中于单一主体,而是要求从国家管理机构、各大学到认证组织和企业界等整个教育生态系统的共同努力和同步行动。每组行动都旨在解决已分析的相应"瓶颈",最终目标是在质量的思维和行动上实现根本性转变。

本章的后续部分将深入分析和论证每一组建议,包括其科学依据、需要开展的具体行动以及配套的关键绩效指标。

\subsection{对国家管理机构的建议}
\label{subsec:khuyen_nghi_qlnn_detailed}

作为体系的"总建筑师",国家管理机构,直接是教育培训部,在为质量改革创造有利的制度环境方面扮演着决定性角色。以下建议侧重于将国家角色从直接控制转变为环境创造和宏观调控,这一趋势已在许多先进教育体系中被证明有效\footcite{OECD_PolicyToAction}。

\subsubsection{建议1.1:将角色从"微观控制者"转变为"环境创造者和宏观调控者"}

\paragraph{论据:}
第三章的分析指出了一个核心矛盾:虽然宏观政策朝向自主,但具体的管理机制仍然带有浓厚的行政色彩,催生了"合规文化",并削弱了学校实质性改进的动力\footcite{pham2021governance}。教育培训部颁布的包含过多流程性、程序性标准的评估标准(例如,根据第12/2017号通知的25项标准\footcite{tt12_2017_bgddt}),无意中引导了各学校专注于应付性地完善档案,而非进行战略性改进。

为使\textbf{2018年修订的《高等教育法》}\footcite{luatvn_gddh_2018}精神真正落到实处,国家管理机构需要转变其角色。国家不应是"手把手指导者",而应是建立公平"游戏规则"者,监督最低标准的遵守情况,最重要的是,推动一个透明的环境,以便社会能够发挥其监督作用。这是从委托代理理论(委托人严密监督代理人)向更现代的治理模式的转变,即国家本着新制度主义精神,创造一个健康的"制度场域"。

\paragraph{具体行动:}
\begin{enumerate}
    \item \textbf{改革质量认证标准:} 研究修订教育机构和培养项目的质量认证标准,使其更精简,侧重于产出和核心质量保障条件(例如:师资能力、基础设施、有效治理),减少带有浓厚行政流程、程序色彩的标准。新标准需要有"开放空间",让各校可以根据自身使命和独特战略,以不同方式证明其质量。
    
    \item \textbf{加强"认证认证机构"的角色:} 教育培训部需要专注于建立严格监督各质量认证中心活动运作的标准和流程,确保这些组织的独立性、专业性和道德性。赋予认证中心更大的权力和更高的问责要求,逐步减少教育部对具体评估活动的直接干预。
    
    \item \textbf{建立国家高等教育质量信息门户:} 建立一个单一的电子信息门户,在此\textbf{完全}公开所有高等教育机构的自评报告、外部评估报告以及主要质量指标(例如:就业率、规模、师资队伍)。这种透明度是最有效的社会监督工具,能创造健康的竞争压力,并迫使各校对其质量真正负责。
\end{enumerate}

\paragraph{关键绩效指标:}
\begin{itemize}
    \item 新标准体系中侧重于产出的标准与侧重于投入/流程的标准的比例。
    \item 各大学和认证中心对新政策的清晰度和支持性的满意度。
    \item 国家质量信息门户的数据访问和利用次数。
\end{itemize}

\subsubsection{建议1.2:推动基于绩效的拨款机制}

\paragraph{论据:}
单纯的行政压力通常只能带来合规。要创造实质性的改进动力,需要有财政激励工具。已在许多经合组织国家成功应用的基于绩效的拨款机制,是一个强大的政策工具,能将各校的重点从维持运作转向追求卓越\footcite{oecd_pbf_2021}。通过将部分预算与产出结果挂钩,国家将发出一个明确的信息:"质量将得到奖赏"。这将促使各校领导层更具战略性地思考如何改善核心质量指标。

\paragraph{具体行动:}
\begin{enumerate}
    \item \textbf{构建国家基于绩效的拨款指标体系:} 教育培训部主导,与专家、各大学及相关方协调,构建一套清晰、透明且可衡量的关键绩效指标体系,作为预算分配的依据。该指标体系可包括以下几组:
        \begin{itemize}
            \item \textit{培养质量:} 毕业生12个月内从事对口工作的比例(数据来自高等教育管理信息系统与社会保险系统联动),学生满意度。
            \item \textit{研究能力:} 每位教师在权威期刊(Scopus/WoS)上的国际发表数量,来自科技活动和知识转移的收入。
            \item \textit{国际化水平:} 国际学生和教师比例,由权威国际组织认证的项目数量。
        \end{itemize}
    \item \textbf{按路线图进行试点:} 开始在已实现自主的公立大学群体中试点实施基于绩效的拨款机制。在初期阶段,可以根据关键绩效指标的达成结果分配一部分预算(例如:总经常性支出的10-20%)。
    \item \textbf{评估与调整:} 每2-3年周期后,需要有一份独立的影��评估报告,分析基于绩效的拨款机制的积极和消极影响(如有),从而在考虑推广前,对关键绩效指标体系和分配方式进行适当调整。
\end{enumerate}

\paragraph{关键绩效指标:}
\begin{itemize}
    \item 按基于绩效的拨款机制分配的高等教育预算百分比。
    \item 试点院校群体的关键绩效指标平均改善程度与非试点院校群体的比较。
    \item 各大学对基于绩效的拨款机制的共识度和支持度。
\end{itemize}


% het goi 5 6


\subsubsection{建议1.3:促进一个多样化和竞争性的质量保障生态系统}

\paragraph{论据:}
一个质量保障体系只有在评估主体多样化且他们之间存在良性竞争时,才能真正有效。过度依赖少数几家仍受教育培训部管理的国内认证中心,可能导致缺乏客观性且不鼓励创新\footcite{giaoducnet_kdcl_list_2023}。相反,开放并承认有信誉的国际认证组织将带来诸多好处:
\begin{itemize}
    \item \textbf{增强客观性与国际接轨:} 国际认证组织按全球标准运作,帮助越南大学客观地看待自身质量,并加速国际化进程。实际上,已有\textbf{11所越南大学获得了像德国工商管理认证基金会、英国质量保障署、法国高等教育与研究评估高等委员会等国外权威组织的认证},这表明顶尖大学确有接轨的需求和能力\footcite{thuvienphapluat_11truong_quocte}。
    \item \textbf{创造积极的竞争压力:} 国际组织的存在将产生压力,迫使国内认证中心不断提升其专业能力和服务质量以求竞争。
    \item \textbf{推动按专业领域的认证:} 许多国际组织专门从事特定领域的认证(例如:工程领域的ABET,商科领域的AACSB)。参与专业认证将有助于各院系实质性地提升培养质量,并使其与各行业的要求相符。
\end{itemize}

\paragraph{具体行动:}
\begin{enumerate}
    \item \textbf{颁布承认国际认证结果的规定:} 建立并颁布一个通畅透明的法律走廊,用于承认有信誉的国际组织的认证结果,特别是那些全球网络如\textbf{国际高等教育质量保障机构网络}或区域网络如\textbf{亚太质量网络}的成员组织。
    \item \textbf{鼓励专业认证:} 为优先领域(技术、工程、经济、健康)的培养项目参与并获得世界公认的专业组织认证提供具体鼓励政策(例如:资助部分经费,在评优项目中加分)。
    \item \textbf{推动行业协会的角色:} 创造机制,让国内的行业协会(例如:建筑工程师协会、越南会计与审计协会)参与构建职业能力标准,并逐步参与相关培养项目的评估和承认。
\end{enumerate}

\paragraph{关键绩效指标:}
\begin{itemize}
    \item 获得有信誉的国际组织认证的培养项目数量。
    \item 成为国际网络(国际高等教育质量保障机构网络、亚太质量网络)成员的国内认证中心比例。
    \item 由行业协会颁布并被各校用作参考的职业能力标准数量。
\end{itemize}

\subsubsection{建议1.4:战略性投资建设集成的国家高等教育数据系统}

\paragraph{论据:}
正如多次分析和强调的,数据的薄弱和碎片化是越南高等教育治理体系的"阿喀琉斯之踵"\footcite{aunsec_redesigningIQA}。缺乏同步和可靠的数据,所有宏观管理努力,从政策规划、按绩效拨款到人力需求预测,都变得缺乏科学依据。颁布关于教育行业统计报告制度的\textbf{第25/2024号通知}是朝着正确方向迈出的一步,但需要一个足够强大的技术系统来实现\footcite{luatvietnam_tt25_2024}。

因此,投资建设一个国家高等教育管理信息系统不应被视为一项开支,而应是对\textbf{治理基础设施的一项战略投资},有可能为整个体系带来巨大的效率和透明度回报。

\paragraph{具体行动:}
\begin{enumerate}
    \item \textbf{成立国家高等教育管理信息系统指导委员会:} 教育培训部应主导,与相关部委(计划与投资部、财政部、信息与通信部、越南社会保险)协调,制定一个总体方案并为此项目分配国家资源。
    
    \item \textbf{设计现代化的系统架构:} 国家高等教育管理信息系统需被设计为一个国家\textbf{数据仓库},能够按照一个共同标准整合来自各校信息系统的数据。核心功能包括:收集、清洗、存储、分析和数据可视化。
    
    \item \textbf{部署突破性构件——与社会保险数据联通:} 这是一个旨在解决衡量产出质量难题的革命性方案。建立机制(有严格的个人信息保密规定),将毕业生数据(根据个人身份识别码)与越南社会保险的数据库进行比对。这将提供关于以下方面的\textbf{客观、准确和实时更新}的图景:
        \begin{itemize}
            \item 学生的实际就业率。
            \item 按学校、专业的平均收入水平。
            \item 获得第一份工作的平均时间。
            \item 职业流动和从事与专业不符工作的比例。
        \end{itemize}
    这些数据将是基于绩效的拨款机制最重要的输入,并帮助社会对每个培养项目的真实"价值"有一个透明的了解。
    
    \item \textbf{建立开放数据(Open Data)利用政策:} 在匿名化处理后,一部分高等教育管理信息系统的数据应以开放数据的形式公布,以便研究人员、独立组织和社会可以利用,进行深入分析,有助于推动一个透明和基于证据的治理。
\end{enumerate}

\paragraph{关键绩效指标:}
\begin{itemize}
    \item 已连接并与国家高等教育管理信息系统同步数据的高等教育机构百分比。
    \item 汇总和公布全行业统计报告的平均时间(从每年减少到每季度/每月)。
    \item 基于国家高等教育管理信息系统数据构建的政策分析报告数量。
\end{itemize}

\subsection{对高等教育机构的建议}
\label{subsec:khuyen_nghi_csgddh_detailed}

虽然国家的宏观政策创造了环境和法律走廊,但质量的实质性转变必须来自每所大学自身的内部努力。以下建议按照V-AQA模型的五个要素构建,旨在为教育管理者提供一个具体的行动框架,以从内部引领变革。

\subsubsection{建议2.1:重构领导与治理——从"控制者"转变为"环境创造者"}

\paragraph{论据:}
首要需要解开的症结是领导团队的角色。如第三章所分析,当领导者陷入行政管理和应付合规要求时,他们将没有精力和远见来执行其战略角色\footcite{lypham_aosat_2024}。为打破此恶性循环,高层领导(校董会、校领导班子)的角色需要重新定位。他们不是直接从事质量工作的人,而是为质量能够在各级生根发芽和发展创造一个制度、文化和资源环境的人,这符合"创业型大学"的精神\footcite{clark_1998}。

\paragraph{具体行动:}
\begin{enumerate}
    \item \textbf{制定并承诺执行校级质量战略:} 领导层必须主导制定一份正式的五年期\textbf{质量战略},并将其融入学校的总体发展战略中。该文件不应束之高阁,而应是所有行动的指南,其中必须明确定义:质量愿景、优先目标、具体的关键绩效指标以及相应的资源分配计划。
    \item \textbf{强力分权并辅以问责制:} 在学术事务(改进培养方案、创新教学方法)上给予院/系更实质性的自主权。同时,建立一套清晰的关键绩效指标体系来衡量各单位的绩效。领导层将通过质量保障管理信息系统执行战略监督角色,而不是对日常运作进行微观干预。
    \item \textbf{投资于中层管理团队的治理能力:} 为院/系主任、副主任、处室负责人组织关于现代大学治理、变革管理和数据驱动决策的强制性培训项目。这是对"继任团队"的投资,以确保改革的可持续性。
\end{enumerate}

\paragraph{关键绩效指标:}
\begin{itemize}
    \item 每年在质量战略中设定的目标的完成率。
    \item 院/系领导对所获自主权程度的满意度调查得分。
    \item 100%的中层管理干部在两年内完成现代治理培训课程。
\end{itemize}

\subsubsection{建议2.2:塑造质量文化——从"应付"到"主动"}

\paragraph{论据:}
质量文化是无形资产,但决定了所有质量保障体系的成败。一个拥有最佳流程的体系,如果由抱着应付、被动心态的人来运作,也终将失败\footcite{iosr_passiveparticipation}。因此,从"合规文化"向"改进文化"的转型是一项战略性任务,需要采取既作用于组织中每个成员的认知又作用于其行动的解决方案\footcite{HarveyStensaker2008}。

\paragraph{具体行动:}
\begin{enumerate}
    \item \textbf{发起宣传和意识提升运动:} 组织如年度"质量周"、 "教学创新创举"竞赛、关于质量问题的开放论坛等活动。目标是让所有成员明白,质量不仅是质量保障处的责任。
    \item \textbf{建立表彰和奖励改进创举的体系:} 设立年度奖项("教学创新年度教师奖"、"高效质量保障模式院系奖")。更重要的是,将对质量改进的贡献标准纳入干部、教师的评估、评级和晋升流程中。
    \item \textbf{为"质量改进小组"赋能:} 在系级成立由自愿教师组成的小组(质量圈),任务是定期讨论并提出改进教学质量的方案。学校需要提供小额预算("创举支持基金"),以便这些小组可以试验新想法。
\end{enumerate}

\paragraph{关键绩效指标:}
\begin{itemize}
    \item 全校范围内关于质量文化认知的定期调查得分(逐年提高)。
    \item 每年提出并获资助实施的改进创举数量。
    \item 自愿参与质量改进活动的教职工比例。
\end{itemize}


% het 7 8 chuong 5


\subsubsection{建议2.3:将利益相关者从"咨询对象"转变为"战略伙伴"}

\paragraph{论据:}
"技能差距"\footcite{britishcouncil_skills_gap_2021}最深层的原因之一,是在学术流程中缺乏来自利益相关者,特别是雇主和校友的实质性声音。咨询活动(如果有的话)通常仅停留在形式化和被动的层面,如单向咨询或影响甚微的调查\footcite{vnujs_er_3848}。为打破"象牙塔壁垒",学校需要主动创造机制,使利益相关者成为共同创造价值的伙伴,这符合阿恩斯坦(1969)"参与阶梯"最高阶梯的精神\footcite{Arnstein1969}。

\paragraph{具体行动:}
\begin{enumerate}
    \item \textbf{将行业咨询委员会机制化并赋权:}
    学校需要为院系/专业层面成立行业咨询委员会并颁布正式的运作章程,而不是依赖个人关系。行业咨询委员会的角色不仅是咨询,还必须被赋予具体权力:
        \begin{itemize}
            \item \textbf{强制性审定}培养方案的预期学习成果。
            \item \textbf{定期参与}(每年至少1-2次)培养方案的审查和更新会议。
            \item \textbf{共同指导}和评估学生的毕业设计项目。
        \end{itemize}
    该模型是东盟大学网络质量保障\footcite{aunqa_guidelines_v4}和ABET\footcite{abet_criteria}等国际认证标准的核心要求,将创造一个可持续的对话与合作渠道,确保培养方案始终与实践相符。

    \item \textbf{提升学生和校友的角色:}
        \begin{itemize}
            \item \textbf{在学术委员会中为学生赋权:} 让(通过民主选举产生的)学生代表成为院/校级科学与培养委员会中拥有\textbf{投票权}的正式成员。
            \item \textbf{建立主动的"校友大使"网络:} 建立一个由事业有成且热心的校友组成的体系,其作用是定期提供关于培养方案契合度的反馈,并参与为低年级学生提供职业导向活动。
        \end{itemize}
\end{enumerate}

\paragraph{关键绩效指标:}
\begin{itemize}
    \item 来自行业咨询委员会的建议被记录并整合到培养方案改进计划中的百分比。
    \item 有学生代表参与并投票的学术决策数量。
    \item 企业对毕业生工作胜任度的满意度得分(逐年提高)。
\end{itemize}

\subsubsection{建议2.6:通过合作与比对打破"筒仓思维"}

\paragraph{论据:}
在一个孤立的环境中,质量无法得到可持续的改进。各处、室、院系之间的"孤岛"状态以及各校之间不健康的竞争,削弱了整个体系的协同力量和学习能力。要成为一个真正的"学习型组织"\footcite{Senge2006},各校需要主动创造在内部、校际和国际三个层面上的合作机制。

\paragraph{具体行动:}
\begin{enumerate}
    \item \textbf{促进内部合作:} 设立一个\textbf{专门资助跨单位质量改进项目的基金}。优先为有多个院、处、室参与的项目提供经费(例如:"提升软技能"项目需要学生工作处、各专业院系和质量保障处的协调)。
    
    \item \textbf{建立校际标杆比对网络:} 同一专业领域或地区的学校需要主动成立\textbf{质量标杆比对俱乐部}\footcite{jackson_lund_2000}。这些俱乐部可以:
        \begin{itemize}
            \item 通过一个中立的第三方,统一并分享(已匿名的)关于质量指标的数据集。
            \item 组织定期的同行评审活动和"最佳实践"分享研讨会。
        \end{itemize}
    该机制将把竞争转化为学习的动力,帮助各校客观地认识自己的优势和劣势。

    \item \textbf{战略性地利用国际合作活动:} 参与国际认证不应仅停留在获得证书的目标上。学校需要将一个\textbf{"认证后"流程制度化},其中质量保障处必须制定一个具体的行动计划来落实专家组的建议,跟踪进度并向校领导班子报告。
\end{enumerate}

\paragraph{关键绩效指标:}
\begin{itemize}
    \item 每年成功实施的跨院系/处室改进项目数量。
    \item 组织的标杆比对活动数量以及基于比对结果实施的改进措施数量。
    \item 按计划完成的国际认证建议的百分比。
\end{itemize}

\subsubsection{建议2.4与2.5:流程现代化与数据基础设施建设}

\paragraph{论据:}
第三章的分析已指出,内部流程,特别是课程开发流程和数据管理系统,正是越南质量保障体系的"阿喀琉斯之踵"。一个落后、脱节的培养方案开发流程将产生不合格的产品\footcite{ijlter_elo_copy}。同样,一个薄弱的数据系统将瘫痪基于证据的决策能力,使所有改进努力都变得盲目且低效\footcite{aunsec_redesigningIQA}。因此,实现这些流程的现代化并构建一个坚实的技术基础设施是强制性任务,是整个V-AQA模型能够运作的"硬件"基础。

\paragraph{具体行动1:按照成果导向教育理念标准化培养方案开发流程。}
V-AQA模型建议各大学全面转向基于\textbf{成果导向教育}理念的培养方案开发方法。这种方法有助于确保培养目标、教学活动和评估方法之间的紧密联系("建设性对齐"原则\footcite{biggs_constructive_alignment})。为避免形式化应用,该流程必须包括五个紧密相连的步骤:
\begin{enumerate}
    \item \textbf{利益相关者需求分析:} 始于收集和分析来自劳动力市场、行业专家和校友的要求。
    \item \textbf{预期学习成果的制定与审定:} 专业院系制定清晰、可衡量的预期学习成果,且必须由\textbf{行业咨询委员会强制性审定}。
    \item \textbf{设计课程地图:} 构建矩阵,清晰展示每门课程如何为实现预期学习成果做出贡献。
    \item \textbf{审定与批准:} 完整的培养方案必须由科学与培养委员会(包括外部专家)批准。
    \item \textbf{定期审查与改进:} 建立正式的培养方案审查周期(2-3年一次),基于利益相关者的反馈和实际学习成果。
\end{enumerate}

\paragraph{具体行动2:构建集成的质量保障管理信息系统。}
为打破数据恶性循环,V-AQA模型提出了一个基础性的技术解决方案:构建一个集成的\textbf{质量保障管理信息系统}。这是学校的"神经系统",任务是将来自各分散来源(教学、人事、调查、财务、科研)的数据整合到一个单一的\textbf{数据仓库}中。
质量保障管理信息系统的最终目标是将原始数据转化为对各级决策有用的信息。为实现此目标,系统需配备一个\textbf{商业智能}模块,以自动生成可视化的报告和仪表盘,并为不同用户角色进行定制和授权\footcite{educause_bi_2022}。例如:
\begin{itemize}
    \item \textbf{校领导班子仪表盘:} 显示校级战略关键绩效指标,如就业率、国际发表数量、雇主满意度(如表\ref{tab:dashboard_hieu_truong})。
    \item \textbf{院长仪表盘:} 显示院级运营关键绩效指标,如按时毕业率、学生对培养方案的满意度、院级科研课题数量(如表\ref{tab:dashboard_truong_khoa})。
    \item \textbf{教师仪表盘:} 提供关于教学活动的反馈、所负责班级学生的学习成果以及个人任务的执行进度(如表\ref{tab:dashboard_giang_vien})。
\end{itemize}
部署一个质量保障管理信息系统不仅是一个技术解决方案,更是一场工作方式的改革,需要领导层的坚定承诺和对用户的系统培训计划。

\paragraph{关键绩效指标:}
\begin{itemize}
    \item 按照五步成果导向教育流程构建和审查的培养方案百分比。
    \item (院级及以上)发布的、引用或基于质量保障管理信息系统数据的管理决策比例。
    \item 各级管理人员和教师对质量保障管理信息系统的实用性和便利性的满意度。
\end{itemize}


% het goi 9 10 chuong 5



\section{讨论:挑战、条件与前景}
\label{sec:ban_luan_trien_vong}

提出一个全面的改革模型是必要的,但在实践中成功实施该模型则是一个更大的挑战。本部分将坦诚地讨论应用V-AQA模型时最大的困难和障碍,其中,大学自主问题被视为基础性且具有决定性的因素。

\subsection{大学自主:一个有效质量保障体系的先决条件}
\label{subsec:banluan_tuchu}

纵观本论文的分析和建议,大学自主的角色始终被强调为一个基础性因素。可以肯定地说,\textbf{大学自主并非一个选项,而是V-AQA模型及质量改进方案能够成功和可持续实施的先决条件}。一个质量保障体系只有在大学被赋予权力和创新空间的环境中,才能最大限度地发挥其效用\footcite{eua_autonomy_qa}。

这种关系是辩证的。如果没有在学术、人事组织和财务方面的实质性自主权,领导者和教师将没有足够的动力和工具来按照V-AQA模型进行改革。反之,成功实施V-AQA模型,凭借其数据的透明度和明确的问责制,正是一所大学证明自己值得被赋予更大自主权的最好方式。

因此,实施V-AQA不能脱离执行\textbf{2018年修订的《高等教育法》}的路线图\footcite{luatvn_gddh_2018}。该模型的成功程度将直接取决于学校被赋予的自主程度。下面的矩阵图(图\ref{fig:matrix_vqa_autonomy})分析了V-AQA各要素在三种不同自主情景下的运作情况。

\begin{figure}[h!]
    \centering
    \caption{V-AQA模型按大学自主程度的运作矩阵}
    \label{fig:matrix_vqa_autonomy}
    \resizebox{\textwidth}{!}{%
    \begin{tabular}{|p{3cm}|p{4.5cm}|p{4.5cm}|p{4.5cm}|}
    \hline
    \textbf{V-AQA要素} & \textbf{低自主情景} \newline \textit{(合规环境)} & \textbf{中等自主情景} \newline \textit{(转型环境)} & \textbf{高自主情景} \newline \textit{(改进环境)} \\
    \hline
    \textbf{1. 领导与治理} & 领导主要执行行政命令,专注于满足主管部门的报告要求。 & 开始有战略决策的空间,但仍受行政规定约束。领导角色具有"混合"性质。 & 领导真正扮演战略创造者的角色。校董会在导向和监督方面拥有实质性权力。 \\
    \hline
    \textbf{2. 质量文化} & "反应型"和应付式文化。质量保障被视为负担,仅在有认证要求时才进行。 & 合规文化与改进文化之间存在拉锯。试点单位的"速赢"对于创造动力至关重要。 & 改进文化深入所有活动。质量是每个成员的内在责任。创新精神受到鼓励。 \\
    \hline
    \textbf{3. 利益相关者的参与} & 联系是形式化的,主要为完善证明档案。行业咨询委员会(如有)仅具装饰性。 & 行业咨询委员会开始被更系统地建立。在咨询中开始倾听学生的声音。 & 利益相关者(企业、学生、校友)成为"战略伙伴",共同创造课程和学校的价值。 \\
    \hline
    \textbf{4. 内部流程} & 流程僵化,按国家统一规定标准化。质量保障管理信息系统仅是用于导出报告的工具。 & 少数先锋院/系获准试点新流程(成果导向教育、基于项目的学习)。质量保障管理信息系统开始用于内部分析。 & 流程灵活,具有高度适应性。质量保障管理信息系统成为"神经系统",是各级基于证据决策不可或缺的工具。 \\
    \hline
    \textbf{5. 合作与协调} & 合作受限,通常仅按上级指示进行。"暗中竞争"和"保护领地"的思维普遍存在。 & 与企业和校际的合作活动开始在某些领域被主动推动,但尚未成体系。 & 学校主动领导标杆比对网络,与企业和国际伙伴进行战略性、可持续的研发合作。 \\
    \hline
    \end{tabular}
    }
\end{figure}

该矩阵显示,推动大学自主不仅是一项孤立的政策,更是一个能够"解锁"V-AQA模型全部五个要素潜力的杠杆。缺少这个杠杆,再好的改革建议也难以发挥全部作用。因此,完善法律走廊并扫除障碍,使各校能够充分且负责任地行使自主权,是国家在下一阶段的核心任务,为实质性地提升高等教育质量创造前提。

\subsection{风险分析与成功条件}
\label{subsec:risk_analysis_conditions}

实施像V-AQA这样全面和系统性的改革模型,必然会面临诸多风险和挑战。识别、分析并制定应对这些风险的策略,是确保改革过程成功的先决条件\footcite{kerzner_pm_2017}。本部分将重点关注三大风险群:(1)资源风险,(2)人与文化风险,以及(3)政策环境风险。

\subsubsection{风险分析与应对矩阵}
为获得一个全面和系统的视角,矩阵\ref{tab:risk_matrix}将总结主要风险,评估其影响程度和发生可能性,并提出相应的缓解措施。

% --- 已修改代码开始 ---
\begin{filecontents*}{risk_matrix.tex}
\begin{longtable}{|p{2.8cm}|>{\raggedright\arraybackslash}X|p{1.7cm}|p{1.7cm}|>{\raggedright\arraybackslash}X|}
\caption{实施V-AQA模型的风险分析与应对矩阵}
\label{tab:risk_matrix}\\
\hline
\textbf{风险类别} & \textbf{具体风险描述} & \textbf{发生可能性} & \textbf{影响程度} & \textbf{缓解/应对措施} \\
\hline
\endfirsthead
\multicolumn{5}{c}%
{{\bfseries \tablename\ \thetable{} -- 续前页}} \\
\hline
\textbf{风险类别} & \textbf{具体风险描述} & \textbf{发生可能性} & \textbf{影响程度} & \textbf{缓解/应对措施} \\
\hline
\endhead
\hline \multicolumn{5}{r}{{\textit{续下页}}} \\
\endfoot
\hline
\endlastfoot

% 风险1:资源
\textbf{1. 资源(财务、物质设施)} & 
缺乏预算投资于战略性项目,如构建质量保障管理信息系统、升级物质设施以及培训、奖励项目。 & 
高 & 
高 & 
- 制定详细的财务计划,分阶段实施。 \newline
- 多样化收入来源,推动公私合作伙伴关系以动员企业资源。 \newline
- 优先考虑开源技术解决方案以降低许可成本。 \\
\hline

% 风险2:人与文化
\textbf{2. 人与文化} & 
\begin{itemize}
    \item 干部、教师队伍的惰性、不愿改变的心态和怀疑态度。
    \item 感到利益受损的中层管理(暗中或公开)的抵制。
    \item 缺乏具备数据分析和现代治理能力的人力资源。
\end{itemize} & 
非常高 & 
非常高 & 
- 最高层领导的坚定政治承诺和模范作用。 \newline
- 持续、透明地宣传改革的益处。专注于创造"速赢"\footcite{kotter_1996}。 \newline
- 组织关于变革管理和新技能的强制性培训课程。 \newline
- 建立对创新努力的实质性表彰、奖励政策。 \\
\hline

% 风险3:政策
\textbf{3. 宏观政策环境} & 
自主政策与其他规定(财务、人事、公共投资)之间缺乏同步性,限制了各校的行动空间和能力。 & 
中 & 
非常高 & 
- 各大学需通过协会(例如:越南大学与学院协会)主动向教育培训部和政府提出基于证据的政策建议。 \newline
- 教育培训部主导,与相关部委协调,审查并移除不再适宜的政策障碍。 \\
\hline

\end{longtable}
\end{filecontents*}

% --- 开始:风险分析与应对矩阵 ---
\renewcommand{\arraystretch}{1.3} % 增加行距以便阅读

\begin{longtable}{|p{2.5cm}|p{5cm}|p{1.5cm}|p{1.5cm}|p{5cm}|}
\caption{实施V-AQA模型的风险分析与应对矩阵}
\label{tab:risk_matrix}\\
\hline
\textbf{风险类别} & \textbf{具体风险描述} & \textbf{可能性} & \textbf{影响程度} & \textbf{缓解/应对措施} \\
\hline
\endfirsthead
\multicolumn{5}{l}{\textit{(续表 \ref{tab:risk_matrix})}}\\\hline
\textbf{风险类别} & \textbf{具体风险描述} & \textbf{可能性} & \textbf{影响程度} & \textbf{缓解/应对措施} \\
\hline
\endhead
\hline \multicolumn{5}{r}{\textit{续下页}}\\
\endfoot
\hline
\endlastfoot

\textbf{1. 资源} &
缺乏用于质量保障管理信息系统(QA-MIS)、升级设施、培训和奖励的预算。 &
高 &
高 &
\begin{itemize}
  \item 分阶段实施 (phased implementation)。
  \item 收入来源多样化,推动公私合作(PPP)。
  \item 优先采用开源技术。
\end{itemize} \\
\hline

\textbf{2. 人员与文化} &
\begin{itemize}
  \item 干部的惰性、变革抵触和怀疑心态。
  \item 来自中层管理的抵制。
  \item 缺乏数据分析和现代治理专家。
\end{itemize} &
非常高 &
非常高 &
\begin{itemize}
  \item 高层领导的承诺和示范作用。
  \item 透明沟通,创造“短期胜利”。
  \item 关于变革管理的强制性培训。
  \item 对创新给予实质性奖励。
\end{itemize} \\
\hline

\textbf{3. 政策环境} &
自主权与财务、人事、公共投资等规定之间缺乏同步性。 &
中 &
非常高 &
\begin{itemize}
  \item 学校通过协会主动提出政策建议。
  \item 教育培训部与各部委协调,审查并消除政策障碍。
\end{itemize} \\
\hline

\end{longtable}
% --- 结束:风险分析与应对矩阵 ---
    

\subsubsection{成功条件的深入分析}

从上述风险矩阵中,可以得出对V-AQA模型改革过程成功至关重要的三个先决条件。

\paragraph{条件一:来自最高层的强有力且持久的政治承诺。}
这是最重要的因素。转型过程不可避免地会遇到困难、冲突和短期内的效率下降。如果没有来自校董会和校领导班子足够强大、一致和坚韧的承诺,所有改革努力都将轻易地在遇到初步压力时偏离方向或半途而废。这种承诺必须通过具体行动来体现,从率先垂范、分配相应资源到保护敢于创新的人。

\paragraph{条件二:构建一个系统的变革管理战略。}
不能凭主观意愿强加变革。学校需要有一个有效的内部沟通战略,清晰地解释"为什么要变革?","变革将为每个个人和整个组织带来什么好处?"。建立一个包括各单位有威望人士的"变革领导联盟",并专注于创造"速赢"以建立信任和动力,是已被证明有效的战略步骤\footcite{kotter_1996}。

\paragraph{条件三:对技术基础设施和人的能力进行相应投资。}
质量保障管理信息系统是模型的"硬脊梁",而团队的数据分析和现代治理能力是"软脊梁"。这两个因素必须得到并行和相应的投资。一个现代化的技术系统,如果用户没有足够的能力去利用它,将变得毫无用处。反之,一个即使受过良好培训的团队,如果没有数据和工具来工作,也无能为力。因此,一个针对这两个方面进行同步投资的计划,是转向基于证据的治理模式的强制性条件。

总之,实施V-AQA模型的道路充满挑战,但并非不可行。它需要战略远见、政治决心以及在变革管理中科学、系统方法的结合。


% het goi 11 12 chuong 5











	
    \bibliographystyle{gbt7714-numerical}
	\bibliography{ref/refs}
	
	% \begin{appendix}
	% \input{data/appendix}
	% \end{appendix}
	
	% \include{data/paper}
	% \include{data/ack}

\end{document}
