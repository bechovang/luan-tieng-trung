
\chapter{结论与建议}
\label{chap:ket_luan_khuyen_nghi}

%-----------------------------------------------------------------------
% Mục 5.1: Tổng quan và Tóm tắt các Phát hiện Chính
%-----------------------------------------------------------------------

\section{主要研究成果总结与论文定位}
\label{sec:tong_hop_ket_qua}

本章作为总结与结论章节,汇集并凝练了本论文的全部研究成果。在经历了从理论基础分析(第二章)、“解剖”系统现状(第三章)到构建全面改革模型(第四章)的历程后,最后一章将承担三项核心任务:
\begin{enumerate}
    \item \textbf{总结}本论文的主要科学发现与贡献。
    \item \textbf{提出}一套具体可行的政策与行动建议,旨在完善越南高等教育质量保障体系。
    \item \textbf{坦诚地讨论}实施这些建议时面临的挑战、先决条件,同时指出研究的局限性及未来的潜在研究方向。
\end{enumerate}

本章的目标不仅是为研究工作画上句号,更是为政策制定者、教育管理者以及相关方提供一个系统的视角和一份有科学依据的“行动手册”,共同致力于可持续地提升越南高等教育质量。

\subsection{现状分析的核心发现总结}
\label{subsec:tom_tat_phat_hien}

第三章详细呈现的对2015-2024年间越南高等教育质量保障体系现状的深入分析,得出了重要结论,勾勒出一幅光明与阴影交织的多维度画面。

\paragraph{论点一:规模上的成就与广泛的合规性。}
不可否认,整个体系在扩大规模和执行质量认证规定方面取得了显著的努力和成就。截至2023-2024学年末,越南高等教育体系已发展到\textbf{243所培养机构,学生规模超过235万}\footcite{thanhnien_quymo_2024}。特别是在质量保障方面,截至2025年初,已有\textbf{208所高等教育机构(占74.8\%)完成了认证周期},并被承认达到国内或国际质量标准\footcite{dantri_kiemdinh_2025}。这些数字显示了在意识和合规行动上的广泛积极转变,为更深层次的质量改进步骤奠定了重要的初步基础。

\paragraph{论点二:“质量悖论”与系统性“恶性循环”的存在。}
然而,在规模和认证比例这些亮眼数字的背后,是一个深刻的\textbf{“发展悖论”}。尽管毕业生在12个月内的就业率相当高,但调查显示,其中只有约\textbf{76\%的人从事与所学专业相符的工作}\footcite{neu_tylevieclam}。这种持续的“不匹配”是一个明确的证据,表明培养质量尚未真正满足劳动力市场的要求。

本论文证明,这一悖论并非偶然,而是具有系统性的\textbf{“恶性循环”}的后果。第三章的分析指出了核心症结,包括:
\begin{itemize}
    \item 一种\textbf{带有合规性和应付色彩的质量文化},这种文化被一种仍然带有浓厚行政色彩和自上而下强加的管理机制所“滋养”\footcite{vjol_reactiveculture}。
    \item 与外部利益相关者,特别是企业界的\textbf{联系松散且缺乏实质性},导致培养方案变得陈旧且缺乏应用性\footcite{worldbank_improvingperformance}。
    \item 最严重的是,一个\textbf{薄弱、碎片化的数据管理系统},被视为体系的“阿喀琉斯之踵”,导致改进决策常常缺乏实证依据,未能达到预期效果。
\end{itemize}

清晰地识别这一悖论及其恶性循环,是迫切需要一个整体和系统的解决方案模型而非零散措施的实践基础。这也正是本章后续部分将重点解决的任务。

\subsection{基于V-AQA模型对核心挑战进行系统化梳理}
\label{subsec:he_thong_hoa_thach_thuc}

为获得一个系统的视角并为制定解决方案奠定基础,第三章已分析的挑战将按照V-AQA理论框架的5个要素进行总结。下表\ref{tab:tom_tat_thach_thuc_chuong5}提供了一份综合“诊断书”,指出了越南高等教育质量保障体系中问题的“症状”和“根本原因”。

\begin{table}[h!]
\centering
\caption{基于V-AQA模型5要素的系统性挑战总结表}
\label{tab:tom_tat_thach_thuc_chuong5}
\resizebox{\textwidth}{!}{%
\begin{tabular}{|p{3cm}|p{6.5cm}|p{6cm}|}
\hline
\textbf{V-AQA要素} & \textbf{主要挑战/“症状”} & \textbf{根本性系统原因} \\
\hline
\textbf{1. 领导与治理} & 
推动自主的政策与仍带有浓厚合规色彩的管理实践之间的矛盾。领导层将更多精力用于应付行政要求而非质量战略治理\footcite{lypham_aosat_2024}。 & 
“主管部门”机制依然存在,削弱了校董会的实质性自主权\footcite{khuyen_bochuquan_2022}。自主的法律框架不完善且缺乏同步性。 \\
\hline
\textbf{2. 质量文化} & 
“反应型质量文化”占主导,质量保障活动仅为应付认证而季节性进行。教师被动参与,视质量保障为负担\footcite{vjol_reactiveculture}。 & 
自上而下管理模式的后果。缺乏鼓励和表彰实质性改进努力的机制。缺乏关于质量文化的系统性培训项目。 \\
\hline
\textbf{3. 利益相关者的参与} & 
与企业的联系松散,形式化。在培养方案设计中咨询利益相关者(企业、学生)的做法有限,导致“技能差距”\footcite{worldbank_improvingperformance}。 &
“象牙塔思维”的遗留物。缺乏互利共赢的双边合作机制。学生和校友在学术治理中的角色仍然模糊\footcite{pmc_article_9127449}。 \\
\hline
\textbf{4. 内部流程} & 
\begin{itemize}
    \item 培养方案开发流程落后,预期学习成果模糊,打破了“建设性对齐”原则\footcite{ijlter_elo_copy}。
    \item 物质设施和信息通信技术基础设施薄弱\footcite{worldbank_improvingperformance}。
    \item \textbf{“阿喀琉斯之踵”:} 数据系统碎片化、不可靠,阻碍了基于证据的治理\footcite{aunsec_redesigningIQA}。
\end{itemize} &
重理论的课程开发思维。投资预算有限,采购流程复杂。在国家和学校层面缺乏一个集成的管理信息系统。 \\
\hline
\textbf{5. 合作与协调} & 
校内“孤岛”状态。各校之间的竞争环境压倒了合作与比对精神。学校、其他学校与社会之间缺乏战略合作模式。 & 
根深蒂固的“本位主义”、“部门主义”思维。缺乏推动实质性经验分享和相互学习的机制与平台。 \\
\hline
\end{tabular}
}
\end{table}

\vspace{0.5cm}

上述综合图景表明,越南质量保障体系的问题并非在于单一的某个方面,而是一系列具有因果关系、相互交织并相互强化的挑战集合。一种被动的质量文化是集权管理模式的后果,而它又反过来使得与企业建立联系或改进内部流程的努力变得更加困难。数据管理的薄弱又使得领导者缺乏战略决策的依据,恶性循环就这样持续下去。

清晰地识别这些系统性挑战及其相互关系,是最重要的科学基础,申明了任何想要成功的改革努力都必须采取一种整体性的方法。不能只专注于培训如何编写预期学习成果而忽略了与企业建立合作机制。同样,如果不能赋予学校实质性的自主权和现代化的管理工具,也不能要求学校进行创新。

基于对这些“瓶颈”的深刻理解,本论文将转到下一部分,申明其在理论和实践方面的贡献,然后提出一套旨在打破已指出的恶性循环的战略性解决方案体系。


% het goi 1 2


\section{本论文的新贡献}
\label{sec:dong_gop_luan_an}

基于实证分析结果,本论文在理论和实践两个层面都做出了具有意义的新贡献。本部分将重点阐明其在理论方面的贡献,申明所构建和验证的分析框架的科学价值与独特性。

\subsection{理论贡献}
\label{subsec:dong_gop_ly_luan}

本论文核心且贯穿始终的理论贡献在于成功构建并论证了一个\textbf{新的综合理论框架——混合与适应性质量保障模型(V-AQA)}。该贡献体现在三个主要方面。

\paragraph{首先,本论文指出了在复杂背景下应用单一理论的局限性。}
如第二章所分析和批判的,当经典的治理理论被独立地应用于像越南这样的转型期国家的高等教育领域时,都暴露了其局限性。
\begin{itemize}
    \item \textbf{新制度主义理论}很好地解释了同形现象以及为获得合法性而产生的合规压力\footcite{MeyerPowell2020},但未能充分解释大学自身的主动性、创新能力和对抗性策略。
    \item \textbf{利益相关者理论}揭示了利益冲突以及为多个群体创造价值的必要性\footcite{Freeman1984},但缺乏分析有形监督机制和塑造利益相关者优先级的权力结构的工具。
    \item \textbf{委托代理理论}为分析问责结构和控制机制提供了一套锐利的工具\footcite{JensenMeckling1976},但有将教育中不仅基于合同,还基于文化和无形规范的复杂关系简单化的风险。
\end{itemize}
通过明确指出这些“盲点”,本论文有力地申明了采用一种综合的多理论方法以把握全局的必要性。

\paragraph{其次,本论文成功构建了一个多维度、系统的分析模型。}
V-AQA模型并非机械的拼接,而是将不同理论视角综合并结构化为一个连贯分析框架的努力。该模型的五个要素——(1)领导与治理、(2)质量文化、(3)利益相关者的参与、(4)内部流程,以及(5)合作与协调——是建立在三大基础理论的交汇和互补之上的。
\begin{itemize}
    \item \textit{领导与治理}和\textit{内部流程}要素深受委托代理理论视角的影响。
    \item \textit{质量文化}和\textit{合作与协调}要素通过新制度主义理论得到了清晰的映照。
    \item \textit{利益相关者的参与}要素是利益相关者理论的直接应用。
\end{itemize}
构建这样一个综合模型,提供了一个新的、更全面的分析工具,允许从权力结构、文化规则到技术流程等多个维度来“解剖”质量保障体系的问题。

\paragraph{第三,V-AQA模型在分析转型期国家高等教育体系方面具有特殊的理论价值。}
一个重要的理论贡献在于该模型对特殊背景的适用性。与在欧美国家(拥有悠久的大学自主传统和发达的公民社会)发展的质量保障模型不同,V-AQA模型旨在捕捉像越南这样体系的典型张力:即\textbf{一个强大的、集中的国家管理机制}与来自\textbf{市场和社会利益相关者日益增长的压力}并存\footcite{WB_TransitionalQA}。V-AQA模型,凭借其在问责制与改进之间的“混合”哲学,以及应对变化背景的“适应性”原则,为解释和指导这些处于转型阶段的体系提供了一个更敏锐的理论框架。因此,本论文不仅对越南有价值,也可以在类似的国家背景下得到参考和验证。

总之,通过超越单一理论的局限性并构建一个综合、系统且对背景敏感的分析框架,本论文为丰富高等教育治理和质量保障的理论宝库做出了贡献。

\subsection{实践贡献}
\label{subsec:dong_gop_thuc_tien}

除了理论价值,本论文最重要和最实际的贡献在于其应用能力,即为越南的管理者和政策制定者提供了一个具体的分析框架和行动路线图。

\paragraph{首先,本论文提供了一个系统性诊断工具,用以识别核心“瓶颈”。}
V-AQA模型及第三章的分析为管理者提供了一个系统性的视角来“审视”和“诊断”自己的组织,而不是仅仅描述表面症状。通过从5个相互作用的要素来分析问题,一所大学可以:
\begin{itemize}
    \item 准确确定质量问题的根本原因。例如,学校可以不仅仅得出“学生软技能薄弱”的结论,而是可以追溯到“利益相关者参与”薄弱(没有来自企业的反馈)或“内部流程”薄弱(预期学习成果无法衡量软技能)的根源。
    \item 识别正在抑制自身发展的“恶性循环”,例如缺乏数据与决策效率低下之间的循环\footcite{aunsec_redesigningIQA}。
\end{itemize}
这个诊断工具有助于管理者从被动、零散地解决问题,转向主动和系统性的方法。

\paragraph{其次,本论文构建了一份全面且可行的“行动手册”。}
改革努力最大的挑战之一是政策与执行之间的差距\footcite{OECD_PolicyToAction}。本论文通过不仅仅停留在分析,还在第四章中构建了一套解决方案和一个详细的实施路线图,努力缩小这一差距。
\begin{itemize}
    \item \textbf{行动具体化:} 为V-AQA的每个要素提出的解决方案都非常具体,从建立行业咨询委员会、部署质量保障管理信息系统,到为跨学科项目设立资助基金。
    \item \textbf{优先级排序:} 为期7年、分3个阶段的实施路线图(基础与试点、扩展与标准化、优化与传播)提供了一条清晰的路径,帮助学校知道从何处着手以及接下来的步骤。
    \item \textbf{风险管理:} 对每组解决方案的潜在风险进行分析并提出缓解措施,大大增加了模型应用的现实性和成功可能性。
\end{itemize}
这份手册对于各大学的校领导班子、质量保障处负责人以及政策制定者在构建质量改革战略和行动计划的过程中尤其有用。

\paragraph{第三,本论文提供了一个协调的视角,帮助越南各校在一体化背景下定位。}
本论文深入分析了由国家主导的质量保障模型(中国经验)和基于网络、同行合作的模型(东盟大学网络质量保障框架)之间的差异。在此基础上,提出的V-AQA模型如同一条“中间道路”,一种有选择的综合,帮助越南大学:
\begin{itemize}
    \item 既能满足一个由国家管理体系的问责要求。
    \item 又能建立内在能力和一种主动的质量文化,以趋近国际标准。
\end{itemize}
这种方法帮助管理者避免了两个极端:要么不适当地机械照搬国际标准,要么封闭、自满于国内的最低规定。

总之,本论文不仅是一部供阅读的著作,更是一套供“使用”的工具。它为“诊断”问题提供了实证实据,为“思考”解决方案提供了理论框架,为“执行”改革提供了行动计划。这正是本论文所追求的最重要的实践贡献。

% het goi 3 4





















