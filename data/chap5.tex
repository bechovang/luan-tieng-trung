\chapter{结论与建议}
\label{chap:ket_luan_khuyen_nghi}

%-----------------------------------------------------------------------
% Mục 5.1: Tổng quan và Tóm tắt các Phát hiện Chính
%-----------------------------------------------------------------------

\section{主要研究成果总结与论文定位}
\label{sec:tong_hop_ket_qua}

本章作为总结与结论章节,汇集并凝练了本论文的全部研究成果。在经历了从理论基础分析(第二章)、"解剖"系统现状(第三章)到构建全面改革模型(第四章)的历程后,最后一章将承担三项核心任务:
\begin{enumerate}
    \item \textbf{总结}本论文的主要科学发现与贡献。
    \item \textbf{提出}一套具体可行的政策与行动建议,旨在完善越南高等教育质量保障体系。
    \item \textbf{坦诚地讨论}实施这些建议时面临的挑战、先决条件,同时指出研究的局限性及未来的潜在研究方向。
\end{enumerate}

本章的目标不仅是为研究工作画上句号,更是为政策制定者、教育管理者以及相关方提供一个系统的视角和一份有科学依据的"行动手册",共同致力于可持续地提升越南高等教育质量。

\subsection{现状分析的核心发现总结}
\label{subsec:tom_tat_phat_hien}

第三章详细呈现的对2015-2024年间越南高等教育质量保障体系现状的深入分析,得出了重要结论,勾勒出一幅光明与阴影交织的多维度画面。

\paragraph{论点一:规模上的成就与广泛的合规性。}
不可否认,整个体系在扩大规模和执行质量认证规定方面取得了显著的努力和成就。截至2023-2024学年末,越南高等教育体系已发展到\textbf{243所培养机构,学生规模超过235万}\footcite{thanhnien_quymo_2024}。特别是在质量保障方面,截至2025年初,已有\textbf{208所高等教育机构(占74.8\%)完成了认证周期},并被承认达到国内或国际质量标准\footcite{dantri_kiemdinh_2025}。这些数字显示了在意识和合规行动上的广泛积极转变,为更深层次的质量改进步骤奠定了重要的初步基础。

\paragraph{论点二:"质量悖论"与系统性"恶性循环"的存在。}
然而,在规模和认证比例这些亮眼数字的背后,是一个深刻的"发展悖论"。尽管毕业生在12个月内的就业率相当高,但调查显示,其中只有约\textbf{76\%的人从事与所学专业相符的工作}\footcite{neu_tylevieclam}。这种持续的"不匹配"是一个明确的证据,表明培养质量尚未真正满足劳动力市场的要求。

本论文证明,这一悖论并非偶然,而是具有系统性的"恶性循环"的后果。第三章的分析指出了核心症结,包括:
\begin{itemize}
    \item 一种\textbf{带有合规性和应付色彩的质量文化},这种文化被一种仍然带有浓厚行政色彩和自上而下强加的管理机制所"滋养"\footcite{vjol_reactiveculture}。
    \item 与外部利益相关者,特别是企业界的\textbf{联系松散且缺乏实质性},导致培养方案变得陈旧且缺乏应用性\footcite{worldbank_improvingperformance}。
    \item 最严重的是,一个\textbf{薄弱、碎片化的数据管理系统},被视为体系的"阿喀琉斯之踵",导致改进决策常常缺乏实证依据,未能达到预期效果。
\end{itemize}

清晰地识别这一悖论及其恶性循环,是迫切需要一个整体和系统的解决方案模型而非零散措施的实践基础。这也正是本章后续部分将重点解决的任务。

\subsection{基于V-AQA模型对核心挑战进行系统化梳理}
\label{subsec:he_thong_hoa_thach_thuc}

为获得一个系统的视角并为制定解决方案奠定基础,第三章已分析的挑战将按照V-AQA理论框架的5个要素进行总结。下表\ref{tab:tom_tat_thach_thuc_chuong5}提供了一份综合"诊断书",指出了越南高等教育质量保障体系中问题的"症状"和"根本原因"。

\begin{table}[h!]
\centering
\caption{基于V-AQA模型5要素的系统性挑战总结表}
\label{tab:tom_tat_thach_thuc_chuong5}
\resizebox{\textwidth}{!}{%
\begin{tabular}{|p{3cm}|p{6.5cm}|p{6cm}|}
\hline
\textbf{V-AQA要素} & \textbf{主要挑战/"症状"} & \textbf{根本性系统原因} \\
\hline
\textbf{1. 领导与治理} & 
推动自主的政策与仍带有浓厚合规色彩的管理实践之间的矛盾。领导层将更多精力用于应付行政要求而非质量战略治理\footcite{lypham_aosat_2024}。 & 
"主管部门"机制依然存在,削弱了校董会的实质性自主权\footcite{khuyen_bochuquan_2022}。自主的法律框架不完善且缺乏同步性。 \\
\hline
\textbf{2. 质量文化} & 
"反应型质量文化"占主导,质量保障活动仅为应付认证而季节性进行。教师被动参与,视质量保障为负担\footcite{vjol_reactiveculture}。 & 
自上而下管理模式的后果。缺乏鼓励和表彰实质性改进努力的机制。缺乏关于质量文化的系统性培训项目。 \\
\hline
\textbf{3. 利益相关者的参与} & 
与企业的联系松散,形式化。在培养方案设计中咨询利益相关者(企业、学生)的做法有限,导致"技能差距"\footcite{worldbank_improvingperformance}。 &
"象牙塔思维"的遗留物。缺乏互利共赢的双边合作机制。学生和校友在学术治理中的角色仍然模糊\footcite{pmc_article_9127449}。 \\
\hline
\textbf{4. 内部流程} & 
\begin{itemize}
    \item 培养方案开发流程落后,预期学习成果模糊,打破了"建设性对齐"原则\footcite{ijlter_elo_copy}。
    \item 物质设施和信息通信技术基础设施薄弱\footcite{worldbank_improvingperformance}。
    \item \textbf{"阿喀琉斯之踵":} 数据系统碎片化、不可靠,阻碍了基于证据的治理\footcite{aunsec_redesigningIQA}。
\end{itemize} &
重理论的课程开发思维。投资预算有限,采购流程复杂。在国家和学校层面缺乏一个集成的管理信息系统。 \\
\hline
\textbf{5. 合作与协调} & 
校内"孤岛"状态。各校之间的竞争环境压倒了合作与比对精神。学校、其他学校与社会之间缺乏战略合作模式。 & 
根深蒂固的"本位主义"、"部门主义"思维。缺乏推动实质性经验分享和相互学习的机制与平台。 \\
\hline
\end{tabular}
}
\end{table}

\vspace{0.5cm}

上述综合图景表明,越南质量保障体系的问题并非在于单一的某个方面,而是一系列具有因果关系、相互交织并相互强化的挑战集合。一种被动的质量文化是集权管理模式的后果,而它又反过来使得与企业建立联系或改进内部流程的努力变得更加困难。数据管理的薄弱又使得领导者缺乏战略决策的依据,恶性循环就这样持续下去。

清晰地识别这些系统性挑战及其相互关系,是最重要的科学基础,申明了任何想要成功的改革努力都必须采取一种整体性的方法。不能只专注于培训如何编写预期学习成果而忽略了与企业建立合作机制。同样,如果不能赋予学校实质性的自主权和现代化的管理工具,也不能要求学校进行创新。

基于对这些"瓶颈"的深刻理解,本论文将转到下一部分,申明其在理论和实践方面的贡献,然后提出一套旨在打破已指出的恶性循环的战略性解决方案体系。


% het goi 1 2


\section{本论文的新贡献}
\label{sec:dong_gop_luan_an}

基于实证分析结果,本论文在理论和实践两个层面都做出了具有意义的新贡献。本部分将重点阐明其在理论方面的贡献,申明所构建和验证的分析框架的科学价值与独特性。

\subsection{理论贡献}
\label{subsec:dong_gop_ly_luan}

本论文核心且贯穿始终的理论贡献在于成功构建并论证了一个\textbf{新的综合理论框架——混合与适应性质量保障模型(V-AQA)}。该贡献体现在三个主要方面。

\paragraph{首先,本论文指出了在复杂背景下应用单一理论的局限性。}
如第二章所分析和批判的,当经典的治理理论被独立地应用于像越南这样的转型期国家的高等教育领域时,都暴露了其局限性。
\begin{itemize}
    \item \textbf{新制度主义理论}很好地解释了同形现象以及为获得合法性而产生的合规压力\footcite{MeyerPowell2020},但未能充分解释大学自身的主动性、创新能力和对抗性策略。
    \item \textbf{利益相关者理论}揭示了利益冲突以及为多个群体创造价值的必要性\footcite{Freeman1984},但缺乏分析有形监督机制和塑造利益相关者优先级的权力结构的工具。
    \item \textbf{委托代理理论}为分析问责结构和控制机制提供了一套锐利的工具\footcite{JensenMeckling1976},但有将教育中不仅基于合同,还基于文化和无形规范的复杂关系简单化的风险。
\end{itemize}
通过明确指出这些"盲点",本论文有力地申明了采用一种综合的多理论方法以把握全局的必要性。

\paragraph{其次,本论文成功构建了一个多维度、系统的分析模型。}
V-AQA模型并非机械的拼接,而是将不同理论视角综合并结构化为一个连贯分析框架的努力。该模型的五个要素——(1)领导与治理、(2)质量文化、(3)利益相关者的参与、(4)内部流程,以及(5)合作与协调——是建立在三大基础理论的交汇和互补之上的。
\begin{itemize}
    \item \textit{领导与治理}和\textit{内部流程}要素深受委托代理理论视角的影响。
    \item \textit{质量文化}和\textit{合作与协调}要素通过新制度主义理论得到了清晰的映照。
    \item \textit{利益相关者的参与}要素是利益相关者理论的直接应用。
\end{itemize}
构建这样一个综合模型,提供了一个新的、更全面的分析工具,允许从权力结构、文化规则到技术流程等多个维度来"解剖"质量保障体系的问题。

\paragraph{第三,V-AQA模型在分析转型期国家高等教育体系方面具有特殊的理论价值。}
一个重要的理论贡献在于该模型对特殊背景的适用性。与在欧美国家(拥有悠久的大学自主传统和发达的公民社会)发展的质量保障模型不同,V-AQA模型旨在捕捉像越南这样体系的典型张力:即\textbf{一个强大的、集中的国家管理机制}与来自\textbf{市场和社会利益相关者日益增长的压力}并存\footcite{WB_TransitionalQA}。V-AQA模型,凭借其在问责制与改进之间的"混合"哲学,以及应对变化背景的"适应性"原则,为解释和指导这些处于转型阶段的体系提供了一个更敏锐的理论框架。因此,本论文不仅对越南有价值,也可以在类似的国家背景下得到参考和验证。

总之,通过超越单一理论的局限性并构建一个综合、系统且对背景敏感的分析框架,本论文为丰富高等教育治理和质量保障的理论宝库做出了贡献。

\subsection{实践贡献}
\label{subsec:dong_gop_thuc_tien}

除了理论价值,本论文最重要和最实际的贡献在于其应用能力,即为越南的管理者和政策制定者提供了一个具体的分析框架和行动路线图。

\paragraph{首先,本论文提供了一个系统性诊断工具,用以识别核心"瓶颈"。}
V-AQA模型及第三章的分析为管理者提供了一个系统性的视角来"审视"和"诊断"自己的组织,而不是仅仅描述表面症状。通过从5个相互作用的要素来分析问题,一所大学可以:
\begin{itemize}
    \item 准确确定质量问题的根本原因。例如,学校可以不仅仅得出"学生软技能薄弱"的结论,而是可以追溯到"利益相关者参与"薄弱(没有来自企业的反馈)或"内部流程"薄弱(预期学习成果无法衡量软技能)的根源。
    \item 识别正在抑制自身发展的"恶性循环",例如缺乏数据与决策效率低下之间的循环\footcite{aunsec_redesigningIQA}。
\end{itemize}
这个诊断工具有助于管理者从被动、零散地解决问题,转向主动和系统性的方法。

\paragraph{其次,本论文构建了一份全面且可行的"行动手册"。}
改革努力最大的挑战之一是政策与执行之间的差距\footcite{OECD_PolicyToAction}。本论文通过不仅仅停留在分析,还在第四章中构建了一套解决方案和一个详细的实施路线图,努力缩小这一差距。
\begin{itemize}
    \item \textbf{行动具体化:} 为V-AQA的每个要素提出的解决方案都非常具体,从建立行业咨询委员会、部署质量保障管理信息系统,到为跨学科项目设立资助基金。
    \item \textbf{优先级排序:} 为期7年、分3个阶段的实施路线图(基础与试点、扩展与标准化、优化与传播)提供了一条清晰的路径,帮助学校知道从何处着手以及接下来的步骤。
    \item \textbf{风险管理:} 对每组解决方案的潜在风险进行分析并提出缓解措施,大大增加了模型应用的现实性和成功可能性。
\end{itemize}
这份手册对于各大学的校领导班子、质量保障处负责人以及政策制定者在构建质量改革战略和行动计划的过程中尤其有用。

\paragraph{第三,本论文提供了一个协调的视角,帮助越南各校在一体化背景下定位。}
本论文深入分析了由国家主导的质量保障模型(中国经验)和基于网络、同行合作的模型(东盟大学网络质量保障框架)之间的差异。在此基础上,提出的V-AQA模型如同一条"中间道路",一种有选择的综合,帮助越南大学:
\begin{itemize}
    \item 既能满足一个由国家管理体系的问责要求。
    \item 又能建立内在能力和一种主动的质量文化,以趋近国际标准。
\end{itemize}
这种方法帮助管理者避免了两个极端:要么不适当地机械照搬国际标准,要么封闭、自满于国内的最低规定。

总之,本论文不仅是一部供阅读的著作,更是一套供"使用"的工具。它为"诊断"问题提供了实证实据,为"思考"解决方案提供了理论框架,为"执行"改革提供了行动计划。这正是本论文所追求的最重要的实践贡献。

% het goi 3 4


\section{旨在完善越南高等教育质量保障体系的建议体系}
\label{sec:he_thong_khuyen_nghi}

基于现状分析结果和已确认的科学贡献,本论文提出了一套政策建议和战略行动体系。这些建议并非孤立的解决方案,而是经过精心构建,旨在\textbf{打破第三章已识别的"恶性循环"},并创造一个可持续质量改进的\textbf{"良性循环"}。

该建议体系基于V-AQA模型的核心原则构建:
\begin{itemize}
    \item \textbf{系统性:} 同步作用于模型的全部5个要素,从宏观层面(国家政策)到微观层面(各学校的活动)。
    \item \textbf{可行性:} 建议按优先级划分,并结合越南的实际情况。
    \item \textbf{可衡量性:} 每组行动都提出了具体的关键绩效指标,以便进行跟踪和评估。
\end{itemize}

下表\ref{tab:tong_quan_khuyen_nghi}提供了整个建议体系的概览,按主要对象群体、具体行动、优先级及与V-AQA模型要素的关联进行分类。该表将作为后续详细分析部分的指南。

\begin{longtable}{|p{3.5cm}|p{5.5cm}|p{2cm}|p{3.5cm}|}
\caption{基于V-AQA模型的建议体系概览表}
\label{tab:tong_quan_khuyen_nghi} \\
\hline
\textbf{对象} & \textbf{建议的战略行动} & \textbf{优先级} & \textbf{主要作用的V-AQA要素} \\
\hline
\endfirsthead
\multicolumn{4}{c}%
{{\bfseries \tablename\ \thetable{} -- 续前页}} \\
\hline
\textbf{对象} & \textbf{建议的战略行动} & \textbf{优先级} & \textbf{主要作用的V-AQA要素} \\
\hline
\endhead
\hline \multicolumn{4}{r}{{续下页}} \\
\endfoot
\hline
\endlastfoot

% 第1行:国家管理机构
\textbf{1. 国家管理机构(教育培训部)} & 
1.1. 将角色从微观控制转变为环境创造和宏观调控。 \newline
1.2. 推动基于绩效的拨款机制。 \newline
1.3. 促进一个多样化和竞争性的质量保障生态系统,承认国际质量认证组织。 \newline
1.4. 战略性投资建设集成的国家高等教育数据系统。 & 
短期 (1.1, 1.4) \newline 中期 (1.2, 1.3) & 
1. 领导与治理 \newline
5. 合作与协调 \\
\hline

% 第2行:高等教育机构
\textbf{2. 高等教育机构} & 
2.1. 实质性地制定并执行校级质量战略。 \newline
2.2. 发起从基层建设质量文化的运动。 \newline
2.3. 建立并机制化行业咨询委员会的运作。 \newline
2.4. 按照成果导向教育实现培养方案开发流程现代化并多样化评估方法。 \newline
2.5. 投资建设质量保障管理信息系统。 \newline
2.6. 主动参与标杆比对和同行评审网络。 & 
短期 (2.1, 2.2) \newline 中期 (2.3, 2.4, 2.5) \newline 长期 (2.6) & 
同步作用于V-AQA模型的全部5个要素。 \\
\hline

% 第3行:质量认证中心
\textbf{3. 质量认证中心} & 
3.1. 提升认证员队伍的专业能力和职业素养。 \newline
3.2. 为不同类型的学校开发专门的评估标准或指南。 \newline
3.3. 加强结果和综合分析报告的公开透明化。 & 
中期 & 
4. 内部流程 \newline
5. 合作与协调 \\
\hline

% 第4行:各协会
\textbf{4. 企业与行业协会} & 
4.1. 主导构建"职业能力框架"。 \newline
4.2. 主动参与行业咨询委员会及培养方案评估活动。 & 
中期 & 
3. 利益相关者的参与 \\
\hline

\end{longtable}

\vspace{0.5cm}

如上表所示,各项建议并非仅集中于单一主体,而是要求从国家管理机构、各大学到认证组织和企业界等整个教育生态系统的共同努力和同步行动。每组行动都旨在解决已分析的相应"瓶颈",最终目标是在质量的思维和行动上实现根本性转变。

本章的后续部分将深入分析和论证每一组建议,包括其科学依据、需要开展的具体行动以及配套的关键绩效指标。

\subsection{对国家管理机构的建议}
\label{subsec:khuyen_nghi_qlnn_detailed}

作为体系的"总建筑师",国家管理机构,直接是教育培训部,在为质量改革创造有利的制度环境方面扮演着决定性角色。以下建议侧重于将国家角色从直接控制转变为环境创造和宏观调控,这一趋势已在许多先进教育体系中被证明有效\footcite{OECD_PolicyToAction}。

\subsubsection{建议1.1:将角色从"微观控制者"转变为"环境创造者和宏观调控者"}

\paragraph{论据:}
第三章的分析指出了一个核心矛盾:虽然宏观政策朝向自主,但具体的管理机制仍然带有浓厚的行政色彩,催生了"合规文化",并削弱了学校实质性改进的动力\footcite{pham2021governance}。教育培训部颁布的包含过多流程性、程序性标准的评估标准(例如,根据第12/2017号通知的25项标准\footcite{tt12_2017_bgddt}),无意中引导了各学校专注于应付性地完善档案,而非进行战略性改进。

为使\textbf{2018年修订的《高等教育法》}\footcite{luatvn_gddh_2018}精神真正落到实处,国家管理机构需要转变其角色。国家不应是"手把手指导者",而应是建立公平"游戏规则"者,监督最低标准的遵守情况,最重要的是,推动一个透明的环境,以便社会能够发挥其监督作用。这是从委托代理理论(委托人严密监督代理人)向更现代的治理模式的转变,即国家本着新制度主义精神,创造一个健康的"制度场域"。

\paragraph{具体行动:}
\begin{enumerate}
    \item \textbf{改革质量认证标准:} 研究修订教育机构和培养项目的质量认证标准,使其更精简,侧重于产出和核心质量保障条件(例如:师资能力、基础设施、有效治理),减少带有浓厚行政流程、程序色彩的标准。新标准需要有"开放空间",让各校可以根据自身使命和独特战略,以不同方式证明其质量。
    
    \item \textbf{加强"认证认证机构"的角色:} 教育培训部需要专注于建立严格监督各质量认证中心活动运作的标准和流程,确保这些组织的独立性、专业性和道德性。赋予认证中心更大的权力和更高的问责要求,逐步减少教育部对具体评估活动的直接干预。
    
    \item \textbf{建立国家高等教育质量信息门户:} 建立一个单一的电子信息门户,在此\textbf{完全}公开所有高等教育机构的自评报告、外部评估报告以及主要质量指标(例如:就业率、规模、师资队伍)。这种透明度是最有效的社会监督工具,能创造健康的竞争压力,并迫使各校对其质量真正负责。
\end{enumerate}

\paragraph{关键绩效指标:}
\begin{itemize}
    \item 新标准体系中侧重于产出的标准与侧重于投入/流程的标准的比例。
    \item 各大学和认证中心对新政策的清晰度和支持性的满意度。
    \item 国家质量信息门户的数据访问和利用次数。
\end{itemize}

\subsubsection{建议1.2:推动基于绩效的拨款机制}

\paragraph{论据:}
单纯的行政压力通常只能带来合规。要创造实质性的改进动力,需要有财政激励工具。已在许多经合组织国家成功应用的基于绩效的拨款机制,是一个强大的政策工具,能将各校的重点从维持运作转向追求卓越\footcite{oecd_pbf_2021}。通过将部分预算与产出结果挂钩,国家将发出一个明确的信息:"质量将得到奖赏"。这将促使各校领导层更具战略性地思考如何改善核心质量指标。

\paragraph{具体行动:}
\begin{enumerate}
    \item \textbf{构建国家基于绩效的拨款指标体系:} 教育培训部主导,与专家、各大学及相关方协调,构建一套清晰、透明且可衡量的关键绩效指标体系,作为预算分配的依据。该指标体系可包括以下几组:
        \begin{itemize}
            \item \textit{培养质量:} 毕业生12个月内从事对口工作的比例(数据来自高等教育管理信息系统与社会保险系统联动),学生满意度。
            \item \textit{研究能力:} 每位教师在权威期刊(Scopus/WoS)上的国际发表数量,来自科技活动和知识转移的收入。
            \item \textit{国际化水平:} 国际学生和教师比例,由权威国际组织认证的项目数量。
        \end{itemize}
    \item \textbf{按路线图进行试点:} 开始在已实现自主的公立大学群体中试点实施基于绩效的拨款机制。在初期阶段,可以根据关键绩效指标的达成结果分配一部分预算(例如:总经常性支出的10-20%)。
    \item \textbf{评估与调整:} 每2-3年周期后,需要有一份独立的影��评估报告,分析基于绩效的拨款机制的积极和消极影响(如有),从而在考虑推广前,对关键绩效指标体系和分配方式进行适当调整。
\end{enumerate}

\paragraph{关键绩效指标:}
\begin{itemize}
    \item 按基于绩效的拨款机制分配的高等教育预算百分比。
    \item 试点院校群体的关键绩效指标平均改善程度与非试点院校群体的比较。
    \item 各大学对基于绩效的拨款机制的共识度和支持度。
\end{itemize}


% het goi 5 6


\subsubsection{建议1.3:促进一个多样化和竞争性的质量保障生态系统}

\paragraph{论据:}
一个质量保障体系只有在评估主体多样化且他们之间存在良性竞争时,才能真正有效。过度依赖少数几家仍受教育培训部管理的国内认证中心,可能导致缺乏客观性且不鼓励创新\footcite{giaoducnet_kdcl_list_2023}。相反,开放并承认有信誉的国际认证组织将带来诸多好处:
\begin{itemize}
    \item \textbf{增强客观性与国际接轨:} 国际认证组织按全球标准运作,帮助越南大学客观地看待自身质量,并加速国际化进程。实际上,已有\textbf{11所越南大学获得了像德国工商管理认证基金会、英国质量保障署、法国高等教育与研究评估高等委员会等国外权威组织的认证},这表明顶尖大学确有接轨的需求和能力\footcite{thuvienphapluat_11truong_quocte}。
    \item \textbf{创造积极的竞争压力:} 国际组织的存在将产生压力,迫使国内认证中心不断提升其专业能力和服务质量以求竞争。
    \item \textbf{推动按专业领域的认证:} 许多国际组织专门从事特定领域的认证(例如:工程领域的ABET,商科领域的AACSB)。参与专业认证将有助于各院系实质性地提升培养质量,并使其与各行业的要求相符。
\end{itemize}

\paragraph{具体行动:}
\begin{enumerate}
    \item \textbf{颁布承认国际认证结果的规定:} 建立并颁布一个通畅透明的法律走廊,用于承认有信誉的国际组织的认证结果,特别是那些全球网络如\textbf{国际高等教育质量保障机构网络}或区域网络如\textbf{亚太质量网络}的成员组织。
    \item \textbf{鼓励专业认证:} 为优先领域(技术、工程、经济、健康)的培养项目参与并获得世界公认的专业组织认证提供具体鼓励政策(例如:资助部分经费,在评优项目中加分)。
    \item \textbf{推动行业协会的角色:} 创造机制,让国内的行业协会(例如:建筑工程师协会、越南会计与审计协会)参与构建职业能力标准,并逐步参与相关培养项目的评估和承认。
\end{enumerate}

\paragraph{关键绩效指标:}
\begin{itemize}
    \item 获得有信誉的国际组织认证的培养项目数量。
    \item 成为国际网络(国际高等教育质量保障机构网络、亚太质量网络)成员的国内认证中心比例。
    \item 由行业协会颁布并被各校用作参考的职业能力标准数量。
\end{itemize}

\subsubsection{建议1.4:战略性投资建设集成的国家高等教育数据系统}

\paragraph{论据:}
正如多次分析和强调的,数据的薄弱和碎片化是越南高等教育治理体系的"阿喀琉斯之踵"\footcite{aunsec_redesigningIQA}。缺乏同步和可靠的数据,所有宏观管理努力,从政策规划、按绩效拨款到人力需求预测,都变得缺乏科学依据。颁布关于教育行业统计报告制度的\textbf{第25/2024号通知}是朝着正确方向迈出的一步,但需要一个足够强大的技术系统来实现\footcite{luatvietnam_tt25_2024}。

因此,投资建设一个国家高等教育管理信息系统不应被视为一项开支,而应是对\textbf{治理基础设施的一项战略投资},有可能为整个体系带来巨大的效率和透明度回报。

\paragraph{具体行动:}
\begin{enumerate}
    \item \textbf{成立国家高等教育管理信息系统指导委员会:} 教育培训部应主导,与相关部委(计划与投资部、财政部、信息与通信部、越南社会保险)协调,制定一个总体方案并为此项目分配国家资源。
    
    \item \textbf{设计现代化的系统架构:} 国家高等教育管理信息系统需被设计为一个国家\textbf{数据仓库},能够按照一个共同标准整合来自各校信息系统的数据。核心功能包括:收集、清洗、存储、分析和数据可视化。
    
    \item \textbf{部署突破性构件——与社会保险数据联通:} 这是一个旨在解决衡量产出质量难题的革命性方案。建立机制(有严格的个人信息保密规定),将毕业生数据(根据个人身份识别码)与越南社会保险的数据库进行比对。这将提供关于以下方面的\textbf{客观、准确和实时更新}的图景:
        \begin{itemize}
            \item 学生的实际就业率。
            \item 按学校、专业的平均收入水平。
            \item 获得第一份工作的平均时间。
            \item 职业流动和从事与专业不符工作的比例。
        \end{itemize}
    这些数据将是基于绩效的拨款机制最重要的输入,并帮助社会对每个培养项目的真实"价值"有一个透明的了解。
    
    \item \textbf{建立开放数据(Open Data)利用政策:} 在匿名化处理后,一部分高等教育管理信息系统的数据应以开放数据的形式公布,以便研究人员、独立组织和社会可以利用,进行深入分析,有助于推动一个透明和基于证据的治理。
\end{enumerate}

\paragraph{关键绩效指标:}
\begin{itemize}
    \item 已连接并与国家高等教育管理信息系统同步数据的高等教育机构百分比。
    \item 汇总和公布全行业统计报告的平均时间(从每年减少到每季度/每月)。
    \item 基于国家高等教育管理信息系统数据构建的政策分析报告数量。
\end{itemize}

\subsection{对高等教育机构的建议}
\label{subsec:khuyen_nghi_csgddh_detailed}

虽然国家的宏观政策创造了环境和法律走廊,但质量的实质性转变必须来自每所大学自身的内部努力。以下建议按照V-AQA模型的五个要素构建,旨在为教育管理者提供一个具体的行动框架,以从内部引领变革。

\subsubsection{建议2.1:重构领导与治理——从"控制者"转变为"环境创造者"}

\paragraph{论据:}
首要需要解开的症结是领导团队的角色。如第三章所分析,当领导者陷入行政管理和应付合规要求时,他们将没有精力和远见来执行其战略角色\footcite{lypham_aosat_2024}。为打破此恶性循环,高层领导(校董会、校领导班子)的角色需要重新定位。他们不是直接从事质量工作的人,而是为质量能够在各级生根发芽和发展创造一个制度、文化和资源环境的人,这符合"创业型大学"的精神\footcite{clark_1998}。

\paragraph{具体行动:}
\begin{enumerate}
    \item \textbf{制定并承诺执行校级质量战略:} 领导层必须主导制定一份正式的五年期\textbf{质量战略},并将其融入学校的总体发展战略中。该文件不应束之高阁,而应是所有行动的指南,其中必须明确定义:质量愿景、优先目标、具体的关键绩效指标以及相应的资源分配计划。
    \item \textbf{强力分权并辅以问责制:} 在学术事务(改进培养方案、创新教学方法)上给予院/系更实质性的自主权。同时,建立一套清晰的关键绩效指标体系来衡量各单位的绩效。领导层将通过质量保障管理信息系统执行战略监督角色,而不是对日常运作进行微观干预。
    \item \textbf{投资于中层管理团队的治理能力:} 为院/系主任、副主任、处室负责人组织关于现代大学治理、变革管理和数据驱动决策的强制性培训项目。这是对"继任团队"的投资,以确保改革的可持续性。
\end{enumerate}

\paragraph{关键绩效指标:}
\begin{itemize}
    \item 每年在质量战略中设定的目标的完成率。
    \item 院/系领导对所获自主权程度的满意度调查得分。
    \item 100%的中层管理干部在两年内完成现代治理培训课程。
\end{itemize}

\subsubsection{建议2.2:塑造质量文化——从"应付"到"主动"}

\paragraph{论据:}
质量文化是无形资产,但决定了所有质量保障体系的成败。一个拥有最佳流程的体系,如果由抱着应付、被动心态的人来运作,也终将失败\footcite{iosr_passiveparticipation}。因此,从"合规文化"向"改进文化"的转型是一项战略性任务,需要采取既作用于组织中每个成员的认知又作用于其行动的解决方案\footcite{HarveyStensaker2008}。

\paragraph{具体行动:}
\begin{enumerate}
    \item \textbf{发起宣传和意识提升运动:} 组织如年度"质量周"、 "教学创新创举"竞赛、关于质量问题的开放论坛等活动。目标是让所有成员明白,质量不仅是质量保障处的责任。
    \item \textbf{建立表彰和奖励改进创举的体系:} 设立年度奖项("教学创新年度教师奖"、"高效质量保障模式院系奖")。更重要的是,将对质量改进的贡献标准纳入干部、教师的评估、评级和晋升流程中。
    \item \textbf{为"质量改进小组"赋能:} 在系级成立由自愿教师组成的小组(质量圈),任务是定期讨论并提出改进教学质量的方案。学校需要提供小额预算("创举支持基金"),以便这些小组可以试验新想法。
\end{enumerate}

\paragraph{关键绩效指标:}
\begin{itemize}
    \item 全校范围内关于质量文化认知的定期调查得分(逐年提高)。
    \item 每年提出并获资助实施的改进创举数量。
    \item 自愿参与质量改进活动的教职工比例。
\end{itemize}


% het 7 8 chuong 5


\subsubsection{建议2.3:将利益相关者从"咨询对象"转变为"战略伙伴"}

\paragraph{论据:}
"技能差距"\footcite{britishcouncil_skills_gap_2021}最深层的原因之一,是在学术流程中缺乏来自利益相关者,特别是雇主和校友的实质性声音。咨询活动(如果有的话)通常仅停留在形式化和被动的层面,如单向咨询或影响甚微的调查\footcite{vnujs_er_3848}。为打破"象牙塔壁垒",学校需要主动创造机制,使利益相关者成为共同创造价值的伙伴,这符合阿恩斯坦(1969)"参与阶梯"最高阶梯的精神\footcite{Arnstein1969}。

\paragraph{具体行动:}
\begin{enumerate}
    \item \textbf{将行业咨询委员会机制化并赋权:}
    学校需要为院系/专业层面成立行业咨询委员会并颁布正式的运作章程,而不是依赖个人关系。行业咨询委员会的角色不仅是咨询,还必须被赋予具体权力:
        \begin{itemize}
            \item \textbf{强制性审定}培养方案的预期学习成果。
            \item \textbf{定期参与}(每年至少1-2次)培养方案的审查和更新会议。
            \item \textbf{共同指导}和评估学生的毕业设计项目。
        \end{itemize}
    该模型是东盟大学网络质量保障\footcite{aunqa_guidelines_v4}和ABET\footcite{abet_criteria}等国际认证标准的核心要求,将创造一个可持续的对话与合作渠道,确保培养方案始终与实践相符。

    \item \textbf{提升学生和校友的角色:}
        \begin{itemize}
            \item \textbf{在学术委员会中为学生赋权:} 让(通过民主选举产生的)学生代表成为院/校级科学与培养委员会中拥有\textbf{投票权}的正式成员。
            \item \textbf{建立主动的"校友大使"网络:} 建立一个由事业有成且热心的校友组成的体系,其作用是定期提供关于培养方案契合度的反馈,并参与为低年级学生提供职业导向活动。
        \end{itemize}
\end{enumerate}

\paragraph{关键绩效指标:}
\begin{itemize}
    \item 来自行业咨询委员会的建议被记录并整合到培养方案改进计划中的百分比。
    \item 有学生代表参与并投票的学术决策数量。
    \item 企业对毕业生工作胜任度的满意度得分(逐年提高)。
\end{itemize}

\subsubsection{建议2.6:通过合作与比对打破"筒仓思维"}

\paragraph{论据:}
在一个孤立的环境中,质量无法得到可持续的改进。各处、室、院系之间的"孤岛"状态以及各校之间不健康的竞争,削弱了整个体系的协同力量和学习能力。要成为一个真正的"学习型组织"\footcite{Senge2006},各校需要主动创造在内部、校际和国际三个层面上的合作机制。

\paragraph{具体行动:}
\begin{enumerate}
    \item \textbf{促进内部合作:} 设立一个\textbf{专门资助跨单位质量改进项目的基金}。优先为有多个院、处、室参与的项目提供经费(例如:"提升软技能"项目需要学生工作处、各专业院系和质量保障处的协调)。
    
    \item \textbf{建立校际标杆比对网络:} 同一专业领域或地区的学校需要主动成立\textbf{质量标杆比对俱乐部}\footcite{jackson_lund_2000}。这些俱乐部可以:
        \begin{itemize}
            \item 通过一个中立的第三方,统一并分享(已匿名的)关于质量指标的数据集。
            \item 组织定期的同行评审活动和"最佳实践"分享研讨会。
        \end{itemize}
    该机制将把竞争转化为学习的动力,帮助各校客观地认识自己的优势和劣势。

    \item \textbf{战略性地利用国际合作活动:} 参与国际认证不应仅停留在获得证书的目标上。学校需要将一个\textbf{"认证后"流程制度化},其中质量保障处必须制定一个具体的行动计划来落实专家组的建议,跟踪进度并向校领导班子报告。
\end{enumerate}

\paragraph{关键绩效指标:}
\begin{itemize}
    \item 每年成功实施的跨院系/处室改进项目数量。
    \item 组织的标杆比对活动数量以及基于比对结果实施的改进措施数量。
    \item 按计划完成的国际认证建议的百分比。
\end{itemize}

\subsubsection{建议2.4与2.5:流程现代化与数据基础设施建设}

\paragraph{论据:}
第三章的分析已指出,内部流程,特别是课程开发流程和数据管理系统,正是越南质量保障体系的"阿喀琉斯之踵"。一个落后、脱节的培养方案开发流程将产生不合格的产品\footcite{ijlter_elo_copy}。同样,一个薄弱的数据系统将瘫痪基于证据的决策能力,使所有改进努力都变得盲目且低效\footcite{aunsec_redesigningIQA}。因此,实现这些流程的现代化并构建一个坚实的技术基础设施是强制性任务,是整个V-AQA模型能够运作的"硬件"基础。

\paragraph{具体行动1:按照成果导向教育理念标准化培养方案开发流程。}
V-AQA模型建议各大学全面转向基于\textbf{成果导向教育}理念的培养方案开发方法。这种方法有助于确保培养目标、教学活动和评估方法之间的紧密联系("建设性对齐"原则\footcite{biggs_constructive_alignment})。为避免形式化应用,该流程必须包括五个紧密相连的步骤:
\begin{enumerate}
    \item \textbf{利益相关者需求分析:} 始于收集和分析来自劳动力市场、行业专家和校友的要求。
    \item \textbf{预期学习成果的制定与审定:} 专业院系制定清晰、可衡量的预期学习成果,且必须由\textbf{行业咨询委员会强制性审定}。
    \item \textbf{设计课程地图:} 构建矩阵,清晰展示每门课程如何为实现预期学习成果做出贡献。
    \item \textbf{审定与批准:} 完整的培养方案必须由科学与培养委员会(包括外部专家)批准。
    \item \textbf{定期审查与改进:} 建立正式的培养方案审查周期(2-3年一次),基于利益相关者的反馈和实际学习成果。
\end{enumerate}

\paragraph{具体行动2:构建集成的质量保障管理信息系统。}
为打破数据恶性循环,V-AQA模型提出了一个基础性的技术解决方案:构建一个集成的\textbf{质量保障管理信息系统}。这是学校的"神经系统",任务是将来自各分散来源(教学、人事、调查、财务、科研)的数据整合到一个单一的\textbf{数据仓库}中。
质量保障管理信息系统的最终目标是将原始数据转化为对各级决策有用的信息。为实现此目标,系统需配备一个\textbf{商业智能}模块,以自动生成可视化的报告和仪表盘,并为不同用户角色进行定制和授权\footcite{educause_bi_2022}。例如:
\begin{itemize}
    \item \textbf{校领导班子仪表盘:} 显示校级战略关键绩效指标,如就业率、国际发表数量、雇主满意度(如表\ref{tab:dashboard_hieu_truong})。
    \item \textbf{院长仪表盘:} 显示院级运营关键绩效指标,如按时毕业率、学生对培养方案的满意度、院级科研课题数量(如表\ref{tab:dashboard_truong_khoa})。
    \item \textbf{教师仪表盘:} 提供关于教学活动的反馈、所负责班级学生的学习成果以及个人任务的执行进度(如表\ref{tab:dashboard_giang_vien})。
\end{itemize}
部署一个质量保障管理信息系统不仅是一个技术解决方案,更是一场工作方式的改革,需要领导层的坚定承诺和对用户的系统培训计划。

\paragraph{关键绩效指标:}
\begin{itemize}
    \item 按照五步成果导向教育流程构建和审查的培养方案百分比。
    \item (院级及以上)发布的、引用或基于质量保障管理信息系统数据的管理决策比例。
    \item 各级管理人员和教师对质量保障管理信息系统的实用性和便利性的满意度。
\end{itemize}


% het goi 9 10 chuong 5



\section{讨论:挑战、条件与前景}
\label{sec:ban_luan_trien_vong}

提出一个全面的改革模型是必要的,但在实践中成功实施该模型则是一个更大的挑战。本部分将坦诚地讨论应用V-AQA模型时最大的困难和障碍,其中,大学自主问题被视为基础性且具有决定性的因素。

\subsection{大学自主:一个有效质量保障体系的先决条件}
\label{subsec:banluan_tuchu}

纵观本论文的分析和建议,大学自主的角色始终被强调为一个基础性因素。可以肯定地说,\textbf{大学自主并非一个选项,而是V-AQA模型及质量改进方案能够成功和可持续实施的先决条件}。一个质量保障体系只有在大学被赋予权力和创新空间的环境中,才能最大限度地发挥其效用\footcite{eua_autonomy_qa}。

这种关系是辩证的。如果没有在学术、人事组织和财务方面的实质性自主权,领导者和教师将没有足够的动力和工具来按照V-AQA模型进行改革。反之,成功实施V-AQA模型,凭借其数据的透明度和明确的问责制,正是一所大学证明自己值得被赋予更大自主权的最好方式。

因此,实施V-AQA不能脱离执行\textbf{2018年修订的《高等教育法》}的路线图\footcite{luatvn_gddh_2018}。该模型的成功程度将直接取决于学校被赋予的自主程度。下面的矩阵图(图\ref{fig:matrix_vqa_autonomy})分析了V-AQA各要素在三种不同自主情景下的运作情况。

\begin{figure}[h!]
    \centering
    \caption{V-AQA模型按大学自主程度的运作矩阵}
    \label{fig:matrix_vqa_autonomy}
    \resizebox{\textwidth}{!}{%
    \begin{tabular}{|p{3cm}|p{4.5cm}|p{4.5cm}|p{4.5cm}|}
    \hline
    \textbf{V-AQA要素} & \textbf{低自主情景} \newline \textit{(合规环境)} & \textbf{中等自主情景} \newline \textit{(转型环境)} & \textbf{高自主情景} \newline \textit{(改进环境)} \\
    \hline
    \textbf{1. 领导与治理} & 领导主要执行行政命令,专注于满足主管部门的报告要求。 & 开始有战略决策的空间,但仍受行政规定约束。领导角色具有"混合"性质。 & 领导真正扮演战略创造者的角色。校董会在导向和监督方面拥有实质性权力。 \\
    \hline
    \textbf{2. 质量文化} & "反应型"和应付式文化。质量保障被视为负担,仅在有认证要求时才进行。 & 合规文化与改进文化之间存在拉锯。试点单位的"速赢"对于创造动力至关重要。 & 改进文化深入所有活动。质量是每个成员的内在责任。创新精神受到鼓励。 \\
    \hline
    \textbf{3. 利益相关者的参与} & 联系是形式化的,主要为完善证明档案。行业咨询委员会(如有)仅具装饰性。 & 行业咨询委员会开始被更系统地建立。在咨询中开始倾听学生的声音。 & 利益相关者(企业、学生、校友)成为"战略伙伴",共同创造课程和学校的价值。 \\
    \hline
    \textbf{4. 内部流程} & 流程僵化,按国家统一规定标准化。质量保障管理信息系统仅是用于导出报告的工具。 & 少数先锋院/系获准试点新流程(成果导向教育、基于项目的学习)。质量保障管理信息系统开始用于内部分析。 & 流程灵活,具有高度适应性。质量保障管理信息系统成为"神经系统",是各级基于证据决策不可或缺的工具。 \\
    \hline
    \textbf{5. 合作与协调} & 合作受限,通常仅按上级指示进行。"暗中竞争"和"保护领地"的思维普遍存在。 & 与企业和校际的合作活动开始在某些领域被主动推动,但尚未成体系。 & 学校主动领导标杆比对网络,与企业和国际伙伴进行战略性、可持续的研发合作。 \\
    \hline
    \end{tabular}
    }
\end{figure}

该矩阵显示,推动大学自主不仅是一项孤立的政策,更是一个能够"解锁"V-AQA模型全部五个要素潜力的杠杆。缺少这个杠杆,再好的改革建议也难以发挥全部作用。因此,完善法律走廊并扫除障碍,使各校能够充分且负责任地行使自主权,是国家在下一阶段的核心任务,为实质性地提升高等教育质量创造前提。

\subsection{风险分析与成功条件}
\label{subsec:risk_analysis_conditions}

实施像V-AQA这样全面和系统性的改革模型,必然会面临诸多风险和挑战。识别、分析并制定应对这些风险的策略,是确保改革过程成功的先决条件\footcite{kerzner_pm_2017}。本部分将重点关注三大风险群:(1)资源风险,(2)人与文化风险,以及(3)政策环境风险。

\subsubsection{风险分析与应对矩阵}
为获得一个全面和系统的视角,矩阵\ref{tab:risk_matrix}将总结主要风险,评估其影响程度和发生可能性,并提出相应的缓解措施。

% --- 已修改代码开始 ---
\begin{filecontents*}{risk_matrix.tex}
\begin{longtable}{|p{2.8cm}|>{\raggedright\arraybackslash}X|p{1.7cm}|p{1.7cm}|>{\raggedright\arraybackslash}X|}
\caption{实施V-AQA模型的风险分析与应对矩阵}
\label{tab:risk_matrix}\\
\hline
\textbf{风险类别} & \textbf{具体风险描述} & \textbf{发生可能性} & \textbf{影响程度} & \textbf{缓解/应对措施} \\
\hline
\endfirsthead
\multicolumn{5}{c}%
{{\bfseries \tablename\ \thetable{} -- 续前页}} \\
\hline
\textbf{风险类别} & \textbf{具体风险描述} & \textbf{发生可能性} & \textbf{影响程度} & \textbf{缓解/应对措施} \\
\hline
\endhead
\hline \multicolumn{5}{r}{{\textit{续下页}}} \\
\endfoot
\hline
\endlastfoot

% 风险1:资源
\textbf{1. 资源(财务、物质设施)} & 
缺乏预算投资于战略性项目,如构建质量保障管理信息系统、升级物质设施以及培训、奖励项目。 & 
高 & 
高 & 
- 制定详细的财务计划,分阶段实施。 \newline
- 多样化收入来源,推动公私合作伙伴关系以动员企业资源。 \newline
- 优先考虑开源技术解决方案以降低许可成本。 \\
\hline

% 风险2:人与文化
\textbf{2. 人与文化} & 
\begin{itemize}
    \item 干部、教师队伍的惰性、不愿改变的心态和怀疑态度。
    \item 感到利益受损的中层管理(暗中或公开)的抵制。
    \item 缺乏具备数据分析和现代治理能力的人力资源。
\end{itemize} & 
非常高 & 
非常高 & 
- 最高层领导的坚定政治承诺和模范作用。 \newline
- 持续、透明地宣传改革的益处。专注于创造"速赢"\footcite{kotter_1996}。 \newline
- 组织关于变革管理和新技能的强制性培训课程。 \newline
- 建立对创新努力的实质性表彰、奖励政策。 \\
\hline

% 风险3:政策
\textbf{3. 宏观政策环境} & 
自主政策与其他规定(财务、人事、公共投资)之间缺乏同步性,限制了各校的行动空间和能力。 & 
中 & 
非常高 & 
- 各大学需通过协会(例如:越南大学与学院协会)主动向教育培训部和政府提出基于证据的政策建议。 \newline
- 教育培训部主导,与相关部委协调,审查并移除不再适宜的政策障碍。 \\
\hline

\end{longtable}
\end{filecontents*}

% --- 开始:风险分析与应对矩阵 ---
\renewcommand{\arraystretch}{1.3} % 增加行距以便阅读

\begin{longtable}{|p{2.5cm}|p{5cm}|p{1.5cm}|p{1.5cm}|p{5cm}|}
\caption{实施V-AQA模型的风险分析与应对矩阵}
\label{tab:risk_matrix}\\
\hline
\textbf{风险类别} & \textbf{具体风险描述} & \textbf{可能性} & \textbf{影响程度} & \textbf{缓解/应对措施} \\
\hline
\endfirsthead
\multicolumn{5}{l}{\textit{(续表 \ref{tab:risk_matrix})}}\\\hline
\textbf{风险类别} & \textbf{具体风险描述} & \textbf{可能性} & \textbf{影响程度} & \textbf{缓解/应对措施} \\
\hline
\endhead
\hline \multicolumn{5}{r}{\textit{续下页}}\\
\endfoot
\hline
\endlastfoot

\textbf{1. 资源} &
缺乏用于质量保障管理信息系统(QA-MIS)、升级设施、培训和奖励的预算。 &
高 &
高 &
\begin{itemize}
  \item 分阶段实施 (phased implementation)。
  \item 收入来源多样化,推动公私合作(PPP)。
  \item 优先采用开源技术。
\end{itemize} \\
\hline

\textbf{2. 人员与文化} &
\begin{itemize}
  \item 干部的惰性、变革抵触和怀疑心态。
  \item 来自中层管理的抵制。
  \item 缺乏数据分析和现代治理专家。
\end{itemize} &
非常高 &
非常高 &
\begin{itemize}
  \item 高层领导的承诺和示范作用。
  \item 透明沟通,创造“短期胜利”。
  \item 关于变革管理的强制性培训。
  \item 对创新给予实质性奖励。
\end{itemize} \\
\hline

\textbf{3. 政策环境} &
自主权与财务、人事、公共投资等规定之间缺乏同步性。 &
中 &
非常高 &
\begin{itemize}
  \item 学校通过协会主动提出政策建议。
  \item 教育培训部与各部委协调,审查并消除政策障碍。
\end{itemize} \\
\hline

\end{longtable}
% --- 结束:风险分析与应对矩阵 ---
    

\subsubsection{成功条件的深入分析}

从上述风险矩阵中,可以得出对V-AQA模型改革过程成功至关重要的三个先决条件。

\paragraph{条件一:来自最高层的强有力且持久的政治承诺。}
这是最重要的因素。转型过程不可避免地会遇到困难、冲突和短期内的效率下降。如果没有来自校董会和校领导班子足够强大、一致和坚韧的承诺,所有改革努力都将轻易地在遇到初步压力时偏离方向或半途而废。这种承诺必须通过具体行动来体现,从率先垂范、分配相应资源到保护敢于创新的人。

\paragraph{条件二:构建一个系统的变革管理战略。}
不能凭主观意愿强加变革。学校需要有一个有效的内部沟通战略,清晰地解释"为什么要变革?","变革将为每个个人和整个组织带来什么好处?"。建立一个包括各单位有威望人士的"变革领导联盟",并专注于创造"速赢"以建立信任和动力,是已被证明有效的战略步骤\footcite{kotter_1996}。

\paragraph{条件三:对技术基础设施和人的能力进行相应投资。}
质量保障管理信息系统是模型的"硬脊梁",而团队的数据分析和现代治理能力是"软脊梁"。这两个因素必须得到并行和相应的投资。一个现代化的技术系统,如果用户没有足够的能力去利用它,将变得毫无用处。反之,一个即使受过良好培训的团队,如果没有数据和工具来工作,也无能为力。因此,一个针对这两个方面进行同步投资的计划,是转向基于证据的治理模式的强制性条件。

总之,实施V-AQA模型的道路充满挑战,但并非不可行。它需要战略远见、政治决心以及在变革管理中科学、系统方法的结合。


% het goi 11 12 chuong 5



\section{研究的局限性与未来研究方向}
\label{sec:hanche_huongnghiencuu}

每一项科学研究,无论多么精心,都因方法、范围和资源的限制而存在一定的局限性。坦诚地识别并呈现这些局限性并不会降低本论文的价值,相反,它体现了科学思维的成熟,并为后续研究开辟了新的道路。

\subsection{本论文的局限性}
\label{subsec:han_che_luan_an}

尽管已努力采用一种全面和系统的方法,本论文仍不可避免地存在以下一些核心局限性:

\begin{enumerate}
    \item \textbf{方法论与数据的局限性:} 本论文主要基于\textbf{定性研究方法}构建,主要数据来源是对二手文献(政策报告、统计数据、已发表的研究成果)的分析。
    \textit{该局限性的后果是},各项结论,特别是关于V-AQA模型中各要素之间因果关系的结论,主要带有逻辑推断和解释性质,尚未经过强有力的计量经济学模型检验。缺乏与政策制定者、大学领导和认证员的深度访谈等一手数据,也可能在一定程度上降低了分析的多维度性和深度。

    \item \textbf{研究范围的局限性:} 本论文集中于在宏观和中观层面(系统和教育机构)分析外部质量保障体系。
    \textit{该局限性的后果是},在各个院、系级别的内部质量保障体系的具体流程和多样化实践尚未得到详细考察。外部质量保障与内部质量保障这两个体系之间复杂的相互作用,本身就是一个重要的研究领域\footcite{eua_iqa_eqa_link},尚未得到充分阐明。

    \item \textbf{模型普适性的局限性:} V-AQA模型是作为一个可行的理论框架和解决方案被提出的。
    \textit{该局限性的后果是},该模型及其建议解决方案的实际有效性,需要通过未来的具体试点项目来验证。当应用于使命和背景截然不同的大学类型(例如:研究型大学与应用型大学)时,该模型的适用程度可能会有所不同。
\end{enumerate}

\subsection{未来的研究方向}
\label{subsec:huong_nghien_cuu_tuong_lai}

正是基于上述局限性以及本论文的研究成果,可以提出未来一些重要且具有紧迫性的研究方向:

\begin{enumerate}
    \item \textbf{对V-AQA模型假设的定量检验:} 当国家高等教育管理信息系统和校级质量保障管理信息系统完善并提供同步数据后,未来的研究可以采用计量经济学方法(例如:回归分析、结构方程模型)来科学地检验该模型的假设。研究问题可以包括:
        \begin{itemize}
            \item 财务自主程度对一所大学的国际发表数量有何影响?
            \item 治理质量(通过具体指标衡量)与学生就业率之间有何相关性?
        \end{itemize}
    回答这些问题将为政策制定者提供强有力的实证证据。

    \item \textbf{关于质量保障实施的深度比较案例研究:} 对不同类型学校实施质量保障模型(如V-AQA)的情况进行更深入的比较案例研究。一个包含四个典型案例的比较研究设计将非常有价值:(1)一所顶尖的、已实现自主的公立大学;(2)一所地方性的、仍严重依赖预算的公立大学;(3)一所非营利性私立大学;以及(44)一所营利性私立大学。该研究将有助于阐明组织背景(文化、资源、治理机制)在质量改革过程成功中的作用。

    \item \textbf{关于社会-行业组织角色的研究:} 更深入地分析越南行业协会参与培养项目认证和承认过程的潜力和障碍。这在越南是一个非常新的领域,但在世界范围内却是增强培养与劳动力市场需求契合度的重要趋势\footcite{CHEA_prof_acreditation}。
    
    \item \textbf{关于数字化转型对质量保障影响的研究:} 考察人工智能、大数据分析等新技术在高等教育质量管理和改进中的应用。人工智能如何被用于个性化学习路径、提前预测有辍学风险的学生,或自动化质量报告流程?这是一个充满突破前景的跨学科研究方向。
\end{enumerate}

%  pursuing these research directions will not only deepen the understanding of Vietnam's higher education system but also contribute to the global body of knowledge on educational governance in the context of globalization and the fourth industrial revolution.

\section*{总论}
\addcontentsline{toc}{section}{总论}
\label{sec:ket_luan_tong_the}

经过一番精心和系统的研究旅程,从剖析理论框架到深入分析现状和汲取国际经验,本论文已达到了其设定的核心目标:诊断越南高等教育质量保障体系的系统性问题,并提出了一个全面可行的改革模型。

本论文令人信服地证明,越南高等教育所面临的挑战——体现在规模增长与产出质量之间的\textbf{“发展悖论”}——并非孤立问题,而是系统性\textbf{“恶性循环”}的后果。这些循环是由仍然带有浓厚行政色彩的管理模式、应付式的质量文化、与利益相关者联系薄弱,特别是碎片化、低效的数据系统之间的相互作用所产生和巩固的。这一现状申明,零敲碎打、缺乏系统性的解决方案将无法带来可持续的改变。

在此背景下,本论文最重要和最独特的科学贡献在于成功构建并论证了\textbf{混合与适应性质量保障模型(V-AQA)}。这不仅是一个能够整合现代治理视角的新理论框架,更重要的是,它是一个全面的\textbf{“行动手册”},为重构质量保障体系提供了一条清晰的路线图。凭借其五个核心要素——(1)领导与治理、(2)质量文化、(3)利益相关者的参与、(4)内部流程,以及(5)合作与协调——V-AQA模型提供了一套同步的解决方案,旨在直接打破已识别的恶性循环。

该模型的\textbf{混合}哲学有助于协调来自外部的问责压力与来自内部的改进动力,而\textbf{适应性}原则确保了解决方案始终灵活并适应不断变化的背景。该模型的实施,以强有力的政治承诺、相应的资源投入和系统的变革管理战略为先决条件,有望创造一个\textbf{“良性循环”},在此循环中,各大学被赋予自主权,拥有内在的质量文化,并不断改进以更好地满足社会的要求。

总之,本论文不仅提供了对现状的深刻“诊断”,更提出了一个有科学和实践依据的“治疗方案”。V-AQA模型有望成为一项重要贡献,为越南的政策制定者和高等教育管理者在充满挑战但也充满希望的征途上,提供一个新的视角和一套有效的工具,以提升国家高等教育在一体化和知识经济时代的质量与竞争力。


% het goi 13 14 chuong 5, het chuong 5










