yyyyyyyyyyyyyyyy

\chapter{引言}
\label{chap:gioi_thieu}

\section{研究背景与选题理由}
\label{sec:boi_canh_ly_do}

\subsection{全球背景:质量保障成为必然趋势}

进入21世纪,高等教育(GDĐH)已超越其作为知识保存与传播场所的传统角色,成为决定各国竞争力和可持续发展的战略驱动力\footcite{Altbach2001, WB_KnowledgeEconomy}。在基于知识的全球化经济(knowledge-based economy)背景下,大学被视为“创新引擎”,是培养高素质人力资源、产出科学技术发明以及为社会复杂挑战提供解决方案的场所\footcite{OECD_HE2008}。这一角色的转变对高等教育体系的质量提出了前所未有的紧迫要求。质量不再是学术象牙塔中的抽象概念,而已成为一个可衡量、可比较的因素,是国际舞台上激烈竞争的标准。

正是在这样的背景下,质量保障体系(ĐBCL - Quality Assurance),特别是外部质量保障(External Quality Assurance - EQA),已成为不可逆转的全球趋势。根据世界银行(World Bank)和联合国教科文组织(UNESCO)的报告,在过去几十年里,建立国家级质量保障机构的国家数量急剧增加,覆盖了全球几乎所有地区\footcite{WorldBank_QA_GlobalTrends, UNESCO_QA2018}。这一强劲的扩散势头主要由三大动力推动:

\begin{enumerate}
    \item \textbf{高等教育的大众化 (Massification of Higher Education):} 学生规模和大学数量的爆炸性增长,包括私立院校的迅猛发展,使得高等教育体系变得前所未有的多样化和复杂化。这种多样性虽然带来了更多的教育机会,但同时也带来了质量下降和不均衡的风险。因此,政府和社会需要一个外部监督机制,以确保整个体系的最低质量门槛\footcite{Trow2007}。
    
    \item \textbf{问责制要求的提高 (Increased Demand for Accountability):} 随着公共财政和学习者对高等教育投入的日益增加,对大学问责制的要求也越来越高。包括政府、家长、学生和雇主在内的各利益相关方都想知道他们的投资是否带来了应有的效益。外部质量保障体系及其认证活动和结果公开,是履行这一问责制的最重要工具\footcite{Harvey2005}。
    
    \item \textbf{跨境教育的兴起 (Rise of Cross-border Education):} 全球化极大地促进了学生、教师和教育项目跨越国界的流动。这产生了对学位和学分互认的迫切需求。一个可靠的、其标准与国际惯例接轨的质量保障体系,是一个国家参与全球教育市场、吸引国际学生并确保本国学生文凭在国外获得承认的先决条件\footcite{Knight2006}。
\end{enumerate}

这些动力已将质量保障从一项纯粹的专业活动转变为一项重要的国家治理工具,是现代高等教育体系结构中不可或缺的要素。清晰理解这些全球趋势和动力,是准确分析像越南这样的特定国家质量保障体系所面临的背景与挑战的第一步。

\subsection{区域背景:东盟共同体内的质量协调化}
\label{subsec:boi_canh_khu_vuc}

如果说全球背景带来了竞争与融合的压力,那么区域背景则提出了合作与协调的要求。对越南而言,最重要的区域背景是东盟共同体。2015年东盟经济共同体(AEC)的成立设定了一个宏伟目标:建立一个统一的市场和生产基地,允许商品、服务、投资、资本以及特别是技术劳工(skilled labor)的自由流动\footcite{ASEAN_AEC_Blueprint}。

为实现这一目标,教育水平与质量的协调化及互认成为一项核心要求。在河内接受培训的工程师,需要具备与在曼谷或吉隆坡的雇主所承认的同等能力和文凭。深刻认识到这一要求,区域内的教育领导者们已率先努力建设一个共同的高等教育空间。实现这一努力最重要的工具是东盟大学网络(ASEAN University Network - AUN),特别是其下属的质量保障网络——AUN-QA(ASEAN University Network - Quality Assurance),该网络成立于1998年\footcite{AUNQA_History}。

AUN-QA并非一个具有强制权力的超国家认证机构,而是一个基于自愿、合作和同行学习(peer learning)原则运作的网络。其使命如官方文件所述,是“推动东盟高等教育机构的质量保障,提升高等教育质量,并为东盟共同体的共同利益与区域及国际机构合作”\footcite{AUN-QAGuide}。

AUN-QA的运作理念可概括为\textbf{“多元中的和谐”}原则。该网络不寻求消除各国教育体系之间的差异,而是建立一套共同的标准和评估流程,作为质量的“公分母”和通用语言。AUN-QA标准(目前是针对课程层级的4.0版)包含8项标准和明确的子标准,已成为该地区大学的重要参考框架。同行评审(peer review)流程,即由成员大学的专家相互评估,不仅有助于确保客观性,还创造了一个极其有效的经验和最佳实践分享机制\footcite{AUNQA_Report2023}。

对越南而言,积极参与AUN-QA网络具有战略意义。这不仅是提升国内大学质量和声誉的渠道,也是融入区域高等教育空间最快、最有效的途径。一个教育项目获得AUN-QA认证是其质量的重要保证,有助于该项目的毕业生在东盟劳动力市场上获得更多优势。因此,任何关于越南质量保障体系的分析都不能忽视AUN-QA框架作为区域背景重要因素的角色和影响。