Chắc chắn rồi, đây là nội dung đã được dịch sang tiếng Trung và định dạng lại theo yêu cầu của bạn:

% ======================================================================
% Chương 2
% ======================================================================
\chapter{高等教育质量保障的理论基础}
\label{chap:ly_luan}

% ======================================================================
% TRANG 1-2: GIỚI THIỆU CHƯƠNG
% ======================================================================
\section*{引言}
\addcontentsline{toc}{section}{引言}

在全球化和知识经济崛起的背景下,高等教育(GDĐH)的质量已成为决定每个国家竞争力的一个战略性因素。大学不再是与世隔绝的“象牙塔”,而已成为活跃的主体,受到来自社会、经济和政治等多重复杂压力的影响。为应对这些挑战,世界各地纷纷建立和发展了质量保障(ĐBCL)体系,特别是外部质量保障(External Quality Assurance - EQA)体系。然而,将发达国家的质量保障模式应用于像越南这样的转型经济体背景时,常会遇到诸多困难和矛盾。那些侧重于合规性控制或僵化标准的传统模式,显得不够灵活,难以解释和引导一个正处于快速发展和深刻变革阶段的高等教育体系。

这种复杂性对理论提出了一个迫切的要求:需要一个足够强大和全面的分析框架,以便能够“解剖”高等教育质量保障体系的本质。这样的理论框架不仅要能识别出影响因素,还必须能解释其动因、权力关系以及各主要主体——国家、认证机构和大学之间潜在的矛盾。现实表明,以往的许多研究通常只关注单一层面,或使用单一理论,导致了片面的看法。例如,一些研究可能很好地解释了为什么大学必须遵守国家规定,但却无法解释为什么培养方案在响应企业需求方面仍然变化缓慢。同样,一些分析侧重于问责制,却忽略了对组织行为有深远影响的文化和无形规范等因素。

这一“理论空白”正是本章的出发点。本论文认为,为了获得全面的理解,需要整合多种理论视角。具体而言,一个有效的分析框架必须能够同时解释三个核心问题:(1)为什么各大学会采用相似的质量保障结构和流程(\textit{关于合法性的问题})?(2)质量为谁定义和创造,以及如何平衡不同群体的利益(\textit{关于利益相关者的问题})?(3)采用何种机制来确保大学履行其承诺和责任(\textit{关于问责制的问题})?

为填补这一空白,本章将进行系统性的构建。\textbf{首先},本章将深入分析社会科学与公共管理的三大经典理论支柱:新制度主义、利益相关者理论和委托代理理论。分析将不仅停留在概念介绍,还将批判并指出每种理论在独立应用于高等教育领域时的局限性。\textbf{其次},本章将综述全球现代质量保障的发展趋势,特别是混合模型(Hybrid Model)和适应性框架(Adaptive Framework)的兴起,这些都是旨在超越传统理论局限的实践努力。\textbf{最后,也是最重要的},在这些分析的基础上,本章将提出并详细论证一个新的理论模型——\textbf{越南高等教育混合与适应性质量保障模型(V-AQA Model)}。该模型将作为核心的理论工具,为本论文后续章节的现状分析和方案建议提供指导和视角。


% done chuong 2 goi 1



























