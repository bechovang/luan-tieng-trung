% !Mode:: "TeX:UTF-8"

\ctitle{越南高等教育外部质量保证体系研究}
\etitle{Research on External Quality Assurance System of Vietnamese Higher Education}


\makeatother

% 学士学位封面
% \cbuyuanxi{天文系} % 部院系
% \czhuanye{天文学} % 专业
% \cxuehao{201511160109} % 学号
% \cxueshengxingming{某某某} % 学生姓名
% \czhidaojiaoshi{某某} % 指导教师
% \czhidaojiaoshizhicheng{教授} % 指导教师职称
% \czhidaojiaoshidanwei{北京师范大学天文系} % 指导教师单位

% 博士(硕士)学位封面
\czuozhe{张氏越祯} % 作者
\cdaoshi{高益民\ 教授} % 导师
\cxibienianji{教育学部2019级} % 系别年级
\cxuehao{201939010022} % 学号
\cxuekezhuanye{比较教育学} % 学科专业
\cwanchengriqi{2025年7月} % 完成日期


% 中文摘要
\begin{cabstract}
本论文系统研究了越南高等教育外部质量保障体系在快速扩张、国际化和问责要求不断提高背景下的转型与挑战。通过全面梳理全球与区域发展趋势,论文揭示了推动外部质量保障(EQA)机制发展的主要动力,特别是在东盟区域内。采用定性比较案例研究方法,深入分析了越南EQA体系,并借鉴了中国国家主导模式和东盟大学联盟质量保障(AUN-QA)框架的经验。论文运用现代管理与组织理论,包括新制度主义、利益相关者理论和委托代理理论,构建了适合越南国情的V-AQA混合与适应性模型。研究结果指出,越南高等教育在规模扩张、资源分配和人才培养与劳动力市场需求之间仍面临诸多挑战。论文最终提出了以领导力、质量文化、利益相关方参与、内部流程和合作为核心的EQA体系综合改革模型,旨在提升体系效能,促进质量持续改进,助力越南高等教育更好地融入区域与全球。
\end{cabstract}
\ckeywords{质量保障;高等教育;越南;外部质量保障;改革模型}

% English abstract
\begin{eabstract}
This dissertation investigates the transformation and challenges of Vietnam’s higher education quality assurance system in the context of rapid expansion, international integration, and increasing demands for accountability. Through a comprehensive review of global and regional trends, the study identifies the driving forces behind the development of external quality assurance (EQA) mechanisms, particularly in the ASEAN region. Using a qualitative comparative case study design, the research analyzes Vietnam’s EQA system in depth, drawing lessons from China’s state-driven model and the ASEAN University Network Quality Assurance (AUN-QA) framework. The study applies modern management and organizational theories—including new institutionalism, stakeholder theory, and principal-agent theory—to construct the V-AQA hybrid and adaptive model, tailored to Vietnam’s unique context. Findings highlight both achievements and persistent challenges, such as rapid scale growth, uneven resource distribution, and the gap between training outcomes and labor market needs. The dissertation concludes by proposing a comprehensive reform model for Vietnam’s EQA system, emphasizing leadership, quality culture, stakeholder engagement, internal processes, and cooperation. This model aims to enhance the system’s effectiveness, support sustainable quality improvement, and facilitate Vietnam’s integration into the regional and global higher education landscape.
\end{eabstract}
\ekeywords{quality assurance; higher education; Vietnam; external quality assurance; reform model}


