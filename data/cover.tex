% !Mode:: "TeX:UTF-8"

\ctitle{越南高等教育外部质量保证体系研究}
\etitle{Research on External Quality Assurance System of Vietnamese Higher Education}


\makeatother

% 学士学位封面
% \cbuyuanxi{天文系} % 部院系
% \czhuanye{天文学} % 专业
% \cxuehao{201511160109} % 学号
% \cxueshengxingming{某某某} % 学生姓名
% \czhidaojiaoshi{某某} % 指导教师
% \czhidaojiaoshizhicheng{教授} % 指导教师职称
% \czhidaojiaoshidanwei{北京师范大学天文系} % 指导教师单位

% 博士(硕士)学位封面
\czuozhe{张氏越祯} % 作者
\cdaoshi{高益民\ 教授} % 导师
\cxibienianji{教育学部2019级} % 系别年级
\cxuehao{201939010022} % 学号
\cxuekezhuanye{比较教育学} % 学科专业
\cwanchengriqi{2025年7月} % 完成日期


% 中文摘要
\begin{cabstract}
本论文系统研究了越南高等教育外部质量保障体系在快速扩张、国际化和问责要求不断提高背景下的转型与挑战。研究背景源于越南高等教育在过去二十年间的迅猛发展:从2000年的约200所高校、100万学生,发展到2023年的近500所高校、超过250万学生。这种规模扩张在带来机遇的同时,也带来了质量保障的严峻挑战,特别是在确保教育质量与数量增长同步、满足劳动力市场需求、以及应对国际竞争压力等方面。

通过全面梳理全球与区域发展趋势,论文揭示了推动外部质量保障(EQA)机制发展的主要动力,特别是在东盟区域内。研究发现,全球高等教育质量保障呈现出从单一合规导向向多元化、适应性模式转变的趋势。传统的以政府主导、标准化评估为主的质量保障模式正在被更加灵活、注重持续改进和利益相关者参与的混合模式所替代。在东盟地区,这种趋势表现得尤为明显,各国在保持本土特色的同时,也在积极寻求区域合作与标准协调。

研究采用定性比较案例研究方法,深入分析了越南EQA体系的现状、成就与挑战。通过对越南教育部质量保障政策文件、评估报告和专家访谈的系统分析,研究识别出越南EQA体系在制度建设、评估机制、国际合作等方面的主要成就。同时,研究也揭示了体系存在的关键问题,包括评估标准不够细化、评估专家队伍能力不足、评估结果应用不充分、以及利益相关者参与度不高等。

为了提供更全面的分析视角,研究借鉴了中国国家主导模式和东盟大学联盟质量保障(AUN-QA)框架的经验。中国案例展示了国家强力主导的质量保障体系在实现大规模、快速变革方面的优势,但也凸显了如何在严格控制与学术创新自主权之间取得平衡的挑战。AUN-QA案例则体现了基于合作、共识和同行网络的区域质量保障模式的力量,证明了在多样化环境中无需单一权力机构集中控制也能建立共同标准。

论文运用现代管理与组织理论,包括新制度主义、利益相关者理论和委托代理理论,构建了适合越南国情的V-AQA混合与适应性模型。该模型基于五个核心要素构建:(1)领导与治理,强调战略导向和有效治理结构;(2)质量文化,注重建立持续改进的组织文化;(3)利益相关者参与,确保多方利益的有效整合;(4)内部流程,建立系统化的质量保障机制;(5)合作与协调,促进内部协作和外部合作。这五个要素相互关联、动态互动,形成了一个完整的质量保障生态系统。

研究结果指出,越南高等教育在规模扩张、资源分配和人才培养与劳动力市场需求之间仍面临诸多挑战。具体表现在:教育质量与数量增长不同步,优质教育资源分布不均,人才培养与就业市场需求脱节,以及国际竞争力有待提升等方面。这些挑战的根源在于质量保障体系的不完善,特别是在评估标准、专家队伍、结果应用和利益相关者参与等方面存在不足。

基于理论分析和国际比较,论文最终提出了以领导力、质量文化、利益相关方参与、内部流程和合作为核心的EQA体系综合改革模型。该模型强调:(1)建立多层次、协调统一的质量保障治理体系;(2)培育以质量为核心的组织文化;(3)构建多元利益相关者参与机制;(4)完善系统化的内部质量保障流程;(5)加强国内协作和国际合作。这一改革模型旨在提升体系效能,促进质量持续改进,助力越南高等教育更好地融入区域与全球教育体系。

研究的理论贡献在于构建了适合发展中国家高等教育质量保障的分析框架,丰富了质量保障理论在转型国家背景下的应用。实践意义在于为越南高等教育质量保障体系改革提供了系统性的政策建议,同时也为其他类似背景的国家提供了有价值的参考。研究结论强调,有效的质量保障体系不仅是一套流程的集合,而是一个复杂的生态系统,需要领导力、文化、利益相关者、流程和合作之间的动态平衡。
\end{cabstract}
\ckeywords{质量保障;高等教育;越南;外部质量保障;改革模型}

% English abstract
\begin{eabstract}
This dissertation investigates the transformation and challenges of Vietnam's higher education quality assurance system in the context of rapid expansion, international integration, and increasing demands for accountability. The research background stems from Vietnam's remarkable higher education development over the past two decades: from approximately 200 institutions and 1 million students in 2000 to nearly 500 institutions and over 2.5 million students in 2023. While this scale expansion has brought opportunities, it has also created serious quality assurance challenges, particularly in ensuring synchronized quality and quantity growth, meeting labor market demands, and responding to international competitive pressures.

Through a comprehensive review of global and regional trends, the study identifies the driving forces behind the development of external quality assurance (EQA) mechanisms, particularly in the ASEAN region. The research reveals that global higher education quality assurance is transitioning from single compliance-oriented approaches to diversified, adaptive models. Traditional government-led, standardized assessment quality assurance models are being replaced by more flexible approaches that emphasize continuous improvement and stakeholder engagement. This trend is particularly evident in the ASEAN region, where countries maintain their local characteristics while actively seeking regional cooperation and standard harmonization.

Using a qualitative comparative case study design, the research analyzes Vietnam's EQA system in depth, examining its current status, achievements, and challenges. Through systematic analysis of Vietnam's Ministry of Education quality assurance policy documents, evaluation reports, and expert interviews, the study identifies major achievements in Vietnam's EQA system regarding institutional development, evaluation mechanisms, and international cooperation. Simultaneously, the research reveals critical issues within the system, including insufficiently detailed evaluation standards, inadequate capacity of evaluation expert teams, insufficient application of evaluation results, and low stakeholder participation levels.

To provide a more comprehensive analytical perspective, the study draws lessons from China's state-driven model and the ASEAN University Network Quality Assurance (AUN-QA) framework. The China case demonstrates the advantages of a state-dominated quality assurance system in achieving large-scale, rapid transformation, but also highlights the challenge of balancing strict control with academic innovation autonomy. The AUN-QA case exemplifies the power of a regional quality assurance model based on cooperation, consensus, and peer networks, proving that common standards can be established in diverse environments without centralized control by a single authority.

The study applies modern management and organizational theories—including new institutionalism, stakeholder theory, and principal-agent theory—to construct the V-AQA hybrid and adaptive model, tailored to Vietnam's unique context. This model is built on five core elements: (1) Leadership and Governance, emphasizing strategic orientation and effective governance structures; (2) Quality Culture, focusing on establishing continuous improvement organizational culture; (3) Stakeholder Engagement, ensuring effective integration of multiple interests; (4) Internal Processes, establishing systematic quality assurance mechanisms; and (5) Cooperation and Coordination, promoting internal collaboration and external cooperation. These five elements are interconnected and dynamically interactive, forming a complete quality assurance ecosystem.

Findings highlight both achievements and persistent challenges, such as rapid scale growth, uneven resource distribution, and the gap between training outcomes and labor market needs. Specific manifestations include: asynchronous quality and quantity growth in education, uneven distribution of quality educational resources, disconnection between talent cultivation and employment market demands, and room for improvement in international competitiveness. The root causes of these challenges lie in the imperfection of the quality assurance system, particularly in areas such as evaluation standards, expert teams, result application, and stakeholder participation.

Based on theoretical analysis and international comparison, the dissertation concludes by proposing a comprehensive reform model for Vietnam's EQA system, emphasizing leadership, quality culture, stakeholder engagement, internal processes, and cooperation. This model emphasizes: (1) establishing a multi-level, coordinated quality assurance governance system; (2) cultivating quality-centered organizational culture; (3) constructing multi-stakeholder participation mechanisms; (4) perfecting systematic internal quality assurance processes; and (5) strengthening domestic collaboration and international cooperation. This reform model aims to enhance the system's effectiveness, support sustainable quality improvement, and facilitate Vietnam's integration into the regional and global higher education landscape.

The theoretical contribution of this research lies in constructing an analytical framework suitable for higher education quality assurance in developing countries, enriching the application of quality assurance theory in the context of transitional countries. The practical significance lies in providing systematic policy recommendations for Vietnam's higher education quality assurance system reform, while also offering valuable references for other countries with similar backgrounds. The research conclusion emphasizes that an effective quality assurance system is not merely a collection of processes, but a complex ecosystem requiring dynamic balance among leadership, culture, stakeholders, processes, and cooperation.
\end{eabstract}
\ekeywords{quality assurance; higher education; Vietnam; external quality assurance; reform model}


