\begin{longtable}{|p{2.8cm}|>{\raggedright\arraybackslash}X|p{1.7cm}|p{1.7cm}|>{\raggedright\arraybackslash}X|}
\caption{实施V-AQA模型的风险分析与应对矩阵}
\label{tab:risk_matrix}\\
\hline
\textbf{风险类别} & \textbf{具体风险描述} & \textbf{发生可能性} & \textbf{影响程度} & \textbf{缓解/应对措施} \\
\hline
\endfirsthead
\multicolumn{5}{c}%
{{\bfseries \tablename\ \thetable{} -- 续前页}} \\
\hline
\textbf{风险类别} & \textbf{具体风险描述} & \textbf{发生可能性} & \textbf{影响程度} & \textbf{缓解/应对措施} \\
\hline
\endhead
\hline \multicolumn{5}{r}{{\textit{续下页}}} \\
\endfoot
\hline
\endlastfoot

% 风险1:资源
\textbf{1. 资源(财务、物质设施)} &
缺乏预算投资于战略性项目,如构建质量保障管理信息系统、升级物质设施以及培训、奖励项目。 &
高 &
高 &
- 制定详细的财务计划,分阶段实施。 \newline
- 多样化收入来源,推动公私合作伙伴关系以动员企业资源。 \newline
- 优先考虑开源技术解决方案以降低许可成本。 \\
\hline

% 风险2:人与文化
\textbf{2. 人与文化} &
\begin{itemize}
    \item 干部、教师队伍的惰性、不愿改变的心态和怀疑态度。
    \item 感到利益受损的中层管理(暗中或公开)的抵制。
    \item 缺乏具备数据分析和现代治理能力的人力资源。
\end{itemize} &
非常高 &
非常高 &
- 最高层领导的坚定政治承诺和模范作用。 \newline
- 持续、透明地宣传改革的益处。专注于创造“速赢”\footcite{kotter_1996}。 \newline
- 组织关于变革管理和新技能的强制性培训课程。 \newline
- 建立对创新努力的实质性表彰、奖励政策。 \\
\hline

% 风险3:政策
\textbf{3. 宏观政策环境} &
自主政策与其他规定(财务、人事、公共投资)之间缺乏同步性,限制了各校的行动空间和能力。 &
中 &
非常高 &
- 各大学需通过协会(例如:越南大学与学院协会)主动向教育培训部和政府提出基于证据的政策建议。 \newline
- 教育培训部主导,与相关部委协调,审查并移除不再适宜的政策障碍。 \\
\hline

\end{longtable}
